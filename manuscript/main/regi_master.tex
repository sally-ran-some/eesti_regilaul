%%%%%%%%%%%%%%%%%%%%%%%%%%%%%%%%%%%%%%%%%%%%%%%%%%%%%%%%%%%%%%%%%%%%%%
%%  disstemplate.tex, to be compiled with latex.		     %
%%  08 April 2002	Version 4				     %
%%%%%%%%%%%%%%%%%%%%%%%%%%%%%%%%%%%%%%%%%%%%%%%%%%%%%%%%%%%%%%%%%%%%%%
%%								     %
%%  Writing a Doctoral Dissertation with LaTeX at		     %
%%	the University of Texas at Austin			     %
\documentclass[12pt]{report}	% The documentclass must be ``report''.

\usepackage{utdiss2}  		% Dissertation package style file.

\usepackage{natbib}
%%%%%%%%%%%%%%%%%%%%%%%%%%%%%%%%%%%%%%%%%%%%%%%%%%%%%%%%%%%%%%%%%%%%%%
% Optional packages used for this sample dissertation. If you don't  %
% need a capability in your dissertation, feel free to comment out   %
% the package usage command.					     %
%%%%%%%%%%%%%%%%%%%%%%%%%%%%%%%%%%%%%%%%%%%%%%%%%%%%%%%%%%%%%%%%%%%%%%

\usepackage{amsmath,amsthm,amsfonts,amscd} 
				% Some packages to write mathematics.
\usepackage{eucal} 	 	% Euler fonts
\usepackage{verbatim}      	% Allows quoting source with commands.
\usepackage{makeidx}       	% Package to make an index.
%\usepackage{epsfig}         	% Allows inclusion of eps files.     	
%\usepackage{cyrchar}
%\usepackage{citesort}         	% 
\usepackage{url}		% Allows good typesetting of web URLs.
\usepackage{draftcopy}	
\usepackage{graphicx}
\usepackage{amsthm}
%\usepackage{amssymb}
%\usepackage{epstopdf}
%\DeclareGraphicsRule{.tif}{png}{.png}{`convert #1 `dirname #1`/`basename #1 .tif`.png}


\usepackage{booktabs}
\usepackage{topcapt}
\usepackage{subcaption}
%\usepackage{wrapfig}
\usepackage{fontspec}
%\usepackage[utf8]{inputenc}
%\usepackage{CharisSIL}
\usepackage{lilyglyphs}

\usepackage{leipzig}
\usepackage{gb4e}



\author{sally ransom}  	% Required

\address{305 E. 23rd Street STOP B5100\\ Austin, Texas 78712}  % Required

\title{becoming lyrics: how word prosody and musical meter negotiate the rhythmic terms of prominence}
                                                    % Required


%
% NOTE: Maximum three supervisors. Minimum one supervisor.
% NOTE: The Office of Graduate Studies will accept only two supervisors!
% 
%
\supervisor
	[Scott Myers]
	{Katrin Erk}

%%%%%%%%%%%%%%%%%%%%%%%%%%%%%%%%%%%%%%%%%%%%%

%%%%%%%%%%%%%%%%%%%%%%%%%%%%%%%%%%%%%%%%%%%%%%%%%%%%%%%%%%%%%%%%%%%%%%

\previousdegrees{B.A., Linguistics}
     % The abbreviated form of your previous degree(s).
     % E.g., \previousdegrees{B.S., MBA}.
     %
     % The default value is `B.S., M.S.'

%\graduationmonth{...}      
     % Graduation month, either May, August, or December, in the form
     % as `\graduationmonth{May}'. Do not abbreviate.
     %
     % The default value (either May, August, or December) is guessed
     % according to the time of running LaTeX.

%\graduationyear{...}   
     % Graduation year, in the form as `\graduationyear{2001}'.
     % Use a 4 digit (not a 2 digit) number.
     %
     % The default value is guessed according to the time of 
     % running LaTeX.

%\typist{...}       
     % The name(s) of typist(s), put `the author' if you do it yourself.
     % E.g., `\typist{Maryann Hersey and the author}'.
     %
     % The default value is `the author'.


%%%%%%%%%%%%%%%%%%%%%%%%%%%%%%%%%%%%%%%%%%%%%%%%%%%%%%%%%%%%%%%%%%%%%%
% Commands for master's theses and reports.			     %
%%%%%%%%%%%%%%%%%%%%%%%%%%%%%%%%%%%%%%%%%%%%%%%%%%%%%%%%%%%%%%%%%%%%%%
%
% If the degree you're seeking is NOT Doctor of Philosophy, uncomment
% (remove the % in front of) the following two command lines (the ones
% that have the \ as their second character).
%
\degree{MASTER OF ARTS}
\degreeabbr{M.A.}

% Uncomment the line below that corresponds to the type of master's
% document you are writing.
%
%\masterreport
\masterthesis


%%%%%%%%%%%%%%%%%%%%%%%%%%%%%%%%%%%%%%%%%%%%%%%%%%%%%%%%%%%%%%%%%%%%%%
% Some optional commands to change the document's defaults.	     %
%%%%%%%%%%%%%%%%%%%%%%%%%%%%%%%%%%%%%%%%%%%%%%%%%%%%%%%%%%%%%%%%%%%%%%
%
%\singlespacing
%\oneandonehalfspacing

%\singlespacequote
\oneandonehalfspacequote

\topmargin 0.125in	% Adjust this value if the PostScript file output
			% of your dissertation has incorrect top and 
			% bottom margins. Print a copy of at least one
			% full page of your dissertation (not the first
			% page of a chapter) and measure the top and
			% bottom margins with a ruler. You must have
			% a top margin of 1.5" and a bottom margin of
			% at least 1.25". The page numbers must be at
			% least 1.00" from the bottom of the page.
			% If the margins are not correct, adjust this
			% value accordingly and re-compile and print again.
			%
			% The default value is 0.125"

		% If you want to adjust other margins, they are in the
		% utdiss2-nn.sty file near the top. If you are using
		% the shell script Makediss on a Unix/Linux system, make
		% your changes in the utdiss2-nn.sty file instead of
		% utdiss2.sty because Makediss will overwrite any changes
		% made to utdiss2.sty.

%%%%%%%%%%%%%%%%%%%%%%%%%%%%%%%%%%%%%%%%%%%%%%%%%%%%%%%%%%%%%%%%%%%%%%
% Some optional commands to be tested.				     %
%%%%%%%%%%%%%%%%%%%%%%%%%%%%%%%%%%%%%%%%%%%%%%%%%%%%%%%%%%%%%%%%%%%%%%

% If there are 10 or more sections, 10 or more subsections for a section,
% etc., you need to make an adjustment to the Table of Contents with the
% command \longtocentry.
%
%\longtocentry 



%%%%%%%%%%%%%%%%%%%%%%%%%%%%%%%%%%%%%%%%%%%%%%%%%%%%%%%%%%%%%%%%%%%%%%
%	Some math support.					     %
%%%%%%%%%%%%%%%%%%%%%%%%%%%%%%%%%%%%%%%%%%%%%%%%%%%%%%%%%%%%%%%%%%%%%%
%
%	Theorem environments (these need the amsthm package)
%
 \theoremstyle{plain} %% This is the default

\newtheorem{thm}{Theorem}[section]
\newtheorem{cor}[thm]{Corollary}
\newtheorem{lem}[thm]{Lemma}
\newtheorem{prop}[thm]{Proposition}
\newtheorem{ax}{Axiom}

\theoremstyle{definition}
\newtheorem{defn}{Definition}[section]

\theoremstyle{remark}
\newtheorem{rem}{Remark}[section]
\newtheorem*{notation}{Notation}

\numberwithin{equation}{section}


%%%%%%%%%%%%%%%%%%%%%%%%%%%%%%%%%%%%%%%%%%%%%%%%%%%%%%%%%%%%%%%%%%%%%%
%	Macros.							     %
%%%%%%%%%%%%%%%%%%%%%%%%%%%%%%%%%%%%%%%%%%%%%%%%%%%%%%%%%%%%%%%%%%%%%%
%
%	Here some macros that are needed in this document:


\newcommand{\latexe}{{\LaTeX\kern.125em2%
                      \lower.5ex\hbox{$\varepsilon$}}}

\newcommand{\amslatex}{\AmS-\LaTeX{}}

\chardef\bslash=`\\	% \bslash makes a backslash (in tt fonts)
			%	p. 424, TeXbook

\newcommand{\cn}[1]{\texttt{\bslash #1}}

\makeatletter		% Starts section where @ is considered a letter
			% and thus may be used in commands.
\def\square{\RIfM@\bgroup\else$\bgroup\aftergroup$\fi
  \vcenter{\hrule\hbox{\vrule\@height.6em\kern.6em\vrule}%
                                              \hrule}\egroup}
\makeatother		% Ends sections where @ is considered a letter.
			% Now @ cannot be used in commands.

\makeindex    % Make the index

%%%%%%%%%%%%%%%%%%%%%%%%%%%%%%%%%%%%%%%%%%%%%%%%%%%%%%%%%%%%%%%%%%%%%%
%		The document starts here.			     %
%%%%%%%%%%%%%%%%%%%%%%%%%%%%%%%%%%%%%%%%%%%%%%%%%%%%%%%%%%%%%%%%%%%%%%

\begin{document}

\copyrightpage          % Produces the copyright page.


%
% NOTE: In a doctoral dissertation, the Committee Certification page
%		(with signatures) is BEFORE the Title page.
%	In a masters thesis or report, the Signature page
%		(with signatures) is AFTER the Title page.
%
%	If you are writing a masters thesis or report, you MUST REVERSE
%	the order of the \commcertpage and \titlepage commands below.

\titlepage              % Produces the title page.
%
\commcertpage           % Produces the Committee Certification
			%   of Approved Version page (doctoral)
			%   or Signature page (masters).
			%		





%%%%%%%%%%%%%%%%%%%%%%%%%%%%%%%%%%%%%%%%%%%%%%%%%%%%%%%%%%%%%%%%%%%%%%
% Dedication and/or epigraph are optional, but must occur here.      %
%%%%%%%%%%%%%%%%%%%%%%%%%%%%%%%%%%%%%%%%%%%%%%%%%%%%%%%%%%%%%%%%%%%%%%
%
\begin{dedication}
\index{Epigraph@\emph{Epigraph}}%

\begin{quotation}
{\scriptsize I think that when linguists discuss and dispute sound length and stress among themselves they would definitely benefit from inviting ethno-musicologists to join them. They could discuss the issues together, and not only based on written records but also sung records. \\ 
That what is written is fiction. Only that what is sung is truth. \\
\cite{tormis2007} }



\end{quotation}

\end{dedication}


%\begin{acknowledgments}		% Optional
%\index{Acknowledgments@\emph{Acknowledgments}}%
%I wish to thank the multitudes of people who helped me. Time would
%fail me to tell of \ldots
%\end{acknowledgments}


% The abstract is required. Note the use of ``utabstract'' instead of
% ``abstract''! This was necessary to fix a page numbering problem.
% The abstract heading is generated automatically.
% Do NOT use \begin{abstract} ... \end{abstract}.
%
\utabstract
\index{Abstract}%
\indent
%
%
%This paper analyses the acoustic realizations of sung vowels in Estonian {\it regilaul} lyrical folksong performances to examine the effects of musical metrical structure on the word-prosodic level of the language. Estonian is of particular interest as it has a fixed, predictable word stress pattern and three  syllable weights. In light of these features, I compare documented acoustic correlates of both stress and syllable weight, and the effects of them falling ``on" or ``off" the beat within the trochaic tetrametric pattern of {\it regilaul}. Vowel space and duration of syllable nuclei is compared in stressed and unstressed syllables falling on and off the beat in song, and all three quantities in word-prosodic stressed position examined on and off the beat in the song. 


 \tableofcontents   % Table of Contents will be automatically
                   % generated and placed here.

\listoftables      % List of Tables and List of Figures will be placed
\listoffigures     % here, if applicable.



%%%%%%%%%%%%%%%%%%%%%%%%%%%%%%%%%%%%%%%%%%%%%%%%%%%%%%%%%%%%%%%%%%%%%%
% Actual text starts here.					     %
%%%%%%%%%%%%%%%%%%%%%%%%%%%%%%%%%%%%%%%%%%%%%%%%%%%%%%%%%%%%%%%%%%%%%%
%


%
\chapter{Introduction}
\index{Introduction@\emph{Introduction}}%is

\section{On words becoming lyrics}
\subsection{song lyrics as the music-language interface}

In order to join music in song and become lyrics, language elements undergo modification to be fit into the musical structure. At the same time, the interpretability of those linguistic elements which constitute contrasts in semantic meanings has motivation for preservation. If the performer wishes the audience to perceive meaningful lyrical content, these contrasts must be crystallized at some level of salience in the acoustic signal. 

This project takes folksong performance as an opportunity to examine the interface of language and music. In particular, I investigate the acoustic manifestations of the word-prosodic features attested in spoken Estonian when they are sung in a temporally restricted domain such as music. 


both regilaul and Eesti have heirarchichal metrical systems by which long and short, stressed and unstressed constituents are arranged. In music, the smallest prosodic constituent is an individual note event whose relationship to the other notes in the song are indicated by the time signature, i.e., 3/4 or 4/4. The denominator corresponds to the number of divisions of a ``whole'' note, while the numerator refers to the number of ``beats'' in a single measure. So the prosodic heirarchy in music begins at the note level, then to each measure as constituent, larger phrases and motifs, and eventually the highest level which is the entire song. In Eesti, attested segments are organized into syllables, which in turn combine in strong-weak relationships to form prosodic feet, which make up the words of phrases and so on. 



 In a regilaul verse line, each syllable corresponds to an eighth note in the measure, with every odd positioned syllable-note coinciding with the ``beat'' onset in a typical 4/4 measure. Falling on the beat in music generally coincides with increased prominence compared to notes occurring ``off'' the beat. This is acoustically manifested in increased duration of notes on the beat compared to nominally isochronous notes in other positions. That is, in a measure containing eight eighth notes, the absolute durations of the four eighth notes corresponding to the beat will be longer than the absolute durations of the four eighth notes in even positions, while maintaining the perception of the nominal isochrony. 



%Acoustically, the difference between a stressed and an unstressed syllable is found not in a single cue but in a convergence of several cues: often duration, increased variability in f0, and increased vowel space, corresponding to a notion of localized hyperarticulation \citep{smiljanicProductionPerceptionClear2005, de1995supraglottal, lindblom1990}. 

How do the word-prosodic requirements negotiate with the imposed prosodic heirarchy of music? The rhythmic organization of song is said to integrate the prosodic structure of the language with musical rhythmic principles \citep{palmerLinguisticProsodyMusical1992}. However, earlier studies of {\it regilaul} explored temporal aspects of the songs and found that duration characteristics that would usually indicate important semantic differences lost their distinctions partially or entirely. Assuming that the intention of the singer is for the lyrics to be understood, I hypothesize that if some durational correlates of contrasting word-prosodic constituents are made less distinct in the process of compromising with the song that some other acoustic correlate of the relevant contrast at the word-prosodic level will be present, if not enhanced.  


%
%Whether there is a correlation between poetic metre and the prosodic structure of a language.\citep{lehistePhoneticsMetrics1992} investigated the actual phonetic realization of different metres in different languages, finding evidence 






%to paraphrase: Semantically relevant oppositions of word initial short-long and long-short disyllabic units in speech are not kept completely intact in folk songs. Short-long disyllables are treated in a different manner by the performer, depending on whether their initial syllable occurs at a rise or at a fall in a foot
\citep{rossTimingEstonianFolk1998} Analyzed and concluded that the {\it regilaul} lyrics are the result of an interaction between word and song prosodic heirarchies. This conclusion relied critically on measuring the durations of syllable-notes, where they found being on or off the beat was the better predictor for duration. \\


trade-off Q for S? 
\citep{rossTradeoffQuantityStress1996}. \\
They conclude that duration contrasts ordinarily present at the word level are {\it ``lost"} \citep{rossLostProsodicOppositions1994} to the temporal restrictions of the song. 

same song, timing and quality measured
\citep{rossFormants90}
\citep{rossStudyTimingEstonian1989a} \\

I argue that the cooperation of language in music is more than an interaction: that is, I argue that lyrical folksongs are examples of bidirectional {\it entrainment} of two independed oscillatory processes. I do not question the findings of Ross \& Lehiste, but wish to extend them. If the the syllable is required to fit in the prosodic level of the song, the linguistic contrasts of the words could be present either at a lower level of the prosodic heirarchy, i.e.,  the segment, or the contrast could be preserved in another acoustic dimension: i.e.,  vowel space dispersion. 


Null hypothesis: on or off-beat position is best predictor for vowel duration of syllables in all categories: stressed, unstressed, short, long, and overlong. This would mean that the syllable nucleus is proportional to the total syllable duration, and that the duration contrast is indeed ``lost." 

H_{1}: duration contrasts for syllable quantity will be evident in the vowel duration, with vowel duration {\it decreasing} as syllable weight increases. Given the findings of \citep{rosslehiste}, isochronous syllable-notes would result in heavier syllables having shorter nuclei to accomodate for the coda and complex codas that distinguish the syllable weights from each other. 

H_{2}: stress/unstress contrasts will be evident in the nucleus in terms of hypo and hyper articulation. For this I measure both nucleus duration and vowel space dispersion. 

Perceptual asymmetry: shortening not inducing MNN, but lengthening does \citep{eestiMNNasymmetry}. I expect that while syllable-notes that are on the beat will be prominent compared to those off the beat, unstressed syllables will be reduced compared to stressed syllables in the same position. 

\chapter{Methods}
\index{Methods@\emph{Methods}}%

\section{Design}
To test this hypothesis, the vowel durations of isochronous first and second syllable-notes of polysyllabic words are measured using a semi-automatic transcription approach. A tempograph of the entire song is referenced to identify whether or not a given syllable-note is `1on" or ``off" the beat relative to local and global tempo. 

Then, using the text transcriptions of the verse lines, the locations of syllable nuclei are identified using a forced aligner. 


Once located, the duration of a given vowel is defined as the difference between onset (within 10\% of the sustain level) of the ADSR envelope and the onset (within 10\%) of the release of the note. This is to account for temporal variation in syllable nuclei that arises due to environmental factors: consonant identities differing according to manner and place. Measuring nucleus duration by the sustain level subtracts the heavily coarticulated subsegmental transitions times that could be said to belong to both or either adjacent segments. 








%
%Within each song, samples of syllable-notes consisted only of those matching the nominal isochrony. For example, in verse lines mostly comprised of syllable-notes divided evenly into eighths, neither quarter notes nor sixteenth notes were taken. Syllables in phrase-final position were generally excluded under this criteria. 


\section{Constructing the Corpus}



\subsection{Materials}

Songs for this paper were accessed via The Anthology of Estonian Traditional Music \citep{tampere2016}, which includes audio recordings with transcriptions of the lyrics. 

%Originally published on four vinyl discs in 1970, the digital version showcases a robust sample of the massive collection of {\it regilaul} in Estonian Folklore Archives. In addition to audio, the  compilation includes  photographs, sheet music, and performer demographics of 98 {\it regilaul} songs and 17 instrumental tunes. 

The regilaul sample in this paper is a subset of the entire anthology. To control for regional differences \citep{whyRegion},  songs were all from Pärnumaa county. In order to account for diachronic variation \citep{whyDecade} as well as possible differences due to recording equipment used, all songs came from the same 5 year span from 1961-1966 and were recorded on behalf of the Estonian Folklore Archives by Herbert Tampere, Erna Tampere, and Ottilie Kõiva for the Estonian Folklore Archives\citep{oras2002, tampere2016}. 




\subsection{Annotating the Song Audio }

 Each song's lyrics are copied from the site and saved as .txt files in Estonian orthography, each line of the file corresponding to one melody line.  
Audio files of the selected songs are downloaded from the archive in .ogg format, which is the highest resolution of the two lossy formats available from the digital anthology. Each song is then imported into a Logic Pro X \citep{logic2014} session for preprocessing. 
Since a main predictor in the hypothesis refers to the notes with relation to the beat, it is necessary to annotate the location of beats for each measure in the song while accomodating the fine-grained variation in absolute duration and tempo inherent in natural meter. 
Beat onset detection in general works by analyzing the signal's acoustic dimensions with relation to the downbeat \citep{bKeeper2007}. 
Once the tempo map is aligned with the downbeat of each measure, Musical Instrument Digital Interface (MIDI) is used to synthesize a metronome to indicate beat location and duration with respect to the tempo variation in each measure. This metronome is converted into a PRAAT\citep{boersna2022} TextGrid with an interval tier indicating the onset and offset of the metronome's notes, which correspond to syllable-notes falling ``on" the beat. 


 
Lyrical transcriptions of each verse line is added to another interval tier, with each line corresponding to four measures indicated by the click tier.  Here, the eSpeak forced aligner for Estonian \citep{eSpeak1995} attempts to align the transcription from orthographic verse line to two tiers: one with phonemic transcriptions of words, and one with individual phonemes. Due to its inherent design for speech, the aligner frequently needed manual adjustment of the word boundaries before reliable phonemic alignment could be produced. 

A subset of vowel intervals from initial and pen-initial syllables of polysyllabic words is the data in this study. 



%The beginning of the vowel was aligned according to a combination of acoustic correlates: 
%
%
%
%\begin{itemize}
%\item vowel considerations
%\begin{itemize}
%\item {\bf intensity(dB) contours}: intensity within 2dB of the steady-state medial portion of the vowel, with a positive slope less than one.
%\item {\bf fundamental frequency(Hz)} stabilizing into that syllable-note's pitch category
%\item {\bf resonant frequencies F1, F2, and F3}: visible and approaching steady-state, combined wtih  the presence of a voicing bar.
%\end{itemize}
%\item onset manner particulars: 
%\begin{itemize}
%\item following {\bf plosive onsets}, vowel boundary is after a burst
%\item following {\bf fricative onsets},  the end of visible high-frequency noise in the spectrum was reliable. The phoneme /s/ also consistently showed a carat in the frequency track immediately preceding the transition to vowel.
%\item boundaries between {\bf approximant onsets} and vowel nuclei were determined chiefly by the steady state of formants. 
%\item following {\bf nasal onsets}, vowel intensity actually lowered, but a negative slope approaching zero reliably followed the offset of visible anti-formants.
%\end{itemize} 
%\item The offsets of vowels into codas:
%\begin{itemize}
%\item negative slopes approaching zero for intensity; positive preceding nasal. 
%\item Formants were allowed more variation in transitions to codas
%\item approximant codas {[l, j, w]} were excluded from the dataset, as neither the forced-aligner nor the phonetician could determine a reliable way to define the boundary between them.
%\end{itemize} 
%\end{itemize}
%
%
%
%In cases of vowel adjacency across syllable boundaries, the presence of a visible glottal stop and support from other criteria would qualify both for inclusion. In the absence of these cues, both nuclei were excluded from the measurements. Other exclusions were due to ambient noise (i.e., churchbells in song 41), ambiguity of word boundaries due to wordplay (consulted with native speaker), and cases of adjacent identical vowels across word boundaries. 
%%epenthesized vowels (i.e., pandi mind paju raiumaie) having mind(e) \\
%
%
%In all cases, if the aforementioned cues were unavailable, ambiguous, or misaligned, the token was elided for this analysis. From all nine songs in the corpus, a total of 757 vowel nuclei met the criteria for inclusion in duration measurements. 

%For the measurements of F1, F2 space, formants were extracted from the mid point of the vowel. 


\subsection{Aggregating Acoustic and Text data}

The last step in preprocessing is to integrate the audio annotation with the lexical content of the song lyrics. This task was accomplished using an open-source natural language toolkit in python called estinltk \url{https://github.com/estnltk} \citep{estnltk2020}. For each vowel interval included  in the study, the vowel's word context from the corresponding TextGrid tier is input into estnltk vabamorf syllabifier, which returns the word broken into individual syllables with corresponding prominence data for each syllable. Vowel duration is measured by the difference between the offset and the onset of the interval on the vowel's TextGrid tier. The estnltk data and PRAAT measurements are aggregated using the parselmouth library\citep{parselmouth2018, python1995}. Jupyter notebooks illustrating each step are available at  \url{https://github.com/sally-ran-some/eesti_regilaul}. 


%Linguistic descriptions of the Estonian language date back as early as the seventeenth century, but the ternary quantity contrast was not documented until native Estonian linguists contributed their intuitions. The non-native linguists had only described lexical stress \citep{sargEarlyHistoryEstonian2005a}. \\



\section{Statistical Analysis} 
For each hypothesis, linear mixed-effects models were fit using the lme4 package in R \citep{lme4,r2022}. All the models, random slopes and intercepts, comparisons, etc. 


Anova comparison of the maximal design model with a null model is also statistically significant (p<0.001***). 

\chapter{Results}
\index{Results@\emph{Results}}%

%\begin{wrapfigure}{L}{0.5\textwidth}
%\centering
%\includegraphics[width=0.5\textwidth]{/Users/sarah/Git/regilaul_project/manuscript/main/figures/quictus_durmdl.png}
%\caption{quantity model}
%\label{qmdlt}
%
%
%\end{wrapfigure}





The presence of a coda in comparison to an open syllable-note necessitates a shorter vowel to fit the consonant into the note duration. 




%\ref{qmdlt} shows the full output of the linear mixed effects model. 

%Regilaul's metrical pattern avoids short stressed syllables (Q1) in ictus position, preferring Q2 or Q3 (long and overlong) syllables to fall on the beat. In off-ictus position, short stressed (Q1) syllables are preferred while Q3 avoided.
%The ternary quantity contrast occurs only in primary stressed syllables. To analyze vowel duration measurements in all three quantities, a subset of only primary stressed syllables is taken from the dataset. 




%
%\begin{figure}[ht]
%\centering
%\includegraphics[width =\textwidth]{/Users/sarah/Git/regilaul_project/manuscript/results/q_dur.png}
%\caption{density plot of vowel durations in three syllable quantities}
%\label{qdur}
%
%\end{figure}
%\begin{wrapfigure}{l}{\textwidth}
%\centering
%\includegraphics[width =\textwidth]{/Users/sarah/Git/regilaul_project/manuscript/results/q_dur.png}
%\caption{density plot of vowel durations in three syllable quantities}
%\label{qdur}
%
%\end{wrapfigure}

%
%\ref{qdur} shows the vowel durations of all three syllable quantities, grouped by ictus and off-ictus positions in the song. In ictus position, median vowel duration descends as quantity increases, with the greatest difference between Q1 and Q2. Off the beat, a similar descending pattern is evident: however, in this case the largest difference is between Q2 and Q3 syllables. 
% The intercept is set at ictus position, Q1. Findings are significant results for Q2(p<0.001), Q3, and off-ictus positions (p< 0.05). Comparison with null model was statistically significant (p<0.001). For full model output see \ref{qdurfixed} and \ref{qdurrandoms} in Appendix A. In context of earlier findings that the quantity contrast was ``lost" at the syllable level, the decrease in vowel duration as syllable weight increases supports the notion that the contrast is preserved at the segmental level. That is, rather than the full syllable lengthening in duration, the song-level isochrony of syllable-notes results in vowel nuclei shortening to accommodate codas in Q2, and further for geminates and complex codas in Q3. 
% 
% 
%A null model constructed containing only random effects was compared to the design model by two-way ANOVA. Results are significant for the design model (p < 0.001***).


\section{Stress and Unstress}

\begin{figure}[ht]
\centering
\includegraphics[width = \textwidth]{/Users/sarah/Git/regilaul_project/manuscript/results/dur_density_qfac.png}
\caption{vowel durations of CV and CVC syllables in stressed and unstressed positions falling on and off the beat}
\label{durstrick}
\end{figure}

The two graphs in \ref{durstrick} illustrate the overall pattern of interaction between musical meter and word stress. When syllables are on the beat, the vowel duration normally assoficated with differences in word stress are not apparent. Off the beat, however, stress and unstress remain distinct. This pattern is the same in both open (CV) and closed (CVC) syllable shapes, with overall vowel duration being shorter in closed syllables. 


%In open syllables, ictus position predicts longer vowels in both stressed and unstressed syllables, while stressed syllables are longer overall than unstressed. 

%In Q2, we see longer vowel durations for ictus position in stressed syllables, and higher means for ictus position in unstressed, though the distributions overlap much more here. \\


%\begin{figure}[ht]
%\centering
%\includegraphics[width=\textwidth]{/Users/sarah/Git/regilaul_project/manuscript/results/dur_shape_ictus.png}
%\caption{vowel dur(s) by beat position and syll. shape}
%\label{ickdursh}
%\end{figure}


%
%\begin{figure}[ht]
%\centering
%\includegraphics[width=\textwidth]{/Users/sarah/Git/regilaul_project/manuscript/results/dur_shapestress.png}
%\caption{vowel dur(s) by word stress and syll. shape}
%\label{strdursh}

%\end{figure}



%
%The graph in \ref{ickdursh} illustrates the distribution of vowel durations in different syllable shapes falling on and off the beat. 
%
%A similar pattern can be seen in \ref{strdursh}, where the stressed or unstressed status is shown instead. At both song and word levels of prominence, CV or Q1 syllables are the longest, gradually decreasing in CVC and CVCC, both of which are Q2 syllables.  This further confirms the gradience of the quantity contrast at the segmental level. 


\begin{wrapfigure}{L}{0.5\textwidth}
\centering
\includegraphics[width=0.5\textwidth]{/Users/sarah/Git/regilaul_project/manuscript/main/CVmodout.png}

\caption{stress and ictus duration model output}
\label{strickmdl}
\end{wrapfigure}

%%%%%%%%%%%%%%%%%%%%%%%%%%%%%%%%%%%%%%%%%%

%
%%%%%%%%%%%%

%\subsection{Vowel Dispersion}
%A subset of the Q1 and Q2 vowels used for duration measurements above is taken,  containing only those five vowel phonemes which occur in both stressed and unstressed syllables at the word level: \(a, e, i, o, u\). The total number of vowels in this set is {\bf N}.
%To account for physiological differences between singers, vowel dispersion is calculated as the euclidean distance of each token from the respective singer's vowel center in the (F1, F2) space. 
%
%
%\begin{figure}[htb]
%\centering
%\includegraphics[width=\textwidth]{/Users/sarah/Git/regilaul_project/manuscript/results/space_strictus.png}
%\caption{euclidean distance of vowels in stress and ictus}
%\label{spcstrick}
%
%\end{figure}
%A pattern is marginally visible in the two graphs  \ref{spcstrick}, faceted by Q1 and Q2. Notice that in off-ictus position, vowel dispersion means of stressed syllables are higher than those of stressed syllables in ictus position. This indicates that stressed syllables falling off the beat are being compensated for their shortened vowels by way of increased articulation of quality. This pattern, however, doesn't shake out as statistically significant in the model. \\
%A linear mixed effects model for vowel dispersion is constructed, but only the uninterpretable intercept shows significance. 
%Comparison with null model is not statistically significant. Thus in the case of vowel dispersion, we fail to reject the null hypothesis. This could be due to the relatively smaller size of the data subset. 
%
%

%\chapter{Discussion}
\index{Discussion@\emph{Discussion}}%

\section{Temporal Prosodic Features Crystalized at Segmental level in isochronous syllables}

These results support the hypothesis that prosodic timing modifications resulting from the synchronization of independent rhythms (in this case, syllables and notes into syllable-notes) do not result in the subordination of one system to another. In this case, when syllables and notes become one timing unit, the temporal acoustic correlates often found at suprasegmental levels (syllable, foot, word) are found at the segmental levels. 

This interaction is then more analagous to entrainment of two independent rhythmic systems than to language being forcibly pigeon-holed into the metrical structure of music. 


\section{Vowel Dispersion}
Results for the predictive power of ictus and off-ictus, stressed and unstressed syllables and vowel space dispersion were not significant. The patterns that seem to emerge visually in the graphs of distributions suggest that the situation might be ameliorated by increasing the dataset. For this measurement, there were fewer available tokens than for vowel duration, and therefore less statistical power. Further examination is needed to determine whether or not the observed pattern is simply lacking in power or whether the premises that lead to this hypothesis are missing something entirely. 


\section{Future Studies} 

This study used a sample of nine songs and three singers, all recorded in the 1960s. Annotation has already begun on the remaining songs that fit into this sample's criteria: In total, there are seventeen songs and seven singers from Parnumaa county recorded in the 1960s. Increasing the size of the audio corpus would facilitate exploratory analysis in countless dimensions of music and language, and also shed light on the issue of vowel space unresolved here. 
Several of the singers featured in this sample set were also recorded speaking. Annotating their natural speech would provide a valuable contribution to the song corpus, as findings from the songs could be compared with speech of the same person.  

Extending the findings of vowel duration and the ternary quantity contrast has several obvious paths: synchronic analysis of with song samples from the same approximate time period but differing according to region, or even language: several other Balto-Finnic families have a trochaic tetrameter folksong tradition. Diachronic analysis with song samples of same singers in the same region at different points in time is another possibility with data from the Estonian Folklore Archives. Both these goals are achievable only with the continued annotation of the corpus of regilaul, which is quite demanding work. As I continue to build this corpus, I am also actively exploring ways in which to automate the process. The inclusion of beat tracking software eliminated much observer subjectivity, and also facilitated the forced-aligner: by automatically grouping verse lines into measures, the aligner was given phrase groupings to synthesize and compare, rather than attempting to align the entire song in a linear fashion. However, as the forced aligner used here was made specifically for speech, one way to improve the accuracy of the forced aligner (decreasing manual adjustment of annotations) would be to train a the aligner on sung material using supervised machine learning. As I plan to continue with the annotations either way, I can use the corrections I make as training material for the algorithm, with the hopeful result that the forced-aligner will eventually reach some threshold of accuracy, voiding the need for manual adjustments. 


\section{Conclusion}

This study examined fine-grained acoustic-phonetic features of Estonian prosody in the context of traditional folksongs known as regilaul. The data support the hypothesis that duration contrasts inherent in the ternary quantities of Estonian are still present at the segmental level, even after undergoing modification to fit syllables into isochronous notes of the song. The data also supports that the durational correlates of word-level stress are also crystallized and evident at the segmental level, so while it isn't lexically contrastive, the role of primary stress and unstress to mark word boundaries in spoken Estonian is still present in sung Estonian. Vowel quality patterns were measured but not significant for this dataset. 

This indicates a relationship more akin to the collaboration of two independent rhythmic systems rather than one rhythmic system dominating the other. Combined with the fact that any regilaul text can be sung to any existing regilaul melody brings into question whether {\it spoken} metrical verse text is truly independent of the temporal constraints of the musical meter they are made with, or somewhere between language and music. 

%\chapter{Conclusion}
\index{Conclusion@\emph{Conclusion}}%

%Vowel Duration: \\
%We failed to reject the null hypothesis for vowel duration: while the song level does dominate the temporal domain, there is an inverse relationship to vowel duration and word-level duration contrasts for Q1, Q3, and stressed syllables. So while the song always wins, there is clearly something else going on. It is possible that it is as simple as note-syllable isochrony, and the vowel decrease is simply an accommodation to give the ictus level the appropriate duration syllable. 
%
%Vowel Space: \\
%
%After confirming that song position is the dominant hierarchy for temporal constraints, we continue on to see how vowel space plays out in different levels of prosodic prominence. We are able to reject the null hypothesis with data to support a clear relationship between prominence and vowel space, this time at both prosodic levels (song and word.) This suggests that for the song, ``preserve contrast" ranks lower than the temporal constraints, but that so long as the strong positions in the song are satisfied, prominence can be indicated by vowel space at multiple levels of the prosodic hierarchy.  





%\include{chapter-makingbib}
%
%\include{chapter-tables+figs}
%
%\include{chapter-math}


%%%%%%%%%%%%%%%%%%%%%%%%%%%%%%%%%%%%%%%%%%%%%%%%%%%%%%%%%%%%%%%%%%%%%%
% Appendix/Appendices                                                %
%%%%%%%%%%%%%%%%%%%%%%%%%%%%%%%%%%%%%%%%%%%%%%%%%%%%%%%%%%%%%%%%%%%%%%
%
% If you have only one appendix, use the command \appendix instead
% of \appendices.
%%
%\appendices
%\index{Appendices@\emph{Appendices}}%
%%
%
\chapter{songs} 



\begin{figure}[htbp]
\begin{center}
\includegraphics[width=300pt]{figures/094.png}
\caption{Kuningamäng ``The King Game" performed by Liisa Kümmel}
\label{song 94}
\end{center}
\end{figure}


\begin{figure}[htbp]
\begin{center}
\includegraphics[width=300pt]{figures/009.png}
\caption{Põld lindudele performed by Marie Helimets}
\label{song 9 }
\end{center}
\end{figure}


\begin{figure}[htbp]
\begin{center}
\includegraphics[width=300pt]{figures/018.png}
\caption{Kadrilaul performed by Marie Helimets}
\label{song 18}
\end{center}
\end{figure}


\begin{figure}[htbp]
\begin{center}
\includegraphics[width=300pt]{figures/094.png}
\caption{default}
\label{94}
\end{center}
\end{figure}


\begin{figure}[htbp]
\begin{center}
\includegraphics[width=300pt]{figures/081.png}
\caption{Mehetapja (Maielaul) Greete Jents}
\label{81}
\end{center}
\end{figure}


\begin{figure}[htbp]
\begin{center}
\includegraphics[width=300pt]{figures/077.png}
\caption{{\it Loomine} performed by Liisu Orik }
\label{77}
\end{center}
\end{figure}

\begin{figure}[htbp]
\begin{center}
\includegraphics[width=300pt]{figures/069.png}
\caption{default}
\label{default}
\end{center}
\end{figure}

\begin{figure}[htbp]
\begin{center}
\includegraphics[width=300pt]{figures/045.png}
\caption{default}
\label{default}
\end{center}
\end{figure}






\begin{figure}[htbp]
\begin{center}
\includegraphics[width=300pt]{figures/055.png}
\caption{default}
\label{default}
\end{center}
\end{figure}



\begin{figure}[htbp]
\begin{center}
\includegraphics[width=300pt]{figures/045.png}
\caption{default}
\label{default}
\end{center}
\end{figure}
%\chapter{Estonian Folklore Archives}
\begin{figure}[htb]
\begin{center}
\includegraphics[width=300pt]{figures/vinyyl.png}
\caption{1970 cover art of vinyl anthology}
\label{vinyl1970}
\end{center}
\end{figure}

\begin{figure}[htp]
\centering
\includegraphics[width=0.9\linewidth]{figures/kihelkonnad}
\end{figure} 

%%song only
% latex table generated in R 4.1.2 by xtable 1.8-4 package
% Fri Aug 12 21:55:13 2022
\begin{longtable}{lrrrrrrrr}
  \toprule
Parameter & Rhat & n\_eff & mean & sd & se\_mean & 2.5\% & 50\% & 97.5\% \\ 
  \midrule
Intercept & 1.001 &  801 & 0.193 & 0.024 & 0.001 & 0.143 & 0.194 & 0.242 \\ 
  ictusoff & 1.000 & 4507 & -0.016 & 0.006 & 0.000 & -0.028 & -0.016 & -0.003 \\ 
  stressed1 & 1.001 & 4042 & 0.047 & 0.007 & 0.000 & 0.034 & 0.047 & 0.060 \\ 
  quantity2 & 1.002 & 3755 & -0.025 & 0.007 & 0.000 & -0.039 & -0.025 & -0.012 \\ 
  quantity3 & 1.002 & 3613 & -0.037 & 0.011 & 0.000 & -0.058 & -0.037 & -0.016 \\ 
  b[(Intercept) song:9] & 1.001 &  889 & 0.094 & 0.025 & 0.001 & 0.044 & 0.093 & 0.145 \\ 
  b[(Intercept) song:18] & 1.001 &  799 & -0.006 & 0.024 & 0.001 & -0.054 & -0.006 & 0.043 \\ 
  b[(Intercept) song:41] & 1.001 &  830 & 0.022 & 0.025 & 0.001 & -0.028 & 0.022 & 0.073 \\ 
  b[(Intercept) song:55] & 1.001 &  846 & -0.037 & 0.026 & 0.001 & -0.088 & -0.036 & 0.015 \\ 
  b[(Intercept) song:65] & 1.001 &  860 & 0.075 & 0.025 & 0.001 & 0.025 & 0.075 & 0.125 \\ 
  b[(Intercept) song:69] & 1.000 & 1211 & -0.042 & 0.031 & 0.001 & -0.105 & -0.042 & 0.016 \\ 
  b[(Intercept) song:77] & 1.001 &  993 & -0.105 & 0.027 & 0.001 & -0.159 & -0.105 & -0.051 \\ 
  b[(Intercept) song:92] & 1.001 &  989 & 0.033 & 0.027 & 0.001 & -0.020 & 0.034 & 0.087 \\ 
  b[(Intercept) song:94] & 1.002 &  878 & -0.039 & 0.025 & 0.001 & -0.089 & -0.039 & 0.011 \\ 
   \bottomrule
\caption{posterier parameters} 
\end{longtable}


%performer & word

% latex table generated in R 4.1.2 by xtable 1.8-4 package
% Fri Aug 12 19:59:47 2022
\begin{longtable}{lrrrrrrr}
  \toprule
Parameter & Rhat & n\_eff & mean & sd & 2.5\% & 50\% & 97.5\% \\ 
  \midrule
Intercept & 1.000 & 1871 & 0.197 & 0.036 & 0.122 & 0.197 & 0.270 \\ 
  ictusoff & 1.000 & 4383 & -0.022 & 0.007 & -0.035 & -0.022 & -0.008 \\ 
  stressed1 & 0.999 & 5726 & 0.050 & 0.006 & 0.038 & 0.050 & 0.062 \\ 
  quantity2 & 0.999 & 4036 & -0.029 & 0.007 & -0.043 & -0.029 & -0.015 \\ 
  quantity3 & 0.999 & 3106 & -0.036 & 0.013 & -0.062 & -0.036 & -0.010 \\ 
  b[(Intercept) word:ää] & 1.000 & 3916 & 0.143 & 0.043 & 0.058 & 0.142 & 0.228 \\ 
  b[(Intercept) word:aga] & 1.000 & 6184 & -0.065 & 0.034 & -0.133 & -0.065 & 0.001 \\ 
  b[(Intercept) word:ahju] & 1.000 & 8703 & -0.020 & 0.034 & -0.088 & -0.021 & 0.047 \\ 
  b[(Intercept) word:aia] & 1.000 & 7218 & 0.034 & 0.040 & -0.043 & 0.034 & 0.112 \\ 
  b[(Intercept) word:ajal,] & 1.000 & 7275 & 0.009 & 0.034 & -0.058 & 0.008 & 0.078 \\ 
  b[(Intercept) word:akkas] & 0.999 & 6375 & -0.032 & 0.034 & -0.101 & -0.032 & 0.035 \\ 
  b[(Intercept) word:allitagu,] & 0.999 & 7450 & 0.000 & 0.034 & -0.065 & 0.001 & 0.065 \\ 
  b[(Intercept) word:anisida,] & 1.000 & 7942 & 0.043 & 0.035 & -0.023 & 0.043 & 0.111 \\ 
  b[(Intercept) word:anna] & 0.999 & 7653 & 0.010 & 0.034 & -0.055 & 0.010 & 0.078 \\ 
  b[(Intercept) word:Anna,] & 1.000 & 7784 & -0.008 & 0.033 & -0.071 & -0.009 & 0.055 \\ 
  b[(Intercept) word:ärmätus] & 1.000 & 8103 & -0.021 & 0.034 & -0.086 & -0.020 & 0.043 \\ 
  b[(Intercept) word:Ega] & 1.000 & 6486 & -0.054 & 0.034 & -0.123 & -0.054 & 0.013 \\ 
  b[(Intercept) word:ei] & 1.000 & 6144 & -0.082 & 0.036 & -0.151 & -0.081 & -0.012 \\ 
  b[(Intercept) word:Ei] & 1.000 & 4695 & 0.120 & 0.043 & 0.037 & 0.120 & 0.203 \\ 
  b[(Intercept) word:esta!] & 0.999 & 7306 & -0.002 & 0.035 & -0.071 & -0.002 & 0.066 \\ 
  b[(Intercept) word:hääre] & 1.000 & 8670 & 0.017 & 0.034 & -0.051 & 0.018 & 0.084 \\ 
  b[(Intercept) word:halgu] & 1.000 & 8538 & -0.020 & 0.040 & -0.100 & -0.019 & 0.057 \\ 
  b[(Intercept) word:ilma,] & 0.999 & 8640 & 0.026 & 0.040 & -0.050 & 0.026 & 0.105 \\ 
  b[(Intercept) word:ilutse,] & 0.999 & 7301 & 0.021 & 0.034 & -0.047 & 0.021 & 0.088 \\ 
  b[(Intercept) word:ja] & 0.999 & 5985 & -0.067 & 0.031 & -0.129 & -0.066 & -0.008 \\ 
  b[(Intercept) word:jäi] & 0.999 & 7571 & -0.035 & 0.041 & -0.119 & -0.036 & 0.045 \\ 
  b[(Intercept) word:jalas] & 1.000 & 7026 & 0.029 & 0.034 & -0.036 & 0.029 & 0.095 \\ 
  b[(Intercept) word:jätan] & 1.000 & 7693 & 0.006 & 0.034 & -0.060 & 0.005 & 0.072 \\ 
  b[(Intercept) word:jätän] & 0.999 & 8537 & -0.005 & 0.035 & -0.074 & -0.005 & 0.063 \\ 
  b[(Intercept) word:Jätän] & 1.000 & 6625 & -0.011 & 0.034 & -0.078 & -0.011 & 0.058 \\ 
  b[(Intercept) word:jätan,] & 0.999 & 7348 & 0.050 & 0.034 & -0.015 & 0.051 & 0.116 \\ 
  b[(Intercept) word:juhate:] & 1.000 & 8309 & -0.022 & 0.034 & -0.086 & -0.022 & 0.043 \\ 
  b[(Intercept) word:juksimaie,] & 1.000 & 7291 & 0.023 & 0.033 & -0.041 & 0.023 & 0.088 \\ 
  b[(Intercept) word:julgust] & 1.000 & 6704 & -0.006 & 0.039 & -0.084 & -0.006 & 0.070 \\ 
  b[(Intercept) word:juure] & 0.999 & 6174 & 0.069 & 0.035 & -0.001 & 0.069 & 0.137 \\ 
  b[(Intercept) word:juuri] & 1.000 & 7542 & 0.031 & 0.032 & -0.032 & 0.031 & 0.094 \\ 
  b[(Intercept) word:Kadri] & 1.000 & 6946 & -0.070 & 0.035 & -0.138 & -0.070 & -0.001 \\ 
  b[(Intercept) word:kadril] & 1.000 & 7841 & 0.010 & 0.033 & -0.055 & 0.009 & 0.075 \\ 
  b[(Intercept) word:kadrisandi,] & 0.999 & 8674 & 0.004 & 0.027 & -0.049 & 0.004 & 0.058 \\ 
  b[(Intercept) word:kaelakardasida,] & 1.000 & 7210 & 0.003 & 0.033 & -0.064 & 0.003 & 0.068 \\ 
  b[(Intercept) word:käinud,] & 1.001 & 7140 & -0.042 & 0.034 & -0.108 & -0.042 & 0.025 \\ 
  b[(Intercept) word:käisid] & 0.999 & 7889 & -0.044 & 0.034 & -0.112 & -0.044 & 0.024 \\ 
  b[(Intercept) word:kakud] & 1.000 & 7280 & -0.058 & 0.034 & -0.124 & -0.057 & 0.008 \\ 
  b[(Intercept) word:kambren] & 1.000 & 7709 & -0.028 & 0.034 & -0.095 & -0.029 & 0.041 \\ 
  b[(Intercept) word:kana] & 1.000 & 8581 & 0.003 & 0.034 & -0.063 & 0.003 & 0.070 \\ 
  b[(Intercept) word:kanaliha\_–] & 1.000 & 7102 & 0.010 & 0.034 & -0.055 & 0.010 & 0.075 \\ 
  b[(Intercept) word:kandijaksa,] & 1.000 & 7895 & 0.004 & 0.034 & -0.063 & 0.003 & 0.071 \\ 
  b[(Intercept) word:kannustenu.] & 0.999 & 7820 & -0.021 & 0.034 & -0.087 & -0.021 & 0.045 \\ 
  b[(Intercept) word:kapi] & 1.000 & 7349 & -0.004 & 0.034 & -0.072 & -0.003 & 0.062 \\ 
  b[(Intercept) word:karask] & 1.000 & 5578 & -0.055 & 0.033 & -0.120 & -0.056 & 0.009 \\ 
  b[(Intercept) word:käsud] & 1.000 & 7050 & -0.022 & 0.034 & -0.090 & -0.021 & 0.044 \\ 
  b[(Intercept) word:kauge] & 0.999 & 9251 & 0.018 & 0.035 & -0.049 & 0.018 & 0.088 \\ 
  b[(Intercept) word:keelekene,] & 1.000 & 7304 & 0.035 & 0.035 & -0.032 & 0.035 & 0.101 \\ 
  b[(Intercept) word:kehval] & 1.000 & 6675 & -0.042 & 0.034 & -0.109 & -0.042 & 0.025 \\ 
  b[(Intercept) word:kevadisel] & 0.999 & 7105 & -0.015 & 0.034 & -0.080 & -0.015 & 0.051 \\ 
  b[(Intercept) word:kinni] & 1.000 & 7442 & -0.056 & 0.028 & -0.109 & -0.056 & -0.001 \\ 
  b[(Intercept) word:kinnikiilutud] & 1.000 & 8761 & 0.010 & 0.034 & -0.056 & 0.011 & 0.078 \\ 
  b[(Intercept) word:kirju] & 1.000 & 7611 & -0.035 & 0.034 & -0.106 & -0.034 & 0.032 \\ 
  b[(Intercept) word:kogemata,] & 1.000 & 5693 & 0.066 & 0.034 & -0.000 & 0.066 & 0.132 \\ 
  b[(Intercept) word:kohlapuida,] & 1.000 & 7280 & 0.043 & 0.034 & -0.023 & 0.044 & 0.109 \\ 
  b[(Intercept) word:kolmas] & 1.000 & 7639 & 0.010 & 0.040 & -0.068 & 0.009 & 0.091 \\ 
  b[(Intercept) word:korjama.] & 0.999 & 6943 & 0.009 & 0.034 & -0.058 & 0.008 & 0.077 \\ 
  b[(Intercept) word:kõrtsi] & 1.000 & 6535 & -0.033 & 0.033 & -0.099 & -0.033 & 0.030 \\ 
  b[(Intercept) word:kübara] & 0.999 & 8841 & 0.023 & 0.034 & -0.042 & 0.023 & 0.091 \\ 
  b[(Intercept) word:kübarast] & 0.999 & 7968 & 0.021 & 0.033 & -0.044 & 0.021 & 0.086 \\ 
  b[(Intercept) word:kui] & 0.999 & 6816 & 0.022 & 0.031 & -0.038 & 0.021 & 0.081 \\ 
  b[(Intercept) word:Kui] & 1.000 & 8489 & 0.028 & 0.040 & -0.052 & 0.028 & 0.109 \\ 
  b[(Intercept) word:kul´li] & 0.999 & 8654 & -0.016 & 0.035 & -0.083 & -0.016 & 0.050 \\ 
  b[(Intercept) word:Küll] & 0.999 & 5194 & -0.067 & 0.030 & -0.126 & -0.067 & -0.007 \\ 
  b[(Intercept) word:külmeteve,] & 1.000 & 8682 & -0.014 & 0.041 & -0.097 & -0.015 & 0.067 \\ 
  b[(Intercept) word:kuningas,] & 0.999 & 7216 & -0.042 & 0.034 & -0.108 & -0.041 & 0.025 \\ 
  b[(Intercept) word:kuningas!] & 1.000 & 7651 & -0.026 & 0.034 & -0.092 & -0.026 & 0.040 \\ 
  b[(Intercept) word:künnirauast] & 0.999 & 8192 & 0.009 & 0.033 & -0.056 & 0.009 & 0.075 \\ 
  b[(Intercept) word:künnirauda,] & 1.000 & 7732 & 0.016 & 0.034 & -0.050 & 0.017 & 0.085 \\ 
  b[(Intercept) word:kus] & 0.999 & 7103 & 0.019 & 0.040 & -0.060 & 0.019 & 0.097 \\ 
  b[(Intercept) word:küüdse] & 0.999 & 7862 & 0.012 & 0.034 & -0.054 & 0.013 & 0.080 \\ 
  b[(Intercept) word:laane] & 1.000 & 8693 & 0.018 & 0.034 & -0.050 & 0.018 & 0.084 \\ 
  b[(Intercept) word:läbi] & 0.999 & 7518 & -0.016 & 0.027 & -0.069 & -0.015 & 0.039 \\ 
  b[(Intercept) word:lapse] & 0.999 & 8175 & -0.019 & 0.034 & -0.083 & -0.019 & 0.050 \\ 
  b[(Intercept) word:lasandi.] & 0.999 & 8789 & -0.017 & 0.034 & -0.085 & -0.017 & 0.049 \\ 
  b[(Intercept) word:Laske] & 1.000 & 7286 & -0.019 & 0.032 & -0.082 & -0.018 & 0.046 \\ 
  b[(Intercept) word:lauade] & 0.999 & 6324 & -0.004 & 0.032 & -0.067 & -0.005 & 0.056 \\ 
  b[(Intercept) word:laudjas] & 1.000 & 6368 & -0.026 & 0.034 & -0.095 & -0.026 & 0.042 \\ 
  b[(Intercept) word:laula,] & 0.999 & 7582 & -0.016 & 0.031 & -0.077 & -0.016 & 0.044 \\ 
  b[(Intercept) word:Laula,] & 0.999 & 8017 & -0.015 & 0.031 & -0.076 & -0.015 & 0.046 \\ 
  b[(Intercept) word:Leidsine] & 1.000 & 7520 & -0.014 & 0.033 & -0.080 & -0.014 & 0.050 \\ 
  b[(Intercept) word:ligi] & 0.999 & 7267 & 0.055 & 0.035 & -0.014 & 0.055 & 0.126 \\ 
  b[(Intercept) word:ligunema.] & 0.999 & 6525 & 0.047 & 0.033 & -0.018 & 0.047 & 0.112 \\ 
  b[(Intercept) word:liigu,] & 1.000 & 9371 & 0.024 & 0.034 & -0.042 & 0.024 & 0.092 \\ 
  b[(Intercept) word:linakene,] & 1.000 & 7885 & 0.039 & 0.034 & -0.025 & 0.038 & 0.105 \\ 
  b[(Intercept) word:linnu] & 1.000 & 7560 & -0.016 & 0.034 & -0.085 & -0.016 & 0.052 \\ 
  b[(Intercept) word:linnukene,] & 0.999 & 8010 & 0.018 & 0.027 & -0.034 & 0.018 & 0.071 \\ 
  b[(Intercept) word:lipa-lapa.] & 1.000 & 7025 & -0.025 & 0.034 & -0.093 & -0.025 & 0.042 \\ 
  b[(Intercept) word:lõhna] & 1.000 & 7544 & -0.015 & 0.027 & -0.067 & -0.015 & 0.039 \\ 
  b[(Intercept) word:lõke] & 1.000 & 7632 & -0.015 & 0.035 & -0.082 & -0.015 & 0.052 \\ 
  b[(Intercept) word:lõmmu] & 0.999 & 8751 & -0.019 & 0.027 & -0.073 & -0.018 & 0.037 \\ 
  b[(Intercept) word:loogakirja,] & 0.999 & 8004 & 0.015 & 0.035 & -0.054 & 0.014 & 0.083 \\ 
  b[(Intercept) word:lõõtsu] & 1.000 & 7054 & -0.003 & 0.034 & -0.069 & -0.003 & 0.062 \\ 
  b[(Intercept) word:Lõpe,] & 0.999 & 7897 & 0.054 & 0.034 & -0.013 & 0.054 & 0.122 \\ 
  b[(Intercept) word:lumi] & 1.000 & 7642 & -0.003 & 0.033 & -0.068 & -0.002 & 0.063 \\ 
  b[(Intercept) word:ma] & 1.000 & 7741 & -0.034 & 0.028 & -0.091 & -0.034 & 0.020 \\ 
  b[(Intercept) word:maada] & 0.999 & 7348 & 0.042 & 0.033 & -0.020 & 0.042 & 0.108 \\ 
  b[(Intercept) word:mal´lika,] & 1.000 & 7942 & -0.018 & 0.041 & -0.101 & -0.018 & 0.063 \\ 
  b[(Intercept) word:mal´likast] & 0.999 & 7726 & -0.021 & 0.041 & -0.101 & -0.022 & 0.058 \\ 
  b[(Intercept) word:man´nipil´li,] & 0.999 & 8491 & -0.009 & 0.035 & -0.076 & -0.009 & 0.062 \\ 
  b[(Intercept) word:mära] & 1.000 & 6982 & -0.054 & 0.041 & -0.134 & -0.054 & 0.026 \\ 
  b[(Intercept) word:marja] & 0.999 & 8294 & 0.000 & 0.034 & -0.066 & 0.001 & 0.068 \\ 
  b[(Intercept) word:mätaste] & 1.000 & 7457 & -0.048 & 0.036 & -0.118 & -0.048 & 0.023 \\ 
  b[(Intercept) word:me] & 1.001 & 7056 & -0.016 & 0.041 & -0.095 & -0.016 & 0.065 \\ 
  b[(Intercept) word:meelekene,] & 1.000 & 6798 & 0.039 & 0.035 & -0.027 & 0.039 & 0.108 \\ 
  b[(Intercept) word:meile(e)i] & 1.000 & 7809 & -0.024 & 0.034 & -0.090 & -0.023 & 0.042 \\ 
  b[(Intercept) word:miks] & 1.000 & 5890 & -0.056 & 0.041 & -0.137 & -0.055 & 0.022 \\ 
  b[(Intercept) word:mind(e)] & 1.000 & 6749 & -0.085 & 0.035 & -0.152 & -0.085 & -0.018 \\ 
  b[(Intercept) word:minna,] & 0.999 & 9109 & -0.027 & 0.033 & -0.091 & -0.027 & 0.039 \\ 
  b[(Intercept) word:mis] & 1.000 & 8126 & -0.033 & 0.041 & -0.112 & -0.032 & 0.045 \\ 
  b[(Intercept) word:mõisa] & 1.000 & 8376 & 0.026 & 0.027 & -0.027 & 0.027 & 0.079 \\ 
  b[(Intercept) word:mõlgu,] & 0.999 & 7236 & -0.023 & 0.035 & -0.091 & -0.023 & 0.044 \\ 
  b[(Intercept) word:mõõduvakk.] & 0.999 & 7630 & 0.035 & 0.033 & -0.031 & 0.035 & 0.100 \\ 
  b[(Intercept) word:mullu] & 1.000 & 7892 & -0.020 & 0.041 & -0.097 & -0.020 & 0.061 \\ 
  b[(Intercept) word:murumullu] & 1.000 & 7215 & -0.049 & 0.034 & -0.116 & -0.049 & 0.019 \\ 
  b[(Intercept) word:näl´lasandi,] & 0.999 & 8850 & -0.007 & 0.041 & -0.084 & -0.007 & 0.073 \\ 
  b[(Intercept) word:neede] & 0.999 & 6197 & -0.028 & 0.029 & -0.085 & -0.029 & 0.027 \\ 
  b[(Intercept) word:neidusida,] & 1.000 & 7910 & 0.044 & 0.040 & -0.035 & 0.043 & 0.121 \\ 
  b[(Intercept) word:neljas] & 0.999 & 8450 & 0.028 & 0.035 & -0.039 & 0.027 & 0.097 \\ 
  b[(Intercept) word:Nii] & 0.999 & 8927 & 0.021 & 0.040 & -0.060 & 0.021 & 0.098 \\ 
  b[(Intercept) word:noori] & 0.999 & 7464 & 0.049 & 0.034 & -0.018 & 0.048 & 0.114 \\ 
  b[(Intercept) word:nüüd] & 0.999 & 7869 & -0.010 & 0.040 & -0.091 & -0.009 & 0.066 \\ 
  b[(Intercept) word:Nüüd] & 0.999 & 7708 & -0.012 & 0.040 & -0.091 & -0.013 & 0.067 \\ 
  b[(Intercept) word:obese.] & 1.000 & 7186 & -0.048 & 0.033 & -0.115 & -0.048 & 0.016 \\ 
  b[(Intercept) word:Obu] & 0.999 & 7521 & -0.055 & 0.034 & -0.123 & -0.055 & 0.010 \\ 
  b[(Intercept) word:õel] & 1.000 & 6463 & 0.045 & 0.034 & -0.023 & 0.044 & 0.113 \\ 
  b[(Intercept) word:ojasse,] & 0.999 & 8936 & -0.003 & 0.034 & -0.070 & -0.003 & 0.061 \\ 
  b[(Intercept) word:ole] & 0.999 & 6739 & -0.070 & 0.025 & -0.120 & -0.070 & -0.021 \\ 
  b[(Intercept) word:olem] & 1.001 & 5195 & -0.096 & 0.035 & -0.165 & -0.095 & -0.026 \\ 
  b[(Intercept) word:oli] & 1.000 & 5262 & -0.096 & 0.019 & -0.135 & -0.096 & -0.059 \\ 
  b[(Intercept) word:olid] & 1.000 & 5432 & -0.084 & 0.025 & -0.133 & -0.083 & -0.034 \\ 
  b[(Intercept) word:om] & 1.000 & 5471 & -0.081 & 0.026 & -0.131 & -0.081 & -0.030 \\ 
  b[(Intercept) word:Ööd] & 1.001 & 5257 & 0.109 & 0.041 & 0.028 & 0.108 & 0.191 \\ 
  b[(Intercept) word:oodakuta!] & 1.000 & 6925 & 0.052 & 0.035 & -0.016 & 0.052 & 0.120 \\ 
  b[(Intercept) word:ööles] & 1.000 & 5605 & 0.089 & 0.035 & 0.021 & 0.089 & 0.157 \\ 
  b[(Intercept) word:öösed] & 0.999 & 8055 & 0.032 & 0.034 & -0.035 & 0.032 & 0.099 \\ 
  b[(Intercept) word:orjad,] & 0.999 & 6709 & -0.010 & 0.035 & -0.078 & -0.011 & 0.056 \\ 
  b[(Intercept) word:orjal] & 0.999 & 7986 & 0.043 & 0.040 & -0.035 & 0.042 & 0.121 \\ 
  b[(Intercept) word:otsa,] & 1.000 & 7429 & 0.011 & 0.033 & -0.055 & 0.012 & 0.076 \\ 
  b[(Intercept) word:õvve] & 1.000 & 6891 & -0.009 & 0.026 & -0.058 & -0.009 & 0.042 \\ 
  b[(Intercept) word:pääl.] & 0.999 & 8652 & 0.009 & 0.040 & -0.069 & 0.009 & 0.087 \\ 
  b[(Intercept) word:pääle] & 1.000 & 6486 & 0.073 & 0.035 & 0.005 & 0.073 & 0.140 \\ 
  b[(Intercept) word:päälta,] & 1.000 & 7009 & 0.008 & 0.033 & -0.055 & 0.007 & 0.071 \\ 
  b[(Intercept) word:päälta!] & 1.000 & 7161 & 0.010 & 0.040 & -0.067 & 0.010 & 0.090 \\ 
  b[(Intercept) word:paari] & 1.001 & 7213 & -0.006 & 0.034 & -0.072 & -0.006 & 0.060 \\ 
  b[(Intercept) word:pääsukesel] & 1.000 & 6909 & 0.071 & 0.034 & 0.004 & 0.070 & 0.137 \\ 
  b[(Intercept) word:päivä] & 1.000 & 6670 & 0.048 & 0.034 & -0.019 & 0.048 & 0.114 \\ 
  b[(Intercept) word:paju] & 1.000 & 6496 & 0.058 & 0.034 & -0.007 & 0.058 & 0.124 \\ 
  b[(Intercept) word:palada,] & 1.000 & 7578 & 0.030 & 0.034 & -0.036 & 0.030 & 0.099 \\ 
  b[(Intercept) word:pandi] & 0.999 & 7285 & -0.036 & 0.035 & -0.106 & -0.036 & 0.031 \\ 
  b[(Intercept) word:pani] & 1.000 & 6068 & -0.053 & 0.035 & -0.122 & -0.054 & 0.013 \\ 
  b[(Intercept) word:partisida,] & 1.000 & 7431 & -0.031 & 0.035 & -0.100 & -0.032 & 0.038 \\ 
  b[(Intercept) word:peks(a)\_–] & 1.001 & 6348 & -0.041 & 0.034 & -0.106 & -0.040 & 0.026 \\ 
  b[(Intercept) word:penki!] & 1.000 & 7572 & -0.009 & 0.035 & -0.077 & -0.009 & 0.058 \\ 
  b[(Intercept) word:Perenaine,] & 1.000 & 6371 & 0.071 & 0.028 & 0.016 & 0.071 & 0.126 \\ 
  b[(Intercept) word:petsariinist] & 1.000 & 7436 & -0.003 & 0.034 & -0.070 & -0.003 & 0.063 \\ 
  b[(Intercept) word:piakene?] & 1.000 & 7625 & 0.020 & 0.034 & -0.045 & 0.020 & 0.085 \\ 
  b[(Intercept) word:pihta] & 0.999 & 7090 & -0.036 & 0.033 & -0.101 & -0.036 & 0.027 \\ 
  b[(Intercept) word:piigakesi.] & 1.000 & 6129 & 0.080 & 0.035 & 0.015 & 0.080 & 0.150 \\ 
  b[(Intercept) word:pirdu] & 1.000 & 7710 & -0.033 & 0.027 & -0.085 & -0.032 & 0.019 \\ 
  b[(Intercept) word:pistussenna,] & 0.999 & 7848 & -0.013 & 0.033 & -0.079 & -0.013 & 0.050 \\ 
  b[(Intercept) word:pistussesta,] & 0.999 & 8949 & -0.018 & 0.034 & -0.085 & -0.018 & 0.048 \\ 
  b[(Intercept) word:põrmandulle,] & 1.000 & 7597 & -0.023 & 0.034 & -0.092 & -0.023 & 0.044 \\ 
  b[(Intercept) word:pühkijaksa,] & 1.000 & 8211 & -0.028 & 0.026 & -0.079 & -0.027 & 0.025 \\ 
  b[(Intercept) word:puhu] & 0.999 & 7510 & 0.019 & 0.034 & -0.048 & 0.019 & 0.084 \\ 
  b[(Intercept) word:puksu] & 1.000 & 7147 & -0.009 & 0.034 & -0.079 & -0.009 & 0.057 \\ 
  b[(Intercept) word:punapõs\_ki] & 1.000 & 5494 & 0.083 & 0.034 & 0.017 & 0.083 & 0.149 \\ 
  b[(Intercept) word:räästa] & 0.999 & 8138 & -0.023 & 0.034 & -0.090 & -0.022 & 0.044 \\ 
  b[(Intercept) word:rabadaie.] & 1.000 & 6674 & 0.059 & 0.034 & -0.008 & 0.059 & 0.127 \\ 
  b[(Intercept) word:raiumaie,] & 0.999 & 7297 & 0.056 & 0.034 & -0.010 & 0.057 & 0.122 \\ 
  b[(Intercept) word:ranitsa] & 0.999 & 8436 & -0.000 & 0.033 & -0.066 & -0.001 & 0.066 \\ 
  b[(Intercept) word:ranitsast] & 1.000 & 6332 & -0.003 & 0.034 & -0.067 & -0.002 & 0.063 \\ 
  b[(Intercept) word:rapsuvitsa] & 1.000 & 7175 & -0.015 & 0.033 & -0.079 & -0.015 & 0.049 \\ 
  b[(Intercept) word:riide'eida,] & 0.999 & 8611 & -0.011 & 0.040 & -0.090 & -0.011 & 0.070 \\ 
  b[(Intercept) word:riimuka,] & 0.999 & 7603 & 0.011 & 0.033 & -0.054 & 0.012 & 0.074 \\ 
  b[(Intercept) word:riimukast] & 0.999 & 7402 & -0.019 & 0.033 & -0.084 & -0.019 & 0.048 \\ 
  b[(Intercept) word:riisud] & 0.999 & 7643 & -0.020 & 0.035 & -0.090 & -0.020 & 0.049 \\ 
  b[(Intercept) word:rikkus] & 0.999 & 7122 & -0.029 & 0.027 & -0.082 & -0.029 & 0.022 \\ 
  b[(Intercept) word:rindasõl´ge,] & 0.999 & 8130 & -0.019 & 0.034 & -0.086 & -0.019 & 0.048 \\ 
  b[(Intercept) word:rindasõlesta] & 0.999 & 8144 & -0.009 & 0.034 & -0.077 & -0.010 & 0.060 \\ 
  b[(Intercept) word:roogu] & 1.000 & 8113 & 0.014 & 0.034 & -0.052 & 0.013 & 0.082 \\ 
  b[(Intercept) word:sa] & 1.000 & 5825 & -0.071 & 0.035 & -0.141 & -0.071 & -0.003 \\ 
  b[(Intercept) word:saagu,] & 1.000 & 7776 & -0.029 & 0.027 & -0.083 & -0.030 & 0.024 \\ 
  b[(Intercept) word:Saagu,] & 0.999 & 8214 & 0.007 & 0.027 & -0.047 & 0.007 & 0.060 \\ 
  b[(Intercept) word:saaniteki,] & 0.999 & 7846 & 0.026 & 0.035 & -0.042 & 0.026 & 0.094 \\ 
  b[(Intercept) word:saare] & 1.000 & 8585 & 0.056 & 0.034 & -0.010 & 0.056 & 0.124 \\ 
  b[(Intercept) word:saarekirja.] & 0.999 & 6118 & 0.084 & 0.036 & 0.016 & 0.083 & 0.154 \\ 
  b[(Intercept) word:sadu] & 0.999 & 8344 & 0.026 & 0.034 & -0.039 & 0.027 & 0.093 \\ 
  b[(Intercept) word:sain] & 0.999 & 5427 & -0.040 & 0.023 & -0.085 & -0.040 & 0.004 \\ 
  b[(Intercept) word:sajad] & 1.000 & 7673 & -0.042 & 0.034 & -0.108 & -0.041 & 0.023 \\ 
  b[(Intercept) word:sajatan:] & 1.001 & 5854 & -0.027 & 0.027 & -0.082 & -0.028 & 0.026 \\ 
  b[(Intercept) word:san´dil] & 1.000 & 7068 & -0.028 & 0.040 & -0.106 & -0.028 & 0.050 \\ 
  b[(Intercept) word:sarapuida,] & 1.000 & 7437 & 0.057 & 0.033 & -0.007 & 0.057 & 0.122 \\ 
  b[(Intercept) word:see] & 1.000 & 6842 & -0.007 & 0.040 & -0.086 & -0.007 & 0.072 \\ 
  b[(Intercept) word:seenetagu,] & 0.999 & 8658 & 0.010 & 0.034 & -0.055 & 0.010 & 0.076 \\ 
  b[(Intercept) word:sei] & 1.000 & 7735 & 0.035 & 0.042 & -0.045 & 0.035 & 0.116 \\ 
  b[(Intercept) word:siia] & 0.999 & 7259 & 0.003 & 0.026 & -0.049 & 0.003 & 0.054 \\ 
  b[(Intercept) word:siin] & 1.000 & 7308 & 0.034 & 0.034 & -0.033 & 0.033 & 0.100 \\ 
  b[(Intercept) word:Siin] & 1.001 & 6517 & 0.026 & 0.040 & -0.052 & 0.025 & 0.103 \\ 
  b[(Intercept) word:siis] & 0.999 & 7207 & -0.026 & 0.039 & -0.104 & -0.025 & 0.053 \\ 
  b[(Intercept) word:sina] & 1.000 & 7707 & -0.062 & 0.027 & -0.114 & -0.062 & -0.010 \\ 
  b[(Intercept) word:sinu] & 1.000 & 5954 & -0.087 & 0.034 & -0.153 & -0.087 & -0.021 \\ 
  b[(Intercept) word:sipa-sopa,] & 0.999 & 7016 & -0.021 & 0.035 & -0.089 & -0.020 & 0.047 \\ 
  b[(Intercept) word:sissi] & 0.999 & 7492 & -0.015 & 0.027 & -0.067 & -0.016 & 0.036 \\ 
  b[(Intercept) word:sõitsid] & 1.000 & 6096 & -0.039 & 0.040 & -0.119 & -0.039 & 0.038 \\ 
  b[(Intercept) word:sõmera,] & 0.999 & 7193 & 0.049 & 0.035 & -0.018 & 0.049 & 0.116 \\ 
  b[(Intercept) word:sõmerast] & 1.000 & 6709 & 0.032 & 0.032 & -0.031 & 0.032 & 0.095 \\ 
  b[(Intercept) word:sõnumida,] & 0.999 & 9087 & -0.017 & 0.035 & -0.084 & -0.017 & 0.049 \\ 
  b[(Intercept) word:soo] & 0.999 & 7846 & 0.006 & 0.040 & -0.071 & 0.006 & 0.087 \\ 
  b[(Intercept) word:soolavakast] & 0.999 & 8291 & 0.021 & 0.034 & -0.042 & 0.020 & 0.087 \\ 
  b[(Intercept) word:soolavakka,] & 0.999 & 7589 & 0.022 & 0.034 & -0.042 & 0.022 & 0.087 \\ 
  b[(Intercept) word:südämekene!] & 1.000 & 6147 & -0.002 & 0.035 & -0.072 & -0.002 & 0.068 \\ 
  b[(Intercept) word:südant] & 0.999 & 7899 & 0.027 & 0.035 & -0.042 & 0.026 & 0.094 \\ 
  b[(Intercept) word:süia,] & 1.000 & 5080 & 0.096 & 0.035 & 0.028 & 0.096 & 0.165 \\ 
  b[(Intercept) word:sukad] & 1.000 & 8042 & 0.023 & 0.034 & -0.043 & 0.023 & 0.090 \\ 
  b[(Intercept) word:sulased,] & 0.999 & 7266 & -0.003 & 0.035 & -0.072 & -0.003 & 0.064 \\ 
  b[(Intercept) word:sülle] & 1.000 & 6951 & 0.044 & 0.040 & -0.033 & 0.044 & 0.124 \\ 
  b[(Intercept) word:suukene,] & 0.999 & 6005 & 0.093 & 0.035 & 0.025 & 0.093 & 0.163 \\ 
  b[(Intercept) word:tää,] & 0.999 & 7455 & 0.021 & 0.041 & -0.059 & 0.020 & 0.100 \\ 
  b[(Intercept) word:taevudes,] & 0.999 & 6860 & 0.000 & 0.031 & -0.061 & 0.000 & 0.060 \\ 
  b[(Intercept) word:tagant] & 1.000 & 8954 & -0.008 & 0.034 & -0.075 & -0.008 & 0.058 \\ 
  b[(Intercept) word:taha] & 1.000 & 8478 & -0.022 & 0.034 & -0.088 & -0.022 & 0.044 \\ 
  b[(Intercept) word:talu] & 1.000 & 7367 & 0.044 & 0.026 & -0.007 & 0.044 & 0.095 \\ 
  b[(Intercept) word:tamme] & 1.000 & 9268 & 0.011 & 0.034 & -0.055 & 0.011 & 0.078 \\ 
  b[(Intercept) word:tampimaie.] & 0.999 & 8113 & 0.015 & 0.032 & -0.048 & 0.015 & 0.076 \\ 
  b[(Intercept) word:tasane,] & 0.999 & 8388 & 0.003 & 0.034 & -0.061 & 0.003 & 0.070 \\ 
  b[(Intercept) word:teine] & 1.000 & 6153 & 0.068 & 0.033 & 0.004 & 0.068 & 0.132 \\ 
  b[(Intercept) word:tõsta] & 0.999 & 7912 & -0.011 & 0.040 & -0.090 & -0.011 & 0.067 \\ 
  b[(Intercept) word:tõusis] & 0.999 & 6747 & -0.044 & 0.031 & -0.104 & -0.045 & 0.016 \\ 
  b[(Intercept) word:tsirgu] & 1.000 & 8148 & 0.034 & 0.033 & -0.030 & 0.034 & 0.098 \\ 
  b[(Intercept) word:tuastagi,] & 1.000 & 6688 & 0.030 & 0.033 & -0.034 & 0.030 & 0.095 \\ 
  b[(Intercept) word:tubadesse,] & 0.999 & 7485 & 0.009 & 0.034 & -0.056 & 0.008 & 0.075 \\ 
  b[(Intercept) word:tul´lid] & 1.000 & 7723 & -0.039 & 0.041 & -0.120 & -0.039 & 0.039 \\ 
  b[(Intercept) word:tule] & 0.999 & 8225 & 0.032 & 0.034 & -0.036 & 0.032 & 0.099 \\ 
  b[(Intercept) word:tuli] & 0.999 & 7453 & 0.009 & 0.034 & -0.060 & 0.009 & 0.077 \\ 
  b[(Intercept) word:tulla.] & 1.000 & 4612 & 0.082 & 0.041 & 0.001 & 0.081 & 0.163 \\ 
  b[(Intercept) word:tullu] & 1.000 & 8668 & -0.020 & 0.039 & -0.098 & -0.019 & 0.057 \\ 
  b[(Intercept) word:tulnud,] & 1.000 & 8399 & -0.020 & 0.040 & -0.098 & -0.019 & 0.056 \\ 
  b[(Intercept) word:türnapuida,] & 1.000 & 7896 & 0.048 & 0.035 & -0.021 & 0.048 & 0.116 \\ 
  b[(Intercept) word:tütar] & 1.000 & 8146 & -0.033 & 0.034 & -0.101 & -0.033 & 0.034 \\ 
  b[(Intercept) word:tüve] & 1.000 & 7273 & 0.051 & 0.034 & -0.015 & 0.051 & 0.120 \\ 
  b[(Intercept) word:ümmert] & 1.000 & 8661 & -0.007 & 0.028 & -0.062 & -0.007 & 0.047 \\ 
  b[(Intercept) word:uppus] & 1.000 & 7150 & -0.029 & 0.032 & -0.094 & -0.028 & 0.036 \\ 
  b[(Intercept) word:usselinki,] & 1.000 & 8084 & -0.012 & 0.040 & -0.089 & -0.012 & 0.066 \\ 
  b[(Intercept) word:vaesed] & 0.999 & 7031 & -0.020 & 0.030 & -0.079 & -0.020 & 0.038 \\ 
  b[(Intercept) word:vahele,] & 0.999 & 6831 & 0.042 & 0.027 & -0.010 & 0.042 & 0.095 \\ 
  b[(Intercept) word:vaisiküisi,] & 1.000 & 7441 & 0.007 & 0.035 & -0.060 & 0.006 & 0.076 \\ 
  b[(Intercept) word:vaisiküisist] & 0.999 & 7569 & 0.029 & 0.035 & -0.038 & 0.029 & 0.096 \\ 
  b[(Intercept) word:vait] & 1.000 & 7705 & 0.025 & 0.040 & -0.054 & 0.025 & 0.107 \\ 
  b[(Intercept) word:valada,] & 0.999 & 8224 & 0.030 & 0.034 & -0.035 & 0.031 & 0.099 \\ 
  b[(Intercept) word:valge] & 1.000 & 8275 & -0.001 & 0.034 & -0.068 & -0.001 & 0.066 \\ 
  b[(Intercept) word:valutave!] & 0.999 & 8650 & -0.010 & 0.033 & -0.073 & -0.011 & 0.054 \\ 
  b[(Intercept) word:varba] & 0.999 & 7969 & -0.007 & 0.034 & -0.073 & -0.007 & 0.059 \\ 
  b[(Intercept) word:vemmeldada!] & 1.000 & 7262 & -0.030 & 0.034 & -0.095 & -0.030 & 0.038 \\ 
  b[(Intercept) word:vennal] & 1.000 & 7477 & -0.036 & 0.040 & -0.113 & -0.037 & 0.042 \\ 
  b[(Intercept) word:vennanaistel] & 0.999 & 7466 & -0.020 & 0.035 & -0.088 & -0.021 & 0.047 \\ 
  b[(Intercept) word:vihma] & 1.000 & 7646 & 0.044 & 0.035 & -0.024 & 0.043 & 0.110 \\ 
  b[(Intercept) word:vii] & 1.000 & 6638 & -0.005 & 0.035 & -0.075 & -0.005 & 0.065 \\ 
  b[(Intercept) word:virvekirja,] & 1.000 & 8521 & 0.007 & 0.034 & -0.059 & 0.007 & 0.073 \\ 
  b[(Intercept) word:võta] & 1.000 & 6540 & -0.035 & 0.021 & -0.077 & -0.035 & 0.007 \\ 
  b[(Intercept) word:võtke] & 1.000 & 7322 & -0.071 & 0.034 & -0.138 & -0.071 & -0.005 \\ 
  b[(Intercept) word:Võtke] & 1.000 & 6387 & -0.072 & 0.035 & -0.142 & -0.073 & -0.006 \\ 
  b[(Intercept) performer:LK] & 1.000 & 2045 & 0.030 & 0.035 & -0.043 & 0.029 & 0.103 \\ 
  b[(Intercept) performer:LO] & 1.000 & 2076 & -0.046 & 0.035 & -0.120 & -0.046 & 0.025 \\ 
  b[(Intercept) performer:MH] & 1.000 & 1870 & 0.018 & 0.035 & -0.054 & 0.017 & 0.093 \\ 
  sigma & 1.002 & 1795 & 0.062 & 0.003 & 0.057 & 0.062 & 0.067 \\ 
  Sigma[word:(Intercept),(Intercept)] & 1.006 & 1191 & 0.003 & 0.000 & 0.002 & 0.003 & 0.004 \\ 
  Sigma[performer:(Intercept),(Intercept)] & 1.001 & 2557 & 0.004 & 0.006 & 0.000 & 0.002 & 0.020 \\ 
  mean\_PPD & 1.000 & 3927 & 0.201 & 0.004 & 0.194 & 0.200 & 0.208 \\ 
  log-posterior & 1.009 &  673 & 388.413 & 19.776 & 349.271 & 389.028 & 425.331 \\ 
   \bottomrule
\caption{Parameters plot with all word intercepts (shinystan)} 
\end{longtable}
%

%
%\include{chapter-appendix3}


%%%%%%%%%%%%%%%%%%%%%%%%%%%%%%%%%%%%%%%%%%%%%%%%%%%%%%%%%%%%%%%%%%%%%%
% Generate the bibliography.					     %
%%%%%%%%%%%%%%%%%%%%%%%%%%%%%%%%%%%%%%%%%%%%%%%%%%%%%%%%%%%%%%%%%%%%%%
%								     %
% NOTE: For master's theses and reports, NOTHING is permitted to     %
%	come between the bibliography and the vita. The command      %
%	to generate the index (if used) MUST be moved to before      %
%	this section.	
\printindex%    % Include the index here. Comment out this line      %
%		% with a percent sign if you do not want an index.   %					     %
%								     %
%\nocite{*}      % This command causes all items in the 		     %
                % bibliographic database to be added to 	     %
                % the bibliography, even if they are not 	     %
                % explicitly cited in the text. 		     %
		%						     %
\bibliographystyle{apa-good.bst}  % Here the bibliography 		     %
\bibliography{regilaul}        % is inserted.			     %
\index{Bibliography@\emph{Bibliography}}%			     %
%%%%%%%%%%%%%%%%%%%%%%%%%%%%%%%%%%%%%%%%%%%%%%%%%%%%%%%%%%%%%%%%%%%%%%


%%%%%%%%%%%%%%%%%%%%%%%%%%%%%%%%%%%%%%%%%%%%%%%%%%%%%%%%%%%%%%%%%%%%%%
% Generate the index.						     %
%%%%%%%%%%%%%%%%%%%%%%%%%%%%%%%%%%%%%%%%%%%%%%%%%%%%%%%%%%%%%%%%%%%%%%
%								     %
% NOTE: For master's theses and reports, NOTHING is permitted to     %
%	come between the bibliography and the vita. This section     %
%	to generate the index (if used) MUST be moved to before      %
%	the bibliography section.				     %
%								     %

%%%%%%%%%%%%%%%%%%%%%%%%%%%%%%%%%%%%%%%%%%%%%%%%%%%%%%%%%%%%%%%%%%%%%%


%%%%%%%%%%%%%%%%%%%%%%%%%%%%%%%%%%%%%%%%%%%%%%%%%%%%%%%%%%%%%%%%%%%%%%
% Vita page.							     %
%%%%%%%%%%%%%%%%%%%%%%%%%%%%%%%%%%%%%%%%%%%%%%%%%%%%%%%%%%%%%%%%%%%%%%

\begin{vita}
%Sarah Marie Ransom-Laud
%was born in Sarasota, Florida on 11 July 1990. Her early career was as a songwriter and recording artist, releasing several full-length albums combining digital and analog recording methods over the last decade. 
%
%In 2018 she received the Bachelor of Arts degree in Linguistics from the University of Wisconsin, Milwaukee. Upon graduation, she moved to Austin with her partner, Kavi, and taught English as a Second Language (ESL) to adults until her admission into the PhD program in the department of Linguistics at the University of Texas at Austin, where she matriculated in Fall 2019. 


\end{vita}

\end{document}
