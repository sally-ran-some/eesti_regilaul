%%%%%%%%%%%%%%%%%%%%%%%%%%%%%%%%%%%%%%%%%%%%%%%%%%%%%%%%%%%%%%%%%%%%%%
%%  disstemplate.tex, to be compiled with latex.		     %
%%  08 April 2002	Version 4				     %
%%%%%%%%%%%%%%%%%%%%%%%%%%%%%%%%%%%%%%%%%%%%%%%%%%%%%%%%%%%%%%%%%%%%%%
%%								     %
%%  Writing a Doctoral Dissertation with LaTeX at		     %
%%	the University of Texas at Austin			     %
%%								     %
%%  (Modify this ``template'' for your own dissertation.)	     %
%%								     %
%%%%%%%%%%%%%%%%%%%%%%%%%%%%%%%%%%%%%%%%%%%%%%%%%%%%%%%%%%%%%%%%%%%%%%


\documentclass[12pt]{report}	% The documentclass must be ``report''.

\usepackage{utdiss2}  		% Dissertation package style file.

\usepackage{natbib}
%%%%%%%%%%%%%%%%%%%%%%%%%%%%%%%%%%%%%%%%%%%%%%%%%%%%%%%%%%%%%%%%%%%%%%
% Optional packages used for this sample dissertation. If you don't  %
% need a capability in your dissertation, feel free to comment out   %
% the package usage command.					     %
%%%%%%%%%%%%%%%%%%%%%%%%%%%%%%%%%%%%%%%%%%%%%%%%%%%%%%%%%%%%%%%%%%%%%%

\usepackage{amsmath,amsthm,amsfonts,amscd} 
				% Some packages to write mathematics.
\usepackage{eucal} 	 	% Euler fonts
\usepackage{verbatim}      	% Allows quoting source with commands.
\usepackage{makeidx}       	% Package to make an index.
\usepackage{epsfig}         	% Allows inclusion of eps files.
\usepackage{epsfig}         	% Allows inclusion of eps files.
%\usepackage{cyrchar}
%\usepackage{citesort}         	% 
\usepackage{url}		% Allows good typesetting of web URLs.
%\usepackage{draftcopy}	
\usepackage[utf8]{inputenc}
\usepackage{CharisSIL}
\usepackage{booktabs}
\usepackage{topcapt}
\usepackage{subcaption}
%\usepackage{wrapfig}



\usepackage{leipzig}
\usepackage{gb4e}



	% Uncomment this line to have the
				% word, "DRAFT," as a background
				% "watermark" on all of the pages of
				% of your draft versions. When ready
				% to generate your final copy, re-comment
				% it out with a percent sign to remove
				% the word draft before you re-run
				% Makediss for the last time.

\author{Sally Ransom}  	% Required

\address{305 E. 23rd Street STOP B5100\\ Austin, Texas 78712}  % Required

\title{from lexical to lyrical: how linguistic rhythm negotiates with musical metre to join the song}
                                                    % Required

%%%%%%%%%%%%%%%%%%%%%%%%%%%%%%%%%%%%%%%%%%%%%%%%%%%%%%%%%%%%%%%%%%%%%%
% NOTICE: The total number of supervisors and other members %%%%%%%%%%
%%%%%%%%%%%%%%% MUST be seven (7) or less! If you put in more, %%%%%%%
%%%%%%%%%%%%%%% they are put on the page after the Committee %%%%%%%%%
%%%%%%%%%%%%%%% Certification of Approved Version page. %%%%%%%%%%%%%%
%%%%%%%%%%%%%%%%%%%%%%%%%%%%%%%%%%%%%%%%%%%%%%%%%%%%%%%%%%%%%%%%%%%%%%

%%%%%%%%%%%%%%%%%%%%%%%%%%%%%%%%%%%%%%%%%%%%%%%%%%%%%%%%%%%%%%%%%%%%%%
%
% Enter names of the supervisor and co-supervisor(s), if any,
% of your dissertation committee. Put one name per line with
% the name in square brackets. The name on the last line, however,
% must be in curly braces.
%
% If you have only one supervisor, the entry below will read:
%
%	\supervisor
%		{Supervisor's Name}
%
% NOTE: Maximum three supervisors. Minimum one supervisor.
% NOTE: The Office of Graduate Studies will accept only two supervisors!
% 
%
\supervisor
	[Scott Myers]
	{Katrin Erk}

%%%%%%%%%%%%%%%%%%%%%%%%%%%%%%%%%%%%%%%%%%%%%%%%%%%%%%%%%%%%%%%%%%%%%%
%
% Enter names of the other (non-supervisor) members(s) of your
% dissertation committee. Put one name per line with the name
% in square brackets. The name on the last line, however, must
% be in curly braces.
%
% NOTE: Maximum six other members. Minimum zero other members.
% NOTE: The Office of Graduate Studies may restrict you to a total
%	of six committee members.
%
%


%%%%%%%%%%%%%%%%%%%%%%%%%%%%%%%%%%%%%%%%%%%%%%%%%%%%%%%%%%%%%%%%%%%%%%

\previousdegrees{B.A., Linguistics}
     % The abbreviated form of your previous degree(s).
     % E.g., \previousdegrees{B.S., MBA}.
     %
     % The default value is `B.S., M.S.'

%\graduationmonth{...}      
     % Graduation month, either May, August, or December, in the form
     % as `\graduationmonth{May}'. Do not abbreviate.
     %
     % The default value (either May, August, or December) is guessed
     % according to the time of running LaTeX.

%\graduationyear{...}   
     % Graduation year, in the form as `\graduationyear{2001}'.
     % Use a 4 digit (not a 2 digit) number.
     %
     % The default value is guessed according to the time of 
     % running LaTeX.

%\typist{...}       
     % The name(s) of typist(s), put `the author' if you do it yourself.
     % E.g., `\typist{Maryann Hersey and the author}'.
     %
     % The default value is `the author'.


%%%%%%%%%%%%%%%%%%%%%%%%%%%%%%%%%%%%%%%%%%%%%%%%%%%%%%%%%%%%%%%%%%%%%%
% Commands for master's theses and reports.			     %
%%%%%%%%%%%%%%%%%%%%%%%%%%%%%%%%%%%%%%%%%%%%%%%%%%%%%%%%%%%%%%%%%%%%%%
%
% If the degree you're seeking is NOT Doctor of Philosophy, uncomment
% (remove the % in front of) the following two command lines (the ones
% that have the \ as their second character).
%
\degree{MASTER OF ARTS}
\degreeabbr{M.A.}

% Uncomment the line below that corresponds to the type of master's
% document you are writing.
%
%\masterreport
\masterthesis


%%%%%%%%%%%%%%%%%%%%%%%%%%%%%%%%%%%%%%%%%%%%%%%%%%%%%%%%%%%%%%%%%%%%%%
% Some optional commands to change the document's defaults.	     %
%%%%%%%%%%%%%%%%%%%%%%%%%%%%%%%%%%%%%%%%%%%%%%%%%%%%%%%%%%%%%%%%%%%%%%
%
%\singlespacing
%\oneandonehalfspacing

%\singlespacequote
\oneandonehalfspacequote

\topmargin 0.125in	% Adjust this value if the PostScript file output
			% of your dissertation has incorrect top and 
			% bottom margins. Print a copy of at least one
			% full page of your dissertation (not the first
			% page of a chapter) and measure the top and
			% bottom margins with a ruler. You must have
			% a top margin of 1.5" and a bottom margin of
			% at least 1.25". The page numbers must be at
			% least 1.00" from the bottom of the page.
			% If the margins are not correct, adjust this
			% value accordingly and re-compile and print again.
			%
			% The default value is 0.125"

		% If you want to adjust other margins, they are in the
		% utdiss2-nn.sty file near the top. If you are using
		% the shell script Makediss on a Unix/Linux system, make
		% your changes in the utdiss2-nn.sty file instead of
		% utdiss2.sty because Makediss will overwrite any changes
		% made to utdiss2.sty.

%%%%%%%%%%%%%%%%%%%%%%%%%%%%%%%%%%%%%%%%%%%%%%%%%%%%%%%%%%%%%%%%%%%%%%
% Some optional commands to be tested.				     %
%%%%%%%%%%%%%%%%%%%%%%%%%%%%%%%%%%%%%%%%%%%%%%%%%%%%%%%%%%%%%%%%%%%%%%

% If there are 10 or more sections, 10 or more subsections for a section,
% etc., you need to make an adjustment to the Table of Contents with the
% command \longtocentry.
%
%\longtocentry 



%%%%%%%%%%%%%%%%%%%%%%%%%%%%%%%%%%%%%%%%%%%%%%%%%%%%%%%%%%%%%%%%%%%%%%
%	Some math support.					     %
%%%%%%%%%%%%%%%%%%%%%%%%%%%%%%%%%%%%%%%%%%%%%%%%%%%%%%%%%%%%%%%%%%%%%%
%
%	Theorem environments (these need the amsthm package)
%
%% \theoremstyle{plain} %% This is the default
%
%\newtheorem{thm}{Theorem}[section]
%\newtheorem{cor}[thm]{Corollary}
%\newtheorem{lem}[thm]{Lemma}
%\newtheorem{prop}[thm]{Proposition}
%\newtheorem{ax}{Axiom}
%
%\theoremstyle{definition}
%\newtheorem{defn}{Definition}[section]
%
%\theoremstyle{remark}
%\newtheorem{rem}{Remark}[section]
%\newtheorem*{notation}{Notation}

%\numberwithin{equation}{section}


%%%%%%%%%%%%%%%%%%%%%%%%%%%%%%%%%%%%%%%%%%%%%%%%%%%%%%%%%%%%%%%%%%%%%%
%	Macros.							     %
%%%%%%%%%%%%%%%%%%%%%%%%%%%%%%%%%%%%%%%%%%%%%%%%%%%%%%%%%%%%%%%%%%%%%%
%
%	Here some macros that are needed in this document:


\newcommand{\latexe}{{\LaTeX\kern.125em2%
                      \lower.5ex\hbox{$\varepsilon$}}}

\newcommand{\amslatex}{\AmS-\LaTeX{}}

\chardef\bslash=`\\	% \bslash makes a backslash (in tt fonts)
			%	p. 424, TeXbook

\newcommand{\cn}[1]{\texttt{\bslash #1}}

\makeatletter		% Starts section where @ is considered a letter
			% and thus may be used in commands.
\def\square{\RIfM@\bgroup\else$\bgroup\aftergroup$\fi
  \vcenter{\hrule\hbox{\vrule\@height.6em\kern.6em\vrule}%
                                              \hrule}\egroup}
\makeatother		% Ends sections where @ is considered a letter.
			% Now @ cannot be used in commands.

\makeindex    % Make the index

%%%%%%%%%%%%%%%%%%%%%%%%%%%%%%%%%%%%%%%%%%%%%%%%%%%%%%%%%%%%%%%%%%%%%%
%		The document starts here.			     %
%%%%%%%%%%%%%%%%%%%%%%%%%%%%%%%%%%%%%%%%%%%%%%%%%%%%%%%%%%%%%%%%%%%%%%

\begin{document}

\copyrightpage          % Produces the copyright page.


%
% NOTE: In a doctoral dissertation, the Committee Certification page
%		(with signatures) is BEFORE the Title page.
%	In a masters thesis or report, the Signature page
%		(with signatures) is AFTER the Title page.
%
%	If you are writing a masters thesis or report, you MUST REVERSE
%	the order of the \commcertpage and \titlepage commands below.

\titlepage              % Produces the title page.
%
\commcertpage           % Produces the Committee Certification
			%   of Approved Version page (doctoral)
			%   or Signature page (masters).
			%		





%%%%%%%%%%%%%%%%%%%%%%%%%%%%%%%%%%%%%%%%%%%%%%%%%%%%%%%%%%%%%%%%%%%%%%
% Dedication and/or epigraph are optional, but must occur here.      %
%%%%%%%%%%%%%%%%%%%%%%%%%%%%%%%%%%%%%%%%%%%%%%%%%%%%%%%%%%%%%%%%%%%%%%
%
\begin{dedication}
\index{Epigraph@\emph{Epigraph}}%

\begin{quotation}
{\scriptsize I think that when linguists discuss and dispute sound length and stress among themselves they would definitely benefit from inviting ethno-musicologists to join them. They could discuss the issues together, and not only based on written records but also sung records. \\ 
That what is written is fiction. Only that what is sung is truth. \\
\cite{tormisProblemsThatRegilaul2007} }



\end{quotation}

\end{dedication}


%\begin{acknowledgments}		% Optional
%\index{Acknowledgments@\emph{Acknowledgments}}%
%I wish to thank the multitudes of people who helped me. Time would
%fail me to tell of \ldots
%\end{acknowledgments}


% The abstract is required. Note the use of ``utabstract'' instead of
% ``abstract''! This was necessary to fix a page numbering problem.
% The abstract heading is generated automatically.
% Do NOT use \begin{abstract} ... \end{abstract}.
%
\utabstract
\index{Abstract}%
\indent
%

This paper takes the performance of lyrical folksongs as an opportunity for an exploratory corpus phonetics study of syllable prominence using the Estonian language, which has three syllable quantities, a predictable stress pattern at the word level and a robust tradition of folksongs {\it regilaul} which follow a strict metrical pattern. Using a novel semi-automatic computational annotation method, beat detection algorithms used in live music performance are used to determine the location and duration of the strong beats in a sung performance of a song.  Combined with acoustic analysis and natural language processing tools, this paper compares rhythmic prominence in songs with the  prominence pattern of the language. Using this method, I analyze the acoustic effect that becoming lyrics has on well established acoustic correlates of rhythmic prominence in Estonian compare the attested primary stress system and ternary quantity contrast documented in Estonian with the prescribed rhythmic pattern of an ancient folksong known as {\it regilaul}.  Results show that vowel duration is utilized as a prominence cue in both the song and word level. We also examine the long and overlong syllable quantity contrast in the context of the song, and find that the contrast between these two is not preserved in the temporal structure of the song.

The paper first introduces prosodic prominence in general, then delves into the metrical specifics of Estonian and its folksong tradition. After outlining the research questions, the methods section describes the corpus of {\it regilaul} songs: its construction, annotation, and processing. Investigating whether the established acoustic correlate of stress and quantity in spoken Estonian (duration) are preserved after undergoing subordination to the strict temporal properties of the structure of a song.

 \tableofcontents   % Table of Contents will be automatically
                   % generated and placed here.

\listoftables      % List of Tables and List of Figures will be placed
\listoffigures     % here, if applicable.



%%%%%%%%%%%%%%%%%%%%%%%%%%%%%%%%%%%%%%%%%%%%%%%%%%%%%%%%%%%%%%%%%%%%%%
% Actual text starts here.					     %
%%%%%%%%%%%%%%%%%%%%%%%%%%%%%%%%%%%%%%%%%%%%%%%%%%%%%%%%%%%%%%%%%%%%%%
%
% Including external files for each chapter makes this document simpler,
% makes each chapter simpler, and allows for generating test documents
% with as few as zero chapters (by commenting out the include statements).
% This allows quicker processing by the Makediss command file in case you
% are not working on a specific, long and slow to compile chapter. You
% can even change the chapter order by merely interchanging the order
% of the include statements (something I found helpful in my own
% dissertation).
%

%
%\chapter{Introduction}
\index{Introduction@\emph{Introduction}}%is

\section{On words becoming lyrics}
\subsection{song lyrics as the music-language interface}

In order to join music in song and become lyrics, language elements undergo modification to be fit into the musical structure. At the same time, the interpretability of those linguistic elements which constitute contrasts in semantic meanings has motivation for preservation. If the performer wishes the audience to perceive meaningful lyrical content, these contrasts must be crystallized at some level of salience in the acoustic signal. 

This project takes folksong performance as an opportunity to examine the interface of language and music. In particular, I investigate the acoustic manifestations of the word-prosodic features attested in spoken Estonian when they are sung in a temporally restricted domain such as music. 


both regilaul and Eesti have heirarchichal metrical systems by which long and short, stressed and unstressed constituents are arranged. In music, the smallest prosodic constituent is an individual note event whose relationship to the other notes in the song are indicated by the time signature, i.e., 3/4 or 4/4. The denominator corresponds to the number of divisions of a ``whole'' note, while the numerator refers to the number of ``beats'' in a single measure. So the prosodic heirarchy in music begins at the note level, then to each measure as constituent, larger phrases and motifs, and eventually the highest level which is the entire song. In Eesti, attested segments are organized into syllables, which in turn combine in strong-weak relationships to form prosodic feet, which make up the words of phrases and so on. 



 In a regilaul verse line, each syllable corresponds to an eighth note in the measure, with every odd positioned syllable-note coinciding with the ``beat'' onset in a typical 4/4 measure. Falling on the beat in music generally coincides with increased prominence compared to notes occurring ``off'' the beat. This is acoustically manifested in increased duration of notes on the beat compared to nominally isochronous notes in other positions. That is, in a measure containing eight eighth notes, the absolute durations of the four eighth notes corresponding to the beat will be longer than the absolute durations of the four eighth notes in even positions, while maintaining the perception of the nominal isochrony. 



%Acoustically, the difference between a stressed and an unstressed syllable is found not in a single cue but in a convergence of several cues: often duration, increased variability in f0, and increased vowel space, corresponding to a notion of localized hyperarticulation \citep{smiljanicProductionPerceptionClear2005, de1995supraglottal, lindblom1990}. 

How do the word-prosodic requirements negotiate with the imposed prosodic heirarchy of music? The rhythmic organization of song is said to integrate the prosodic structure of the language with musical rhythmic principles \citep{palmerLinguisticProsodyMusical1992}. However, earlier studies of {\it regilaul} explored temporal aspects of the songs and found that duration characteristics that would usually indicate important semantic differences lost their distinctions partially or entirely. Assuming that the intention of the singer is for the lyrics to be understood, I hypothesize that if some durational correlates of contrasting word-prosodic constituents are made less distinct in the process of compromising with the song that some other acoustic correlate of the relevant contrast at the word-prosodic level will be present, if not enhanced.  


%
%Whether there is a correlation between poetic metre and the prosodic structure of a language.\citep{lehistePhoneticsMetrics1992} investigated the actual phonetic realization of different metres in different languages, finding evidence 






%to paraphrase: Semantically relevant oppositions of word initial short-long and long-short disyllabic units in speech are not kept completely intact in folk songs. Short-long disyllables are treated in a different manner by the performer, depending on whether their initial syllable occurs at a rise or at a fall in a foot
\citep{rossTimingEstonianFolk1998} Analyzed and concluded that the {\it regilaul} lyrics are the result of an interaction between word and song prosodic heirarchies. This conclusion relied critically on measuring the durations of syllable-notes, where they found being on or off the beat was the better predictor for duration. \\


trade-off Q for S? 
\citep{rossTradeoffQuantityStress1996}. \\
They conclude that duration contrasts ordinarily present at the word level are {\it ``lost"} \citep{rossLostProsodicOppositions1994} to the temporal restrictions of the song. 

same song, timing and quality measured
\citep{rossFormants90}
\citep{rossStudyTimingEstonian1989a} \\

I argue that the cooperation of language in music is more than an interaction: that is, I argue that lyrical folksongs are examples of bidirectional {\it entrainment} of two independed oscillatory processes. I do not question the findings of Ross \& Lehiste, but wish to extend them. If the the syllable is required to fit in the prosodic level of the song, the linguistic contrasts of the words could be present either at a lower level of the prosodic heirarchy, i.e.,  the segment, or the contrast could be preserved in another acoustic dimension: i.e.,  vowel space dispersion. 


Null hypothesis: on or off-beat position is best predictor for vowel duration of syllables in all categories: stressed, unstressed, short, long, and overlong. This would mean that the syllable nucleus is proportional to the total syllable duration, and that the duration contrast is indeed ``lost." 

H_{1}: duration contrasts for syllable quantity will be evident in the vowel duration, with vowel duration {\it decreasing} as syllable weight increases. Given the findings of \citep{rosslehiste}, isochronous syllable-notes would result in heavier syllables having shorter nuclei to accomodate for the coda and complex codas that distinguish the syllable weights from each other. 

H_{2}: stress/unstress contrasts will be evident in the nucleus in terms of hypo and hyper articulation. For this I measure both nucleus duration and vowel space dispersion. 

Perceptual asymmetry: shortening not inducing MNN, but lengthening does \citep{eestiMNNasymmetry}. I expect that while syllable-notes that are on the beat will be prominent compared to those off the beat, unstressed syllables will be reduced compared to stressed syllables in the same position. 

%\chapter{Introduction}
\index{Introduction@\emph{Introduction}}

Lyrical songs are the result of bidirectional entrainment of language and music.

%Perceptual asymmetry: shortening not inducing MNN, but lengthening does \cite{eestiMNNasymmetry}. I expect that while syllable-notes that are on the beat will be prominent compared to those off the beat, unstressed syllables will be reduced compared to stressed syllables in the same position. 
%
%Will the lyrics prefer lengthening or shortening of vowel nuclei to preserve the contrast?



Estonian is useful because it has both fixed word-level stress and three quantity degrees. \cite{thatone} Analyzed and concluded that the {\it regilaul} lyrics are the result of an interaction between word and song prosodic heirarchies. This conclusion relied critically on measuring the durations of syllable-notes, where they found that song position (on or off the beat) was the better predictor for duration. They conclude that duration contrasts ordinarily present at the word level are {\it ``lost"} \cite{rosslostprosodic}   to the temporal restrictions of the song. 

I argue that the cooperation of language in music is more than an interaction: that is, I argue that lyrical folksongs are examples of bidirectional {\it entrainment} of two independed oscillatory processes. I do not question the findings of Ross \& Lehiste, but wish to extend them. If the the syllable is required to fit in the prosodic level of the song, the linguistic contrasts of the words could be present either at a lower level of the prosodic heirarchy, i.e.,  the segment, or the contrast could be preserved in another acoustic dimension: i.e.,  vowel space dispersion. 


In order to test the hypothesis of bidirectional entrainment, I look at the syllable nucleus. 

Null hypothesis: on or off-beat position is best predictor for vowel duration of syllables in all categories: stressed, unstressed, short, long, and overlong. This would mean that the syllable nucleus is proportional to the total syllable duration, and that the duration contrast is indeed ``lost." 

H_{1}: duration contrasts for syllable quantity will be evident in the vowel duration, with vowel duration {\it decreasing} as syllable weight increases. Given the findings of \cite{rosslehiste}, isochronous syllable-notes would result in heavier syllables having shorter nuclei to accomodate for the coda and complex codas that distinguish the syllable weights from each other. 

H_{2}: stress/unstress contrasts will be evident in the nucleus in terms of hypo and hyper articulation. For this I measure both nucleus duration and vowel space dispersion. 


Since \cite{r and l} found that song is the best predictor for syllable-note duration, I expect that the song is the higher level of the prosodic heirarchy. 

Entrainment: 
In order for entrainment to happen, two oscillators must be able to act independently of each other: resonance, while it is oscillatory, is not an example of entrainment. If a tuning fork is removed from a box, the box no longer oscillates. Both language and music meet the criteria for independent oscillation: independently of a song, the linguistic constituents have periodicity, relative prominence, heirarchy. Likewise, music without any lyrics at all has periodicity and heirarchy. 

unidirectional: If R \& L's findings extend to the segmental level, then the song is oscillating within the shape of the words of the language, and rhythmic contrasts from the linguistic/word level rely heavily on ``top down" processing, i.e., the semantic context of the verses in the song. 

bidirectional: If R \& L's findings do not extend below the syllabic level: that is, if prominence contrasts are preserved at a level below the syllable-note, then this would be bidirectional entrainment, as the two oscillators are operating independently. (complex wave metaphor). 


%\chapter{The Rhythmic Domain}
\index{The Rhythmic Domain@\emph{The Rhythmic Domain}}
I begin with a short discussion on the four dimensions necessary to examine the notion of rhythm and define the concept for the sake of this paper. This is especially necessary because our psychological ``right now" is only a few seconds, but with a song we are able to store much longer spans of time as complete entities with structured constituents. 

Oscillations!

what we perceive as tone is very fast rhythm
what we term rhythm is very slow tone

Both the physical events, i.e., the periodic frequency of a signal x in a specified time, and the psychological events, i.e., the pitch as perceived by auditors. Since we know that pitch is not always the same as frequency, it follows that this applies to lower frequencies as well: i.e., beats. 

The psychological event data give us access to the underlying cognitive structure of the rhythm user. The acoustic signal details the manifestations of a singer or talker's temporal organization, which cues they are using to differentiate constituents of these structures. 

A rhythm's accent structure functions in songs is to indicate the the position within a given constituent. It can be manifested acoustically in a variety of ways: lengthening of duration, faster attack, increased force. 


It seems that many theories of metrical prominence were imagined as the wave forms of pure tones, when the reality is much more complex. While there appears to be an available binary categorization level for perceived oscillatory magnitudes, i.e., Strong/weak note pairs, at least four dimensions are necessary to capture the relevant structural relationships of spatial-temporal elements of a song as a constituent. 

\section{Acoustic Dimensions}



First, physical dimensions of measurements


The position of air particles in time

\subsection{Dimension One: Particles as Points}
An air particle is a single point
a speck of dust
Spectral Slice for x moment in time
x at time m
\subsection{Dimension Two: Point Locations}
Frequency = how fast particles move
move at different rates, but at x time, how were the points dispersed? What is their density, what is their shape (resonant frequencies) 
x,y at time m 
\subsection{Dimension Three: Concentration of Points}
euc((x,y), (x2,y2)) 
\subsection{Dimension Four: Points in Motion} 

duration of constituent category 


\section{Perceptual Dimensions}

\subsection{Linguistic Audition}
\subsection{Musical Audition} 

falsely hear ``beat" wherever told, even especially isochronous equitone sequences
MNN studies

asymmetry: short-long easier than long-short
shortening not induce MNN, only lengthening 
underspecified for the length of short(er) constituents so long as the relative different-ness is preserved. 
%
\chapter{Background}
\index{Background@\emph{Background}}%is

\section{Meter and Rhythm}

Given the interdisciplinary nature of the research questions, an earnest attempt was made to find a suitable intersection in the literature from which to approach this. Computational studies of music, acoustic-phonetic studies of Estonian both proved useful, but it was in the specific domain of metrical prosody that problems arose. 

For one thing, the oft-cited \citep{essensPovel1985}\footnote{cited by 927 and 329, respectively} offers a definition of metrical  temporal patterns according to what they call "integer ratios" such as 2:1, 3:1, etc, and nonmetrical patterns corresponding to "non-integer ratios," which they offer in the forms 1.5:1, 2.5:1, 3.5:1, and so on. This definition defies the principles of elementary number theory in several ways: 1.5:1, 2.5:1, and 3.5:1 are equivalent to 3:2, 5:2, and 7:2, respectively. In their simplest form, it is clear that they are in fact ratios composed of prime integers. Indeed the term "non-integer ratio" is a pecuiliar one, as integers are a subset of all rational numbers, who are defined by their representability as ratios. Outside the set of rational numbers are the irrationals: but if we are cognitively restricted to timekeeping with simple ratios only, we should have considerable difficulty reading time from a clock: the ratio of a circle's radius to its diameter is the transcendental \pi.  

``unpredictable patterns" oxymoronic. pattern repeats by definition. If the conclusion drawn from a valid and sound argument is false, then it is the premises that are at fault. In this case, the pattern isn't unpredictible: they are using the wrong key. 


The results of \citep{essensPovel1985} indicating subjects' difficulty in repeating these patterns compared to others can be explained by the fact that the stimulus design did not offer complete repetitions of this pattern, due to the authors choice to represent the prime ratios in this way. It also violates a basic understanding of human chronobiology more generally: extending their conclusions with respect to the 7:2 ratio would predict that the expression ``same time next week" should be incomprehensible. 
Likewise, studies following this definition demonstrated a difference in neural representation \citep{katsuyuki1999}, who follow up \citep{essensPovel1985} and add an additional level called ``weakly metrical," with their example given in \ref{weaksauce} below. 

\begin{figure}[htbp]
\begin{center}
\includegraphics{/Users/sarah/Git/regilaul_project/manuscript/main/schultsSMWMNM2013.png}
\caption{default}
\label{default}
\end{center}
\end{figure}

I would first like to point out that the pattern for ``nonmetrical" is exactly the same sequence as the one given for ``strongly metrical," with the only difference being a bit of rush and drag with respect to the beat. The rush and drag themselves only exist in view of their relationship to the beat grid they have imposed the events on: it can't be ahead of some time without time being defined. 

In spite of the misguidedness due to paradigmatic dogma, the results of \citep{katsuyuki1999} show no significant differences in the RT of pattern identification nor in the learnability between the levels of proposed metrical strength. 




This brings forth another issue which I think is the result of a microscopic view of rhythm. There is a notion in the literature of timekeeping relying on "inducing" some sort of internal clock. This elides the fact that we are rhythmic biological entities. Our minds do not "make" a clock for external stimulus: we {\it are} the clock. 


Neuroscience long ago accepted that it is oscillators who are responsible for coordinating fundamental processes such as breathing in and distributing oxygen throughout the body. Complex pattern generators guide the movements of wings in cricket songs \citep{buzsaki2006}. 
%The simplest biological systems need no input: using controlled system of economical motor output (a rhythmic contraction of muscles). 


Exogenous rhythms extend far beyond local sound events in a room. Many are inaudible: the planet itself turns periodically on a tilted axis of its own, and also on a periodic cycle centered on the gravitational pull of the sun. We know this in the field of acoustics, but for whatever reason these factors have not been taken into account in developing theories of metrics in either generative phonology or of musicology \citep{berger2011}. 

\citep{palmer1997}
on the beat in music: longer durations, shorter attack 
attack-onset 

%
%Essens PJ. 1986. Hierarchical organization of temporal patterns. Percept. Psychophys. 40:69–73
%Essens PJ, Povel D-J. 1985. Metrical and non- metrical representations of temporal pat- terns. Percept. Psychophys. 37:1–7
%Performance expression also serves to differentiate among simultaneously occurring voices in multivoiced music. Voices can be distinguished by their intensity or timing. Early analyses of Duo-art (player piano) rolls indicated that pianists played tones comprising the melodic voice sooner than other tones notated as simultaneous (Vernon 1936). Recordings of wind, string, and recorder ensembles also indicated asynchronies among the voices for notated simultaneities, with a spread of 30–50 ms and a small relative lead (7 ms) of the instrument leading the ensemble (Rasch 1979). The amount of spread was larger for instruments whose rise (attack) time was longer, which suggests that musicians may adjust the asynchronies to establish appropriate timing of per- ceptual onsets

 Keeping time with external rhythms is the result of entrainment of our own rhythm to those in our environment. \citep{chronobiology, musicCognition} 

From the perspective of a percussionist, there is no issue with complex or odd meters. A novel pattern may be more difficult than a familiar one, but only while it remains novel: you might not produce a piece in 11/4 perfectly the first or second time you try it. Once had, it can be produced on demand: in synchronisation with others, and seemingly from ``thin air."

\citep{patel2005} defines metrical as "periodic at multiple time scales," but anything periodic would be such at multiple time scales, by definition of period and scale. The results confirm this as well. 






%Thus, such models predict improved synchronization to a beat when listening to metrical (vs. simple isochronous) sequences (E. Large, personal communication).Footnote 2 However, any improvement due to meter is likely to depend on the number and regularity of events that reinforce temporal structure at different time scales. Thus, an oscillator may be more strongly entrained when all cycles are marked by event onsets than when some event onsets are missing, e.g., due to syncopation (cf. Large and Jones 1999).



Recent work comparing metrical categories in infants and adults additionally supports the notion that the ``simple" meters demonstrated to be easier for westerner university students is the result of enculturalization rather than a cognitive predisposition. 
Much of the music from Eastern Europe, South Asia, the Middle East, and Africa has an underlying pulse of alternating long and short durations in a 3:2 ratio (Clayton, 2000). These complex meters, which are common in Bulgarian and Macedonian folk music, pose no apparent problems for adult and child singers and dancers from those countries (London, 1995; Rice, 1994). \citep{hannon2005}


This is in line with psychoacoustic findings of segmental categories of phonemes in infants and adults \citep{babies_r_l}. 

Rhythm and pitch are not disparate dimensions of music as stated by \citep{Krumhansl2000}. A frequency is what we perceive periodic events as at a certain threshold. Many humans start to hear repetitions faster than 50ms as a tone. \citep{wright2011}


%\section{Phonetic Realization of Trochees in Spoken Verse} 


%In order to discuss questions regarding word-prosodic Estonian rhythm in the context of musical rhythm, I define the acoustic and perceptual notions of both domains, beginning with musical meter more generally, and continuing into the Estonian language and relevant verbal art forms. Once established, I then give a thorough overview of
% previous research in the domain of song and word prominence in Estonian {\it regilaul}. All this culminates into the research questions and hypotheses that are the subject of this paper. 
%\section{musical meter}
%
%
%
%
%\section{Estonian Word Prosody}
%
%
%
%The rhythmic organization of song integrates the prosodic structure of the language with musical rhythmic principles \citep{palmerLinguisticProsodyMusical1992}. 
%
%Duration can be a phonetic correlate of stress, and it can also be independently contrastive at the segmental level \citep{lehistePhoneticsMetrics1992}. 
%
%
%Whether there is a correlation between poetic metre and the prosodic structure of a language.
%
%
%
%
%
%\citep{lehistePhoneticsMetrics1992} investigated the actual phonetic realization of different metres in different languages, finding evidence for one language's trochee, for example, to be realized in a way that is systematically different from another language's trochee. 
%
%Ross found vowel reduction in certain notes, but they were all unstressed, off-ictus for that song\citep{rossFormants90}.
%
%
%The role of primary stress in Estonian is described by Ilse Lehiste as {\it identificational} rather than contrastive \citep{lehistePhoneticsMetrics1992}. In other words, there are no stress minimal pairs at the lexical level, so the prominence cue is to indicate the onset of a new word. This is sometimes also called {\it demarcative} stress.
%
%
%
%Three proposed levels of stress: primary stress, unstress, and secondary stress. \citep{lippusAcousticStudyEstonian2014a}
%
%Primary lexical stress in native Estonian words is fixed, falling on word-initial syllables. 
%\cite{eekmeisterUralica98}
%\begin{exe}
%\ex \gll laul-da \\
%	{[ˈlɑuːl.dɑ]} \\
%	sing-\Tr{} 
%	\glt	`singing'
%\ex 	ööbik \\
%	{[ˈøː.pikː]} \\
%	nightingale.\Nom{} 
%	\glt`nightingale'
%\end{exe}
%
%Estonian has three syllable weights, also called degrees or quantities, that are contrastive in primary stress position. The first degree or Q1 is described as short, 
%In \ref{quant_cont}, we see two minimal pairs illustrating this contrast. 
%
%This ternary contrast has long been the subject of debate in the phonological literature of metrics: Q3 syllables have been analyzed both as a monosyllabic foot \citep{princeMetricalTheoryEstonian1980} and as a trimoraic syllable \citep{hayesCompensatoryLengtheningMoraic1989, kuznetsovaEstonianWordProsody2018,prillopMoraeEstonianReply2020}. 
%
%
% \begin{table}[htb]
%\centering
%\begin{tabular}{lcc}
%\hline
%
%Q1 &		 sada 		& 	kabi  \\  
%	&	 {\it `hundred'} 	&	 {\it`hoof' }\\
%\hline
%Q2 &		saada 		&	kapi \\
%	&	 {\it`send' }		&	{\it`of the cupboard' }		\\
%\hline
%Q3 &		saada 	&	 kappi 	\\
%	&	{\it`recieve' }	&	{\it`into the cupboard' }	\\
%\hline
%\end{tabular}
%\label{qexamps}
%\caption{ternary syllable weight contrast}
%\end{table}
%
%
%Detail the acoustic-phonetic features of syllable weight and stress in spoken Estonian alongside the relevant findings from perception studies. 
% 
%
%
%
%
%
%Duration can be a phonetic correlate of stress, and it can also be independently contrastive at the segmental level \citep{lehistePhoneticsMetrics1992}. 
%
%
%only two quantity levels of syllables in monosyllables: function words (CV) are taken to be light, and lexical monosyllables (CV(V)C(C)) strong.
%
%\paragraph{INTO PROSE}
%domain of syllabic quantity is evident whre the first, stressed syllable increases in duration with increasing degree of quantity, the following unstressed syllable decreases. 
%Lehiste's duration ratios. 
%Different f0 patterns in Q2 and Q3: early peak, dramatic fall in 3, late peak and no fall in 2 (contradicts that asymmetry paper that still saw falls in all three... that could be related to the Q/A issue! Dang.)
%Laboratory speech usually confirms temporal and tonal characteristics, but conversational speech shows only duration (ratio) as stable, with Q3 fall often absent. 
%
%RQ: can quantities in conversational speech be distinguished by acoustics alone, or are listeners making use of semantic context. 
%
%disyllabic words from recorded conversational speech presented without context. 
%
%Q3: V1 durations had strongest influence on listener decisions, followed by f0 change within V1. duration ratio had a weaker influence, was only significant for stimuli of single speaker. f0 movement across intervo aclic consonant not significant in any case. similar results for recognition of Q3 when presented in combination.
%
%For combined Q1 and Q2, only duration of V1 significant across speakers, duration ratio only relevant for (same as earlier) speaker. 
%
%f0 peak position had no significant effect in the cases where it was present. 
%
%for good recognition of all three quantities, duration of first vowel important. 
%
%differences in V1 duration robust, even in changing speaking rate. ``characteristic" fall in f0 of Q3 neutralized. 
%
% listeners did not recognize the majority of Q3 syllables in the absence of context. 
% 
% certain minimum duration of V1 needed for high recog rate of Q3, 
% 
% these were all words that could change in meaning with degree of quantity,
%\citep{krullUralica98} 
%doctrine of ternary contrasts, lol 
%a: three contrastive segmental lengths
%b: segment structure {\it and} prosody of stressed syllable
%c: in the foot
%
%\par
%
%domain of prosodic patterns is the foot, but only the stucture of the primary stressed syllable is relevant in determining the Q degree of both syllable and foot. (i.e., there is no Q3 foot that does not contain a Q3 syllable as its initial.) 
%
%Q1 and Q2 must have at least two syllables, may have three (trochee, dactyl). Having a third syll does not effect quantity. This is in contrast to finnish trisyllables in folksongs, which were 40\% longer than disyllables. 
%
%Second syllable does not influence the quantity of the first. If the duration of second syllable is predictable from Q of first syllable, this is what phonologists refer to as dependent features. !!!RATIO ARGUMENT
%
%monosyllabic Q3 feet in succession in connected speech {\it `khev `kõhn `poiss `läks `kepp `käes; `tõu `suur `selts `kond `likkus}
%
%ratio theory initially proposed by Lauri Posti in 1950. 
%
%length of the vowel of the second syllable is redundant, dependent, and predictable. 
%
%Q3 is monosyllabic foot, making ``disyllables" technically trisyllables... 
%
%from segments toward long syllables is turning point, segments are a failure (lol) 
%\citep{hintUralic98}
%
%oot quantity paradigm
%short-long not separate phonological categories. long monophthongs behave like diphthongs, geminates as consonant clusters
%
%(in english, short and long vowels have different qualities: Wiik 1965
%
%(stressed syllables) Q3 largest area, Q1 most centralized in F1xF2 plane. 
%\par
%
%when f3 and f4 are removed from spectrum, /i/ is perceived as /ü/, /e/ as /ö/
%state that contrast twixt above two vowel pairs on basis of f3. authors say that f3 being close to the strong f2 in round-front is `amplified' the cumulative of f2 and f3.
%
%conclude perceptual param f2 describes well the perceptual phenom governing this contrast. 
%
%``effective" f2' values??
%calculate with: Bladon, Fant 1978: 3 
%
%long and short vary very little in quality, defining as different phonemes based on length is not justified. 
%
%Q3 more ``prototypical" or best-contrast version of vowel phonemes in space of stressed syll
%
%in unstress: 
%
%Q1 {\it least} central, unstressed following Q3 init most centralized. V in sylls following Q2 intermediate between others. (ok, they are analyzing these with the ``feet" as having the quantity.. 
%
%/i/ most resistant to centralization
%
%\citep{rossFormants90}
%REASON TO RE-EXAMINE VOWEL FORMANTS/SPACE IN  REGILAUL!!! 
%also, data to support ``vowel space" as an available acoustic modification (measurable) in 
%singing, evidence favors reduction in word-level-weak syllables AND off-ictus are reduced, 
%but to what extent, and how compared across ictus-stress and ictus-quantity?
%TLDR are vowels reduced in singing compared to speech, or were those vowels reduced compared with strong positions of song, of word? etc. 
%all these ``long" notes are off-ictus: so the shorter notes are corresponding to HEAVIER and STRONGER syllables. 
%singer's formant in untrained female voice very unlikely. measure: LTAS for /a/,/e/, /i/, /u/. SPL of peaks around 3kHz ~30-40dB less than that of the first formant in all four vowels. 
%
%
%
%HOWEVER all vowels selected were in off-ictus-- HUAT
%FIND THIS SONG NOW
%
%READING LIST: Rossing et al 1987- formant SPL in opera and choir singing: singer's formant usually closer to and sometimes converging on f1 SPL. 
%
%SAMEAUTHORSAMESONG: no significant timing differences between performers with respect to note durations (Ross 1989)
%*******Discussion similarity f3 in spoken and sung vowels 
%Ternström and Sundberg 1989: caution against using f3 and f4 standard deviations
%
%standard dev for third and fourth formants less vowel dependent, more ``personal," especially compared to standard dev of f2 and f2 f1 f2 which are {\it fairly} independent of subjects!!!
%
%
%inverse filter results to confirm T n S 87: f3 lower in singing? 
%saw some similar patterns in with spoken data studies, but overall the size of the variation was small. For the third formant, deviations of the sung vowels from spoken tend to be minor and irregular  n = 2
%
%overall f1/f2: sung vowels cluster compared to spoken, specifically:
%f1 raised in everything but /a/,
%f2 lower in front, raised in back. 
%
%So, gradient modification of vowel quality contrast (less different than in speech).. but I am getting an impression of a larger overall vowel space
%
%
%\subsection{syllable prominence}
%
%Three proposed levels of stress: primary stress, unstress, and secondary stress. \citep{lippusAcousticStudyEstonian2014a}
%
%Primary lexical stress in native Estonian words is fixed, falling on word-initial syllables. 
%\citep{eekmeisterUralica98}
%\begin{exe}
%\ex \gll laul-da \\
%	{[ˈlɑuːl.dɑ]} \\
%	sing-\Tr{} 
%	\glt	`singing'
%\ex 	ööbik \\
%	{[ˈøː.pikː]} \\
%	nightingale.\Nom{} 
%	\glt`nightingale'
%\end{exe}
%
%Some loanwords will allow primary stress to fall on a non-initial syllable: {\it example, example}. 
%Borrowed names occasionally shift stress to the typical Estonian position: for this reason one could find both {\it Maria} and {\it Maarja} in the same classroom. 
%
%The role of primary stress in Estonian is described by Ilse Lehiste as {\it identificational} rather than contrastive \citep{lehistePhoneticsMetrics1992}. In other words, there are no stress minimal pairs at the lexical level, so the prominence cue is to indicate the onset of a new word. This is sometimes also called {\it demarcative} stress.
%
%Spectral tilt suggested, but no
%\citep{sluijterSpectralBalanceAcoustic1996, lippusAcousticStudyEstonian2014a} \\
%
%
%\begin{table}
%\centering
%\begin{tabular}{lcccr}
%a & & & & u \\
%\end{tabular}
%\label{vowelinv} 
%\end{table}
%Vowels in this position are rarely reduced, and the full inventory of vowels is allowed. A total of 36 diphthongs are allowed in primary stressed positions. All nine vowels can be the first portion of a diphthong, but only [ɑ e i o u] appear as the second portion of the diphthong \citep{asuEstonian2009}. 
%
%\begin{itemize}
%\item diphthongs 
%\item example
%\end{itemize}
%
%
%Unstressed syllables, often have reduced vowels, and are more restricted in inventory: only three diphthongs [ɑi ei ui] are allowed in this position \citep{lippusAcousticStudyEstonian2014a}. 
%
%Examined the phonetic correlates of stress in Estonian, but it was small scale and dealt only with nasal flow, amplitude, and duration, and at multiple levels of prosodic hierarchy \citep{gordonPhoneticCorrelatesStress1997}, but with as few as two participants. 
%More recently and with more statistical power, researchers measured mean F0, standard deviation of F0, vowel duration, and spectral emphasis. They found increased vowel duration to be the most important acoustic correlate of primary stress in Estonian words, and that Unstressed syllables in Estonian have been documented to attract creaky voice \citep{lippusAcousticStudyEstonian2014a}. 
%Expound on this study a bit more
%Estonian facts \citep{alma991001659729706011} \\
%A pattern of secondary stress (neutral) has been attested, though phonetic evidence is limited \citep{asuAcousticCorrelatesSecondary2018}. Only initial and peninitial syllables are examined in the present study.
%
%studies on the perception of these contrasts to be inserted in prose: 
%
%\citep{meisterPerceptionShortVs2011a, kaskPerceptualAsymmetriesAuditory2021, eekSimplePerceptionExperiments1997}
%
%
%\subsection{syllable quantity}
%Primary stress position is where the well-known ternary syllable weight contrast in Estonian appears.
%
%
%About this section: flows well, contains good information. Needs a little more information, as indicated below.
%Missing in this section: there's no discussion of what syllable weight/quantity means, either generally or in relation to Estonian. We represent syllable in terms of moras. What would the moaic representations for Q1, Q2, and Q3 be? Your designations short, long and overlong aren't specific enough and it's ot clear what the basis for the ratio is.
%
%
%Estonian has three syllable weights, also called degrees or quantities, that are contrastive in primary stress position. The first degree or Q1 is described as short, 
%In \ref{quant_cont}, we see two minimal pairs illustrating this contrast. 
%
%This ternary contrast has long been the subject of debate in the phonological literature of metrics: Q3 syllables have been analyzed both as a monosyllabic foot \citep{princeMetricalTheoryEstonian1980} and as a trimoraic syllable \citep{hayesCompensatoryLengtheningMoraic1989, kuznetsovaEstonianWordProsody2018,prillopMoraeEstonianReply2020}. 
%
%
%
%
% \begin{figure}[htb]
% \begin{center}
% \includegraphics[width=200pt]{figures/Quantity.png}\\
%
% \caption{Ternary Quantity Contrast {\citep{krullFOOTISOCHRONYESTONIAN1999}}}
% \label{quant_cont}
% \end{center}
% \end{figure}
%
% \begin{table}[htb]
%\centering
%\begin{tabular}{lcc}
%\hline
%
%Q1 &		 sada 		& 	kabi  \\  
%	&	 {\it `hundred'} 	&	 {\it`hoof' }\\
%\hline
%Q2 &		saada 		&	kapi \\
%	&	 {\it`send' }		&	{\it`of the cupboard' }		\\
%\hline
%Q3 &		saada 	&	 kappi 	\\
%	&	{\it`recieve' }	&	{\it`into the cupboard' }	\\
%\hline
%\end{tabular}
%\label{qexamps}
%\caption{ternary syllable weight contrast}
%\end{table}
%
%
%\gll valge lauade vahele, \\
%white boards between \\
%\glt in between white boards, \\
% 
%Syllable weight or quantity also influences the otherwise predictable stress pattern. When a Q3 syllable is not in the first position, primary stress falls on it: this usually happens in the case of borrowed words. Otherwise, primary word stress is on the first syllable \citep{lehisteFunctionQuantityFinnish1965}. 
%  
%  \begin{table}[htb]
%\centering
%\begin{tabular}{lcr}
%\hline
% 	 Q1 – short 	&	ratio ~2/3 \\
%	 Q2 – long		& 	ratio ~3/2	\\
%	 Q3 – overlong	&	ratio ~2/1 	\\
%	 \hline
%\end{tabular}
%\label{stressick}
%\caption{c}
%\end{table}
%
%In the initial syllable of a word, syllable duration increases with syllable quantity, while the second syllable in the foot decreases as the first syllable's quantity increases. 
%
%It has been suggested, that the duration of the second vowel is a diagnostic of the quantity of the preceding syllable (duration ratios) \citep{lehisteFunctionQuantityFinnish1965, lehisteProsodicChangeProgress2003}. It has been suggested that this is due to the 
%a documented tendency for foot isochrony, wherein the total duration of disyllabic feet varies little in both spontaneous and read speech \citep{krullFOOTISOCHRONYESTONIAN1999, traunmullerEffectLocalSpeaking2003}resulting in the inverse relationship between the durations of the two syllables in a foot: the longer the first syllable, the shorter the second. More specifically: a second syllable that follow an overlong syllable will be realized shorter than a second syllable that follows a long syllable, with the longest second syllables following short (Q1) first syllables.  
%%\footnote{further discussion of perception of ternary quantity and duration ratios to put into prose \citep{foxDiscriminationDurationRatios1987,foxDiscriminationDurationRatios1989}}
%
%
%Duration ratios, but all proportionally 50\% differences. The ``long" second syllable of Q1 is shorter still than the ``short" second syllables of Q2 and Q3 feet. 
%\citep{lehiste1960segmental}
%\subsection{perception of word prosody}
%
%
%
%
%suggested analysis as types of foot accents rather than syllable degrees. 
%rising or falling f0 alone was insufficient to change the perceived category of all three weights.
%lengthening V2 of third quantity, but maintaining its reduction, did not suffice to change the perceived cat of A3 to A2: only when the tone contour also changed to rising did the nucleus duration facilitate. 
%reducing quality and shortening duration of nucleus V2 of an A2 alone also did not change perception to A3: only when falling tone added.
%\citep{eekSimplePerceptionExperiments1997}
%
%
%
%
%foot isochrony resulting in inversely proportional duration ratios of S1:S2 syllables, or more plainly that as S1 increases in quantity, S2 nucleus decreases. \citep{eekSimplePerceptionExperiments1997}
%
%Q3 vowels described as having especially large standard deviations \citep{eek1975}. 
%
%\section{Emphasis in Songs}
%\subsection{strong beats}
%
%\section{Template for {\it runosong}} 
%Kalevala and regilaul 
%\citep{sarv1998language}\\
%The singing of folksongs has long been an important part of the Estonian national heritage, and large scale community gathering to sing traditional {\it regilaul} songs is argued to have aided the preservation of Estonian cultural and linguistic heritage through long periods of occupation. \citep{bruggemannSingingOneselfNation2014}. The first documented song festival in Estonia was held in the seventeenth century \citep{ruutelTRADITIONALMUSICESTONIA2004}.
%
%Most descriptions of {\it regilaul} will describe it as ``non-western" folk music, though the definition of western escapes the author. It is perhaps the reduced use of tempered scale, although the songs have distinct tonal centers, and the melodies rise and fall in intervals from it.  The ``non-western" ness could equally be the lack of ``end-rhyming," a signature innovation in newer Estonian songs, as traditional {\it regilaul} rhymes little in favor of alliteration and parallelism. If you listen to them, though, even completely naive of the Estonian language, you will not mistake them for anything but a folksong, and a strictly structured one at that. 
%
%A song called {\it regilaul} is a type of Estonian folksong, found within the greater songwriting tradition of Balto-Finnic language family: variations on these songs are found in Finnish, Karelian, Votic, and Ingrian traditions \citep{tormisKalevalaEstonianPerspective1985}. 
%
%Collectively, songs in this tradition are often referred to as ``runosongs" or ``runic songs." 
%
%These runic songs are set in what scholars refer to as The ``Kalevala" metre, named for the title of the epic poem in Finnish; Estonia's own version is called ``Kalevipoeg" {\it Kalev's son}. 
%
%In the story, Kalevipoeg is a giant with a hedgehog for a companion. 
%
%The poem contains songs, too: a consistent theme across these poems and Estonian folklore generally has protagonists requesting song lyrics and incantations from deities, often in pursuit of skill in dance or music. 
%
%In studies of metre, a trochee is a disyllabic sequence with a strong first syllable and a weak second. In {\it regilaul} songs, the strong position of a two note sequence is called ``ictus," and the weak position is called ``off-ictus"
%
% The basic template of the Kalevala metre is four trochees per line, for a total of eight syllables. There are of course variations, but the basic invariant {\it regilaul} verse line will contain eight beats evenly divided into a single measure: in a 4/4 song, each of the eight syllables corresponds with one eighth note. In this case each eighth note will correspond to one syllable in the text, the constituent henceforth referred to as the ``syllable-note" \citep{ruutelResultsComputerizedComparative1999}. The late composer and {\it regilaul} revitalist Veljo Tormis referred to the runic songs as ``singable songs" \citep{tormisProblemsThatRegilaul2007}. 
% 
% ``collaborative outcomes" 
% in regilaul, the musical rhythm is directly connected to the verse structure \citep{orasMusicalManifestationsTextual2010}
% 
% Due to the strict metrical parametres of the form, any runosong text can be sung to any runosong melody. An example of this in English: one can sing {\it Amazing Grace} to the tune of {\it House of the Rising Sun}, and vice a versa.\footnote{Also try TLC's ``Scrubs" exchanged with America's ``Horse With No Name"}. 
%\begin{figure}[htb]
%\begin{center}
%\includegraphics[width=300pt]{figures/069.png}
%\caption{Öised orjad, performed by Liisu Orik}
%\label{song69}
%\end{center}
%\end{figure}
%
%In \ref{song69}, we see the musical notation of a typical {\it regilaul} verse: eight notes evenly divided into a measure (eight eighth notes) and corresponding one syllable per eighth note as indicated by the text. The refrain {\it kas`-ke}, which is repeated after every verse line, is part of a separate musical phrase, though it is not a full measure on its own (it contains only three beats). This is indicated by the dashed line following the first measure of the verse. The solid bar after the {\it kas`-ke} refrain indicates the onset of a new ``regular" measure, another {\it regilaul} verse line of eight syllables evenly divided into eighth notes in the measure. 
%
%\begin{figure}[htb]
%\begin{center}
%\includegraphics[width=300pt]{figures/094.png}
%\caption{`The King Game" as performed by Liisa Kümmel on track 94 of the anthology}
%\label{kinggame}
%\end{center}
%\end{figure}
%
%In \ref{kinggame} we see an example of extra syllables fit into the Kalevala metre. The first measure in the figure is another refrain. Looking at the verse annotation, we see that there are in fact nine syllables instead of the requisite eight. The measure accomodates the extra syllable by dividing one of the eighth notes into two sixteenths. Thus, the one to one ratio of syllable to note remains, but only seven of the syllable-notes are annotated as isochronous. 
%
%Only those syllable-notes which are in evenly divided verse measures are the subject of this study. Non-verse notes, such as those in refrains, as well as notes in verses that occupy a different categorical duration than other notes in its measure are not included. 
%
%
%The composer Veljo Tormis refers to folksongs in this tradition as ``singable song," 
%
%shared structure of other Finno-Ugric language family, i.e., %
%%
%the singable song 
%
%
% ancient epic tale shared by many members of the Finno-Ugric language family
% 
% The common folklore of Finno Ugric language family: Kalavala in Finland, Kalevipoeg (Kalev's son) in Estonia. Kalev is a giant. His companion is a hedgehog.  
% many stories involve a character seeking incantations (song lyrics) to acquire some skill
% 
%
%
%\subsection{Previous phonetic studies of Estonian metrics and {\it regilaul}}
%
% \citep{palmerLinguisticProsodyMusical1992} hypothesize that prosodic stress and musical metre align in song, but their measurements were only of English vocalists and ``western" music. 
%English vocalists were asked to perform songs in which stressed syllables aligned or clashed with the musical meter, finding relative increases in syllable duration in at both word-level and song-level positions of prominence \citep{palmerLinguisticProsodyMusical1992}, suggesting that the two systems contribute separately but use similar organizational properties.
%
%Ilse Lehiste analyzed spoken poems in Estonian and Finnish, examining the differences in quantity relations under metrical subjugation and found that in strict metrical forms (i.e., the Kalevala metre), the absolute durations of Q2 and Q3 metrical feet lost their contrast, but retained the approximate ratio of durations between the first and second syllable. In freer verse, the differences in absolute durations remained, suggesting that the duration ration between Q2 and Q3 is an important cue for contrast, especially in the contexts of temporal demands of an imposed paralinguistic metre. 
%\citep{lehistePhoneticsMetrics1992} Later, in collaboration with Jaan Ross, three papers examined the phonetics of metrics in old Estonian {\it regilaul} folksongs. The first (Funeral laments study )\citep{rossLostProsodicOppositions1994}, finding that duration was best predicted by metrical position and not syllable quantity. 
%
%
%
%
%They later found that duration was a better predictor of ictus  than of word stress: stressed syllables in off-ictus lost their durational contrast with unstressed syllables.  \citep{rossTradeoffQuantityStress1996} 
%
%In \citep{rossTimingEstonianFolk1998} Ross and Lehiste measure syllable duration ratios for Q1 initial syllables of disyllabic feet: falling in ictus position and falling in off-ctus position. They found that the duration ratios of Q1 syllables in ictus were greater than those in off-ictus. In ictus position, Q1 syllables were roughly the same length as each other, while those in off-ictus were shortened. They conclude the duration cue for contrastive stress is subordinated to the song. 
%This paper aims to extend the findings of the aforementioned studies \citep{lehistePhoneticsMetrics1992, rossLostProsodicOppositions1994,rossTradeoffQuantityStress1996,rossTimingEstonianFolk1998} by annotating a larger corpus of {\it regilaul}, to compare conflicts of song stress (ictus) and word stress 
%and investigate whether syllable quantity contrast is preserved in song. Vowel duration is an acoustic correlate of both syllable stress and quantity, and so I measure vowel duration in the context of performed {\it regilaul} songs. 
%
%Vowel duration, if subordinated to the metre of the song, would confirm Ross \& Lehiste's results \citep{rossLostProsodicOppositions1994, rossTimingEstonianFolk1998, rossTradeoffQuantityStress1996}. 
%
%If this question has already been investigated, what motivates the author to pursue it further? First, the aforementioned studies were limited in sample size: each one examined only one word-level interaction with song-level prominence, when both are well documented to influence duration in spoken Estonian, and with each other. Thus, the present study examines all three predictors of prominence: song ictus, word stress, and syllable quantity. I also exponentially increase the {\it n} of trials for increased robustness, and then analyze the dataset using Bayesian modeling techniques to capture the nuanced covariance of the three predictor categories. In addition to replicating duration measurements undertaken in earlier studies, I also introduce vowel space as a potential predictor for word-level prominence. 
%
%more recently, text setting in regilaul
%\citep{orasMusicalManifestationsTextual2010}
%
%
%
%
%A good deal of literature has been devoted to the classification of languages by typological rhythms: in particular, ``stress-timed," ``syllable-timed," and ``morae-timed" systems have been proposed. An extremely reductionist description of these typologies would assert a relative isochrony of the named constituent. That is, a language that has been described as ``stress-timed" is said to be "timed" according to, and having relative isochrony of stressed syllables. Likewise, a syllable-timed language would be described as having syllabic isochrony, and so on. Many of these notions are based on the intuitions of native speakers, however, supporting phonetic data is limited  \citep{kohlerRhythmSpeechLanguage2009}. 
%%add later!!: (Bertinetto 1989) (Kohler 2009a, b). 
%More recently, alternate rhythmic metrics have entered the discussion in support of timing classes: i.e., Pairwise Variability Index (PVI), normalized Pairwise Variability Index (nPVI), both of which use the relationship of durations of adjacent syllables to describe their timing classes.\citep{arvanitiRhythmClassesSpeech2012}. English, for example, has higher syllable PVI (Pairwise Variability Index) than Estonian, due to English's drastic reduction of weak syllables, and the presence of prominent short syllables ( i.e., syllable durations in English are more variable than in Estonian)\citep{asuEstonianEnglishRhythm2006}. 
%
%
% 
% 
%\begin{itemize}
%\item is stress or ictus a better predictor of vowel duration?
%\item is the ternary syllable quantity contrast preserved in {\it regilaul} songs?
%\end{itemize} 
%
%\subsection{Design of the Study}
%
%Measurements:
%
%vowel duration
%
%vowel dispersion
%
%stressed syllable less reduced, more intelligible, 
%
%Fine-grained acoustic-phonetic cues to stress (cross-linguistic) 
%\citep{lindblom1990,moonInteractionDurationContext1994, de1995supraglottal, bradlowIntelligibilityNormalSpeech1996,smiljanicProductionPerceptionClear2005} \\
%\subsection{On foot isochrony and duration ratios} 
%
%However, in both cases, the presence of foot isochrony and consistent duration ratios across these three types of feet would not indicate that the foot is the domain of the contrast, nor that the ratio itself is the measurement of importance. Instead, these two findings point to the level of contrast between Q1, Q2, and Q3 syllables at the level of the syllable, and the resulting fluctuations in ratios of first and second syllables as inevitable consequences of preserving the contrast in initial position. 
%
%Another consideration is that the three quantities are not restricted to length contrasts of identical segments, in the case of the geminates seen in minimal triads, but also can be comprised of diphthongs and complex codas {\it in addition} to geminates. 
%, i.e., ui: (diphthong ending in long V), diphthong complex coda geminate consonant, etc. \\ 
%
%
%contain more phoneme segments. The minimal triads illustrate the segmental length contrasts that provide evidence for the three weights of syllables in primary stress position, but they do not adequately convey that the different quantities are not simply different lengths of segments, but that with an increase syllable quantity comes an increase in the quantity of segments contained within that constituent. To illustrate this, the table in \ref{} contains representations of available segment combinations in each syllable, and examples of Estonian near-minimal triads in disyllabic words. 
%

\chapter{Handout}


The rhythmic organization of song integrates the prosodic structure of the language with musical rhythmic principles \citep{palmerLinguisticProsodyMusical1992}. 

Duration can be a phonetic correlate of stress, and it can also be independently contrastive at the segmental level \citep{lehistePhoneticsMetrics1992}. 


%Whether there is a correlation between poetic metre and the prosodic structure of a language.

%A trochee is a strong, stressed syllable followed by a short(er), weak syllable. 




\citep{lehistePhoneticsMetrics1992} investigated the actual phonetic realization of different metres in different languages, finding evidence for one language's trochee, for example, to be realized in a way that is systematically different from another language's trochee. 

Ross found vowel reduction in certain notes, but they were all unstressed, off-ictus for that song\citep{rossFormants90}.


The role of primary stress in Estonian is described by Ilse Lehiste as {\it identificational} rather than contrastive \citep{lehistePhoneticsMetrics1992}. In other words, there are no stress minimal pairs at the lexical level, so the prominence cue is to indicate the onset of a new word. This is sometimes also called {\it demarcative} stress.



Three proposed levels of stress: primary stress, unstress, and secondary stress. \citep{lippusAcousticStudyEstonian2014a}

Primary lexical stress in native Estonian words is fixed, falling on word-initial syllables. 
\cite{eekmeisterUralica98}
\begin{exe}
\ex \gll laul-da \\
	{[ˈlɑuːl.dɑ]} \\
	sing-\Tr{} 
	\glt	`singing'
\ex 	ööbik \\
	{[ˈøː.pikː]} \\
	nightingale.\Nom{} 
	\glt`nightingale'
\end{exe}

Estonian has three syllable weights, also called degrees or quantities, that are contrastive in primary stress position. The first degree or Q1 is described as short, 
%In \ref{quant_cont}, we see two minimal pairs illustrating this contrast. 

This ternary contrast has long been the subject of debate in the phonological literature of metrics: Q3 syllables have been analyzed both as a monosyllabic foot \citep{princeMetricalTheoryEstonian1980} and as a trimoraic syllable \citep{hayesCompensatoryLengtheningMoraic1989, kuznetsovaEstonianWordProsody2018,prillopMoraeEstonianReply2020}. 


 \begin{table}[htb]
\centering
\begin{tabular}{lcc}
\hline

Q1 &		 sada 		& 	kabi  \\  
	&	 {\it `hundred'} 	&	 {\it`hoof' }\\
\hline
Q2 &		saada 		&	kapi \\
	&	 {\it`send' }		&	{\it`of the cupboard' }		\\
\hline
Q3 &		saada 	&	 kappi 	\\
	&	{\it`recieve' }	&	{\it`into the cupboard' }	\\
\hline
\end{tabular}
\label{qexamps}
\caption{ternary syllable weight contrast}
\end{table}

%\chapter{Regilaul: Estonia's {\it runosong} tradition}
\index{Runsong and the Kalevala}@\emph{Runosong and the Kalevala}



\begin{figure}[htbp]
\begin{center}
\includegraphics[width=300pt]{figures/094.png}
\caption{music notation of ``The King Game" as performed by Liisa Kümmel}
\label{The King Game}
\end{center}
\end{figure}

\section{The Runic Song Tradition} 


A given {\it regilaul} verse line contains eight beats, generally evenly divided into a single measure so that each of the eight syllables corresponds with one eighth note. The ictus position of a song's line is thus determinable by counting every other beat as ictus, starting with the first.

The first song festival in Estonia held in \cite{ruutelTRADITIONALMUSICESTONIA2004}. 


One runic invariant is that the tonal center is placed on ``the both syntactically stable 


dominating pitch value of the most stable syntactical positions of a given melody's typological group


the syllable-note is the basic unit of the melody-line. Thus syllables take up the space of the note in their position of the melody as prescribed. 


Melodic accents coincide with the stressed syllables, the pitch resolves as the phrase resulves. Said to be a musical abstraction of the natural prosodic intonation of the 8 syllable (spoken) runoverse. 
\cite{ruutelResultsComputerizedComparative1999} 

%
%\section{The singing giant: a common folklore epic} 
%Estonian, Finnish, Karelian, Ingrian runic songs
%
%whether it be Finnish, Estonian, Karelian, Votic, Ingrian and to same extent even Livonian an Veps)
%
%
%the singable song \cite{tormisKalevalaEstonianPerspective1985}
%
%
% ancient epic tale shared by many members of the Finno-Ugric language family
% 
% The common folklore of Finno Ugric language family: Kalavala in Finland, Kalevipoeg (Kalev's son) in Estonia. Kalev is a giant. His companion is a hedgehog.  
% many stories involve a character seeking incantations (song lyrics) to acquire some skill
% 
 
 \cite{tormisProblemsThatRegilaul2007}
\section{regilaul in prosodic contexts}



\cite{lehistePhoneticsMetrics1992}








 
 
\section{Previous Studies with regilaul}

\begin{figure}[htbp]
\begin{center}
\includegraphics[width=300pt]{figures/069.png}
\caption{default}
\label{default}
\end{center}
\end{figure}

\cite{rossStudyTimingEstonian1989,rossLostProsodicOppositions1994,rossTradeoffQuantityStress1996,sargDoesMelodicAccent2006, sargMelodicAccentEstonian2007,orasMusicalManifestationsTextual2011}

%
\chapter{Methods}
\index{Methods@\emph{Methods}}%

\section{Design}
To test this hypothesis, the vowel durations of isochronous first and second syllable-notes of polysyllabic words are measured using a semi-automatic transcription approach. A tempograph of the entire song is referenced to identify whether or not a given syllable-note is `1on" or ``off" the beat relative to local and global tempo. 

Then, using the text transcriptions of the verse lines, the locations of syllable nuclei are identified using a forced aligner. 


Once located, the duration of a given vowel is defined as the difference between onset (within 10\% of the sustain level) of the ADSR envelope and the onset (within 10\%) of the release of the note. This is to account for temporal variation in syllable nuclei that arises due to environmental factors: consonant identities differing according to manner and place. Measuring nucleus duration by the sustain level subtracts the heavily coarticulated subsegmental transitions times that could be said to belong to both or either adjacent segments. 








%
%Within each song, samples of syllable-notes consisted only of those matching the nominal isochrony. For example, in verse lines mostly comprised of syllable-notes divided evenly into eighths, neither quarter notes nor sixteenth notes were taken. Syllables in phrase-final position were generally excluded under this criteria. 


\section{Constructing the Corpus}



\subsection{Materials}

Songs for this paper were accessed via The Anthology of Estonian Traditional Music \citep{tampere2016}, which includes audio recordings with transcriptions of the lyrics. 

%Originally published on four vinyl discs in 1970, the digital version showcases a robust sample of the massive collection of {\it regilaul} in Estonian Folklore Archives. In addition to audio, the  compilation includes  photographs, sheet music, and performer demographics of 98 {\it regilaul} songs and 17 instrumental tunes. 

The regilaul sample in this paper is a subset of the entire anthology. To control for regional differences \citep{whyRegion},  songs were all from Pärnumaa county. In order to account for diachronic variation \citep{whyDecade} as well as possible differences due to recording equipment used, all songs came from the same 5 year span from 1961-1966 and were recorded on behalf of the Estonian Folklore Archives by Herbert Tampere, Erna Tampere, and Ottilie Kõiva for the Estonian Folklore Archives\citep{oras2002, tampere2016}. 




\subsection{Annotating the Song Audio }

 Each song's lyrics are copied from the site and saved as .txt files in Estonian orthography, each line of the file corresponding to one melody line.  
Audio files of the selected songs are downloaded from the archive in .ogg format, which is the highest resolution of the two lossy formats available from the digital anthology. Each song is then imported into a Logic Pro X \citep{logic2014} session for preprocessing. 
Since a main predictor in the hypothesis refers to the notes with relation to the beat, it is necessary to annotate the location of beats for each measure in the song while accomodating the fine-grained variation in absolute duration and tempo inherent in natural meter. 
Beat onset detection in general works by analyzing the signal's acoustic dimensions with relation to the downbeat \citep{bKeeper2007}. 
Once the tempo map is aligned with the downbeat of each measure, Musical Instrument Digital Interface (MIDI) is used to synthesize a metronome to indicate beat location and duration with respect to the tempo variation in each measure. This metronome is converted into a PRAAT\citep{boersna2022} TextGrid with an interval tier indicating the onset and offset of the metronome's notes, which correspond to syllable-notes falling ``on" the beat. 


 
Lyrical transcriptions of each verse line is added to another interval tier, with each line corresponding to four measures indicated by the click tier.  Here, the eSpeak forced aligner for Estonian \citep{eSpeak1995} attempts to align the transcription from orthographic verse line to two tiers: one with phonemic transcriptions of words, and one with individual phonemes. Due to its inherent design for speech, the aligner frequently needed manual adjustment of the word boundaries before reliable phonemic alignment could be produced. 

A subset of vowel intervals from initial and pen-initial syllables of polysyllabic words is the data in this study. 



%The beginning of the vowel was aligned according to a combination of acoustic correlates: 
%
%
%
%\begin{itemize}
%\item vowel considerations
%\begin{itemize}
%\item {\bf intensity(dB) contours}: intensity within 2dB of the steady-state medial portion of the vowel, with a positive slope less than one.
%\item {\bf fundamental frequency(Hz)} stabilizing into that syllable-note's pitch category
%\item {\bf resonant frequencies F1, F2, and F3}: visible and approaching steady-state, combined wtih  the presence of a voicing bar.
%\end{itemize}
%\item onset manner particulars: 
%\begin{itemize}
%\item following {\bf plosive onsets}, vowel boundary is after a burst
%\item following {\bf fricative onsets},  the end of visible high-frequency noise in the spectrum was reliable. The phoneme /s/ also consistently showed a carat in the frequency track immediately preceding the transition to vowel.
%\item boundaries between {\bf approximant onsets} and vowel nuclei were determined chiefly by the steady state of formants. 
%\item following {\bf nasal onsets}, vowel intensity actually lowered, but a negative slope approaching zero reliably followed the offset of visible anti-formants.
%\end{itemize} 
%\item The offsets of vowels into codas:
%\begin{itemize}
%\item negative slopes approaching zero for intensity; positive preceding nasal. 
%\item Formants were allowed more variation in transitions to codas
%\item approximant codas {[l, j, w]} were excluded from the dataset, as neither the forced-aligner nor the phonetician could determine a reliable way to define the boundary between them.
%\end{itemize} 
%\end{itemize}
%
%
%
%In cases of vowel adjacency across syllable boundaries, the presence of a visible glottal stop and support from other criteria would qualify both for inclusion. In the absence of these cues, both nuclei were excluded from the measurements. Other exclusions were due to ambient noise (i.e., churchbells in song 41), ambiguity of word boundaries due to wordplay (consulted with native speaker), and cases of adjacent identical vowels across word boundaries. 
%%epenthesized vowels (i.e., pandi mind paju raiumaie) having mind(e) \\
%
%
%In all cases, if the aforementioned cues were unavailable, ambiguous, or misaligned, the token was elided for this analysis. From all nine songs in the corpus, a total of 757 vowel nuclei met the criteria for inclusion in duration measurements. 

%For the measurements of F1, F2 space, formants were extracted from the mid point of the vowel. 


\subsection{Aggregating Acoustic and Text data}

The last step in preprocessing is to integrate the audio annotation with the lexical content of the song lyrics. This task was accomplished using an open-source natural language toolkit in python called estinltk \url{https://github.com/estnltk} \citep{estnltk2020}. For each vowel interval included  in the study, the vowel's word context from the corresponding TextGrid tier is input into estnltk vabamorf syllabifier, which returns the word broken into individual syllables with corresponding prominence data for each syllable. Vowel duration is measured by the difference between the offset and the onset of the interval on the vowel's TextGrid tier. The estnltk data and PRAAT measurements are aggregated using the parselmouth library\citep{parselmouth2018, python1995}. Jupyter notebooks illustrating each step are available at  \url{https://github.com/sally-ran-some/eesti_regilaul}. 


%Linguistic descriptions of the Estonian language date back as early as the seventeenth century, but the ternary quantity contrast was not documented until native Estonian linguists contributed their intuitions. The non-native linguists had only described lexical stress \citep{sargEarlyHistoryEstonian2005a}. \\



\section{Statistical Analysis} 
For each hypothesis, linear mixed-effects models were fit using the lme4 package in R \citep{lme4,r2022}. All the models, random slopes and intercepts, comparisons, etc. 


Anova comparison of the maximal design model with a null model is also statistically significant (p<0.001***). 


%\chapter{Results}
\index{Results@\emph{Results}}%

%\begin{wrapfigure}{L}{0.5\textwidth}
%\centering
%\includegraphics[width=0.5\textwidth]{/Users/sarah/Git/regilaul_project/manuscript/main/figures/quictus_durmdl.png}
%\caption{quantity model}
%\label{qmdlt}
%
%
%\end{wrapfigure}





The presence of a coda in comparison to an open syllable-note necessitates a shorter vowel to fit the consonant into the note duration. 




%\ref{qmdlt} shows the full output of the linear mixed effects model. 

%Regilaul's metrical pattern avoids short stressed syllables (Q1) in ictus position, preferring Q2 or Q3 (long and overlong) syllables to fall on the beat. In off-ictus position, short stressed (Q1) syllables are preferred while Q3 avoided.
%The ternary quantity contrast occurs only in primary stressed syllables. To analyze vowel duration measurements in all three quantities, a subset of only primary stressed syllables is taken from the dataset. 




%
%\begin{figure}[ht]
%\centering
%\includegraphics[width =\textwidth]{/Users/sarah/Git/regilaul_project/manuscript/results/q_dur.png}
%\caption{density plot of vowel durations in three syllable quantities}
%\label{qdur}
%
%\end{figure}
%\begin{wrapfigure}{l}{\textwidth}
%\centering
%\includegraphics[width =\textwidth]{/Users/sarah/Git/regilaul_project/manuscript/results/q_dur.png}
%\caption{density plot of vowel durations in three syllable quantities}
%\label{qdur}
%
%\end{wrapfigure}

%
%\ref{qdur} shows the vowel durations of all three syllable quantities, grouped by ictus and off-ictus positions in the song. In ictus position, median vowel duration descends as quantity increases, with the greatest difference between Q1 and Q2. Off the beat, a similar descending pattern is evident: however, in this case the largest difference is between Q2 and Q3 syllables. 
% The intercept is set at ictus position, Q1. Findings are significant results for Q2(p<0.001), Q3, and off-ictus positions (p< 0.05). Comparison with null model was statistically significant (p<0.001). For full model output see \ref{qdurfixed} and \ref{qdurrandoms} in Appendix A. In context of earlier findings that the quantity contrast was ``lost" at the syllable level, the decrease in vowel duration as syllable weight increases supports the notion that the contrast is preserved at the segmental level. That is, rather than the full syllable lengthening in duration, the song-level isochrony of syllable-notes results in vowel nuclei shortening to accommodate codas in Q2, and further for geminates and complex codas in Q3. 
% 
% 
%A null model constructed containing only random effects was compared to the design model by two-way ANOVA. Results are significant for the design model (p < 0.001***).


\section{Stress and Unstress}

\begin{figure}[ht]
\centering
\includegraphics[width = \textwidth]{/Users/sarah/Git/regilaul_project/manuscript/results/dur_density_qfac.png}
\caption{vowel durations of CV and CVC syllables in stressed and unstressed positions falling on and off the beat}
\label{durstrick}
\end{figure}

The two graphs in \ref{durstrick} illustrate the overall pattern of interaction between musical meter and word stress. When syllables are on the beat, the vowel duration normally assoficated with differences in word stress are not apparent. Off the beat, however, stress and unstress remain distinct. This pattern is the same in both open (CV) and closed (CVC) syllable shapes, with overall vowel duration being shorter in closed syllables. 


%In open syllables, ictus position predicts longer vowels in both stressed and unstressed syllables, while stressed syllables are longer overall than unstressed. 

%In Q2, we see longer vowel durations for ictus position in stressed syllables, and higher means for ictus position in unstressed, though the distributions overlap much more here. \\


%\begin{figure}[ht]
%\centering
%\includegraphics[width=\textwidth]{/Users/sarah/Git/regilaul_project/manuscript/results/dur_shape_ictus.png}
%\caption{vowel dur(s) by beat position and syll. shape}
%\label{ickdursh}
%\end{figure}


%
%\begin{figure}[ht]
%\centering
%\includegraphics[width=\textwidth]{/Users/sarah/Git/regilaul_project/manuscript/results/dur_shapestress.png}
%\caption{vowel dur(s) by word stress and syll. shape}
%\label{strdursh}

%\end{figure}



%
%The graph in \ref{ickdursh} illustrates the distribution of vowel durations in different syllable shapes falling on and off the beat. 
%
%A similar pattern can be seen in \ref{strdursh}, where the stressed or unstressed status is shown instead. At both song and word levels of prominence, CV or Q1 syllables are the longest, gradually decreasing in CVC and CVCC, both of which are Q2 syllables.  This further confirms the gradience of the quantity contrast at the segmental level. 


\begin{wrapfigure}{L}{0.5\textwidth}
\centering
\includegraphics[width=0.5\textwidth]{/Users/sarah/Git/regilaul_project/manuscript/main/CVmodout.png}

\caption{stress and ictus duration model output}
\label{strickmdl}
\end{wrapfigure}

%%%%%%%%%%%%%%%%%%%%%%%%%%%%%%%%%%%%%%%%%%

%
%%%%%%%%%%%%

%\subsection{Vowel Dispersion}
%A subset of the Q1 and Q2 vowels used for duration measurements above is taken,  containing only those five vowel phonemes which occur in both stressed and unstressed syllables at the word level: \(a, e, i, o, u\). The total number of vowels in this set is {\bf N}.
%To account for physiological differences between singers, vowel dispersion is calculated as the euclidean distance of each token from the respective singer's vowel center in the (F1, F2) space. 
%
%
%\begin{figure}[htb]
%\centering
%\includegraphics[width=\textwidth]{/Users/sarah/Git/regilaul_project/manuscript/results/space_strictus.png}
%\caption{euclidean distance of vowels in stress and ictus}
%\label{spcstrick}
%
%\end{figure}
%A pattern is marginally visible in the two graphs  \ref{spcstrick}, faceted by Q1 and Q2. Notice that in off-ictus position, vowel dispersion means of stressed syllables are higher than those of stressed syllables in ictus position. This indicates that stressed syllables falling off the beat are being compensated for their shortened vowels by way of increased articulation of quality. This pattern, however, doesn't shake out as statistically significant in the model. \\
%A linear mixed effects model for vowel dispersion is constructed, but only the uninterpretable intercept shows significance. 
%Comparison with null model is not statistically significant. Thus in the case of vowel dispersion, we fail to reject the null hypothesis. This could be due to the relatively smaller size of the data subset. 
%
%

%
%
%
%\chapter{Discussion}
\index{Discussion@\emph{Discussion}}%

\section{Temporal Prosodic Features Crystalized at Segmental level in isochronous syllables}

These results support the hypothesis that prosodic timing modifications resulting from the synchronization of independent rhythms (in this case, syllables and notes into syllable-notes) do not result in the subordination of one system to another. In this case, when syllables and notes become one timing unit, the temporal acoustic correlates often found at suprasegmental levels (syllable, foot, word) are found at the segmental levels. 

This interaction is then more analagous to entrainment of two independent rhythmic systems than to language being forcibly pigeon-holed into the metrical structure of music. 


\section{Vowel Dispersion}
Results for the predictive power of ictus and off-ictus, stressed and unstressed syllables and vowel space dispersion were not significant. The patterns that seem to emerge visually in the graphs of distributions suggest that the situation might be ameliorated by increasing the dataset. For this measurement, there were fewer available tokens than for vowel duration, and therefore less statistical power. Further examination is needed to determine whether or not the observed pattern is simply lacking in power or whether the premises that lead to this hypothesis are missing something entirely. 


\section{Future Studies} 

This study used a sample of nine songs and three singers, all recorded in the 1960s. Annotation has already begun on the remaining songs that fit into this sample's criteria: In total, there are seventeen songs and seven singers from Parnumaa county recorded in the 1960s. Increasing the size of the audio corpus would facilitate exploratory analysis in countless dimensions of music and language, and also shed light on the issue of vowel space unresolved here. 
Several of the singers featured in this sample set were also recorded speaking. Annotating their natural speech would provide a valuable contribution to the song corpus, as findings from the songs could be compared with speech of the same person.  

Extending the findings of vowel duration and the ternary quantity contrast has several obvious paths: synchronic analysis of with song samples from the same approximate time period but differing according to region, or even language: several other Balto-Finnic families have a trochaic tetrameter folksong tradition. Diachronic analysis with song samples of same singers in the same region at different points in time is another possibility with data from the Estonian Folklore Archives. Both these goals are achievable only with the continued annotation of the corpus of regilaul, which is quite demanding work. As I continue to build this corpus, I am also actively exploring ways in which to automate the process. The inclusion of beat tracking software eliminated much observer subjectivity, and also facilitated the forced-aligner: by automatically grouping verse lines into measures, the aligner was given phrase groupings to synthesize and compare, rather than attempting to align the entire song in a linear fashion. However, as the forced aligner used here was made specifically for speech, one way to improve the accuracy of the forced aligner (decreasing manual adjustment of annotations) would be to train a the aligner on sung material using supervised machine learning. As I plan to continue with the annotations either way, I can use the corrections I make as training material for the algorithm, with the hopeful result that the forced-aligner will eventually reach some threshold of accuracy, voiding the need for manual adjustments. 


\section{Conclusion}

This study examined fine-grained acoustic-phonetic features of Estonian prosody in the context of traditional folksongs known as regilaul. The data support the hypothesis that duration contrasts inherent in the ternary quantities of Estonian are still present at the segmental level, even after undergoing modification to fit syllables into isochronous notes of the song. The data also supports that the durational correlates of word-level stress are also crystallized and evident at the segmental level, so while it isn't lexically contrastive, the role of primary stress and unstress to mark word boundaries in spoken Estonian is still present in sung Estonian. Vowel quality patterns were measured but not significant for this dataset. 

This indicates a relationship more akin to the collaboration of two independent rhythmic systems rather than one rhythmic system dominating the other. Combined with the fact that any regilaul text can be sung to any existing regilaul melody brings into question whether {\it spoken} metrical verse text is truly independent of the temporal constraints of the musical meter they are made with, or somewhere between language and music. 

%
%\chapter{Conclusion}
\index{Conclusion@\emph{Conclusion}}%

%Vowel Duration: \\
%We failed to reject the null hypothesis for vowel duration: while the song level does dominate the temporal domain, there is an inverse relationship to vowel duration and word-level duration contrasts for Q1, Q3, and stressed syllables. So while the song always wins, there is clearly something else going on. It is possible that it is as simple as note-syllable isochrony, and the vowel decrease is simply an accommodation to give the ictus level the appropriate duration syllable. 
%
%Vowel Space: \\
%
%After confirming that song position is the dominant hierarchy for temporal constraints, we continue on to see how vowel space plays out in different levels of prosodic prominence. We are able to reject the null hypothesis with data to support a clear relationship between prominence and vowel space, this time at both prosodic levels (song and word.) This suggests that for the song, ``preserve contrast" ranks lower than the temporal constraints, but that so long as the strong positions in the song are satisfied, prominence can be indicated by vowel space at multiple levels of the prosodic hierarchy.  



%\include{chapter-makingbib}
%
%\include{chapter-tables+figs}
%
%\include{chapter-math}


%%%%%%%%%%%%%%%%%%%%%%%%%%%%%%%%%%%%%%%%%%%%%%%%%%%%%%%%%%%%%%%%%%%%%%
% Appendix/Appendices                                                %
%%%%%%%%%%%%%%%%%%%%%%%%%%%%%%%%%%%%%%%%%%%%%%%%%%%%%%%%%%%%%%%%%%%%%%
%
% If you have only one appendix, use the command \appendix instead
% of \appendices.
%%
%\appendices
%\index{Appendices@\emph{Appendices}}%
%%
%
\chapter{songs} 



\begin{figure}[htbp]
\begin{center}
\includegraphics[width=300pt]{figures/094.png}
\caption{Kuningamäng ``The King Game" performed by Liisa Kümmel}
\label{song 94}
\end{center}
\end{figure}


\begin{figure}[htbp]
\begin{center}
\includegraphics[width=300pt]{figures/009.png}
\caption{Põld lindudele performed by Marie Helimets}
\label{song 9 }
\end{center}
\end{figure}


\begin{figure}[htbp]
\begin{center}
\includegraphics[width=300pt]{figures/018.png}
\caption{Kadrilaul performed by Marie Helimets}
\label{song 18}
\end{center}
\end{figure}


\begin{figure}[htbp]
\begin{center}
\includegraphics[width=300pt]{figures/094.png}
\caption{default}
\label{94}
\end{center}
\end{figure}


\begin{figure}[htbp]
\begin{center}
\includegraphics[width=300pt]{figures/081.png}
\caption{Mehetapja (Maielaul) Greete Jents}
\label{81}
\end{center}
\end{figure}


\begin{figure}[htbp]
\begin{center}
\includegraphics[width=300pt]{figures/077.png}
\caption{{\it Loomine} performed by Liisu Orik }
\label{77}
\end{center}
\end{figure}

\begin{figure}[htbp]
\begin{center}
\includegraphics[width=300pt]{figures/069.png}
\caption{default}
\label{default}
\end{center}
\end{figure}

\begin{figure}[htbp]
\begin{center}
\includegraphics[width=300pt]{figures/045.png}
\caption{default}
\label{default}
\end{center}
\end{figure}






\begin{figure}[htbp]
\begin{center}
\includegraphics[width=300pt]{figures/055.png}
\caption{default}
\label{default}
\end{center}
\end{figure}



\begin{figure}[htbp]
\begin{center}
\includegraphics[width=300pt]{figures/045.png}
\caption{default}
\label{default}
\end{center}
\end{figure}
%\chapter{Estonian Folklore Archives}
\begin{figure}[htb]
\begin{center}
\includegraphics[width=300pt]{figures/vinyyl.png}
\caption{1970 cover art of vinyl anthology}
\label{vinyl1970}
\end{center}
\end{figure}

\begin{figure}[htp]
\centering
\includegraphics[width=0.9\linewidth]{figures/kihelkonnad}
\end{figure} 

%%song only
% latex table generated in R 4.1.2 by xtable 1.8-4 package
% Fri Aug 12 21:55:13 2022
\begin{longtable}{lrrrrrrrr}
  \toprule
Parameter & Rhat & n\_eff & mean & sd & se\_mean & 2.5\% & 50\% & 97.5\% \\ 
  \midrule
Intercept & 1.001 &  801 & 0.193 & 0.024 & 0.001 & 0.143 & 0.194 & 0.242 \\ 
  ictusoff & 1.000 & 4507 & -0.016 & 0.006 & 0.000 & -0.028 & -0.016 & -0.003 \\ 
  stressed1 & 1.001 & 4042 & 0.047 & 0.007 & 0.000 & 0.034 & 0.047 & 0.060 \\ 
  quantity2 & 1.002 & 3755 & -0.025 & 0.007 & 0.000 & -0.039 & -0.025 & -0.012 \\ 
  quantity3 & 1.002 & 3613 & -0.037 & 0.011 & 0.000 & -0.058 & -0.037 & -0.016 \\ 
  b[(Intercept) song:9] & 1.001 &  889 & 0.094 & 0.025 & 0.001 & 0.044 & 0.093 & 0.145 \\ 
  b[(Intercept) song:18] & 1.001 &  799 & -0.006 & 0.024 & 0.001 & -0.054 & -0.006 & 0.043 \\ 
  b[(Intercept) song:41] & 1.001 &  830 & 0.022 & 0.025 & 0.001 & -0.028 & 0.022 & 0.073 \\ 
  b[(Intercept) song:55] & 1.001 &  846 & -0.037 & 0.026 & 0.001 & -0.088 & -0.036 & 0.015 \\ 
  b[(Intercept) song:65] & 1.001 &  860 & 0.075 & 0.025 & 0.001 & 0.025 & 0.075 & 0.125 \\ 
  b[(Intercept) song:69] & 1.000 & 1211 & -0.042 & 0.031 & 0.001 & -0.105 & -0.042 & 0.016 \\ 
  b[(Intercept) song:77] & 1.001 &  993 & -0.105 & 0.027 & 0.001 & -0.159 & -0.105 & -0.051 \\ 
  b[(Intercept) song:92] & 1.001 &  989 & 0.033 & 0.027 & 0.001 & -0.020 & 0.034 & 0.087 \\ 
  b[(Intercept) song:94] & 1.002 &  878 & -0.039 & 0.025 & 0.001 & -0.089 & -0.039 & 0.011 \\ 
   \bottomrule
\caption{posterier parameters} 
\end{longtable}


%performer & word

% latex table generated in R 4.1.2 by xtable 1.8-4 package
% Fri Aug 12 19:59:47 2022
\begin{longtable}{lrrrrrrr}
  \toprule
Parameter & Rhat & n\_eff & mean & sd & 2.5\% & 50\% & 97.5\% \\ 
  \midrule
Intercept & 1.000 & 1871 & 0.197 & 0.036 & 0.122 & 0.197 & 0.270 \\ 
  ictusoff & 1.000 & 4383 & -0.022 & 0.007 & -0.035 & -0.022 & -0.008 \\ 
  stressed1 & 0.999 & 5726 & 0.050 & 0.006 & 0.038 & 0.050 & 0.062 \\ 
  quantity2 & 0.999 & 4036 & -0.029 & 0.007 & -0.043 & -0.029 & -0.015 \\ 
  quantity3 & 0.999 & 3106 & -0.036 & 0.013 & -0.062 & -0.036 & -0.010 \\ 
  b[(Intercept) word:ää] & 1.000 & 3916 & 0.143 & 0.043 & 0.058 & 0.142 & 0.228 \\ 
  b[(Intercept) word:aga] & 1.000 & 6184 & -0.065 & 0.034 & -0.133 & -0.065 & 0.001 \\ 
  b[(Intercept) word:ahju] & 1.000 & 8703 & -0.020 & 0.034 & -0.088 & -0.021 & 0.047 \\ 
  b[(Intercept) word:aia] & 1.000 & 7218 & 0.034 & 0.040 & -0.043 & 0.034 & 0.112 \\ 
  b[(Intercept) word:ajal,] & 1.000 & 7275 & 0.009 & 0.034 & -0.058 & 0.008 & 0.078 \\ 
  b[(Intercept) word:akkas] & 0.999 & 6375 & -0.032 & 0.034 & -0.101 & -0.032 & 0.035 \\ 
  b[(Intercept) word:allitagu,] & 0.999 & 7450 & 0.000 & 0.034 & -0.065 & 0.001 & 0.065 \\ 
  b[(Intercept) word:anisida,] & 1.000 & 7942 & 0.043 & 0.035 & -0.023 & 0.043 & 0.111 \\ 
  b[(Intercept) word:anna] & 0.999 & 7653 & 0.010 & 0.034 & -0.055 & 0.010 & 0.078 \\ 
  b[(Intercept) word:Anna,] & 1.000 & 7784 & -0.008 & 0.033 & -0.071 & -0.009 & 0.055 \\ 
  b[(Intercept) word:ärmätus] & 1.000 & 8103 & -0.021 & 0.034 & -0.086 & -0.020 & 0.043 \\ 
  b[(Intercept) word:Ega] & 1.000 & 6486 & -0.054 & 0.034 & -0.123 & -0.054 & 0.013 \\ 
  b[(Intercept) word:ei] & 1.000 & 6144 & -0.082 & 0.036 & -0.151 & -0.081 & -0.012 \\ 
  b[(Intercept) word:Ei] & 1.000 & 4695 & 0.120 & 0.043 & 0.037 & 0.120 & 0.203 \\ 
  b[(Intercept) word:esta!] & 0.999 & 7306 & -0.002 & 0.035 & -0.071 & -0.002 & 0.066 \\ 
  b[(Intercept) word:hääre] & 1.000 & 8670 & 0.017 & 0.034 & -0.051 & 0.018 & 0.084 \\ 
  b[(Intercept) word:halgu] & 1.000 & 8538 & -0.020 & 0.040 & -0.100 & -0.019 & 0.057 \\ 
  b[(Intercept) word:ilma,] & 0.999 & 8640 & 0.026 & 0.040 & -0.050 & 0.026 & 0.105 \\ 
  b[(Intercept) word:ilutse,] & 0.999 & 7301 & 0.021 & 0.034 & -0.047 & 0.021 & 0.088 \\ 
  b[(Intercept) word:ja] & 0.999 & 5985 & -0.067 & 0.031 & -0.129 & -0.066 & -0.008 \\ 
  b[(Intercept) word:jäi] & 0.999 & 7571 & -0.035 & 0.041 & -0.119 & -0.036 & 0.045 \\ 
  b[(Intercept) word:jalas] & 1.000 & 7026 & 0.029 & 0.034 & -0.036 & 0.029 & 0.095 \\ 
  b[(Intercept) word:jätan] & 1.000 & 7693 & 0.006 & 0.034 & -0.060 & 0.005 & 0.072 \\ 
  b[(Intercept) word:jätän] & 0.999 & 8537 & -0.005 & 0.035 & -0.074 & -0.005 & 0.063 \\ 
  b[(Intercept) word:Jätän] & 1.000 & 6625 & -0.011 & 0.034 & -0.078 & -0.011 & 0.058 \\ 
  b[(Intercept) word:jätan,] & 0.999 & 7348 & 0.050 & 0.034 & -0.015 & 0.051 & 0.116 \\ 
  b[(Intercept) word:juhate:] & 1.000 & 8309 & -0.022 & 0.034 & -0.086 & -0.022 & 0.043 \\ 
  b[(Intercept) word:juksimaie,] & 1.000 & 7291 & 0.023 & 0.033 & -0.041 & 0.023 & 0.088 \\ 
  b[(Intercept) word:julgust] & 1.000 & 6704 & -0.006 & 0.039 & -0.084 & -0.006 & 0.070 \\ 
  b[(Intercept) word:juure] & 0.999 & 6174 & 0.069 & 0.035 & -0.001 & 0.069 & 0.137 \\ 
  b[(Intercept) word:juuri] & 1.000 & 7542 & 0.031 & 0.032 & -0.032 & 0.031 & 0.094 \\ 
  b[(Intercept) word:Kadri] & 1.000 & 6946 & -0.070 & 0.035 & -0.138 & -0.070 & -0.001 \\ 
  b[(Intercept) word:kadril] & 1.000 & 7841 & 0.010 & 0.033 & -0.055 & 0.009 & 0.075 \\ 
  b[(Intercept) word:kadrisandi,] & 0.999 & 8674 & 0.004 & 0.027 & -0.049 & 0.004 & 0.058 \\ 
  b[(Intercept) word:kaelakardasida,] & 1.000 & 7210 & 0.003 & 0.033 & -0.064 & 0.003 & 0.068 \\ 
  b[(Intercept) word:käinud,] & 1.001 & 7140 & -0.042 & 0.034 & -0.108 & -0.042 & 0.025 \\ 
  b[(Intercept) word:käisid] & 0.999 & 7889 & -0.044 & 0.034 & -0.112 & -0.044 & 0.024 \\ 
  b[(Intercept) word:kakud] & 1.000 & 7280 & -0.058 & 0.034 & -0.124 & -0.057 & 0.008 \\ 
  b[(Intercept) word:kambren] & 1.000 & 7709 & -0.028 & 0.034 & -0.095 & -0.029 & 0.041 \\ 
  b[(Intercept) word:kana] & 1.000 & 8581 & 0.003 & 0.034 & -0.063 & 0.003 & 0.070 \\ 
  b[(Intercept) word:kanaliha\_–] & 1.000 & 7102 & 0.010 & 0.034 & -0.055 & 0.010 & 0.075 \\ 
  b[(Intercept) word:kandijaksa,] & 1.000 & 7895 & 0.004 & 0.034 & -0.063 & 0.003 & 0.071 \\ 
  b[(Intercept) word:kannustenu.] & 0.999 & 7820 & -0.021 & 0.034 & -0.087 & -0.021 & 0.045 \\ 
  b[(Intercept) word:kapi] & 1.000 & 7349 & -0.004 & 0.034 & -0.072 & -0.003 & 0.062 \\ 
  b[(Intercept) word:karask] & 1.000 & 5578 & -0.055 & 0.033 & -0.120 & -0.056 & 0.009 \\ 
  b[(Intercept) word:käsud] & 1.000 & 7050 & -0.022 & 0.034 & -0.090 & -0.021 & 0.044 \\ 
  b[(Intercept) word:kauge] & 0.999 & 9251 & 0.018 & 0.035 & -0.049 & 0.018 & 0.088 \\ 
  b[(Intercept) word:keelekene,] & 1.000 & 7304 & 0.035 & 0.035 & -0.032 & 0.035 & 0.101 \\ 
  b[(Intercept) word:kehval] & 1.000 & 6675 & -0.042 & 0.034 & -0.109 & -0.042 & 0.025 \\ 
  b[(Intercept) word:kevadisel] & 0.999 & 7105 & -0.015 & 0.034 & -0.080 & -0.015 & 0.051 \\ 
  b[(Intercept) word:kinni] & 1.000 & 7442 & -0.056 & 0.028 & -0.109 & -0.056 & -0.001 \\ 
  b[(Intercept) word:kinnikiilutud] & 1.000 & 8761 & 0.010 & 0.034 & -0.056 & 0.011 & 0.078 \\ 
  b[(Intercept) word:kirju] & 1.000 & 7611 & -0.035 & 0.034 & -0.106 & -0.034 & 0.032 \\ 
  b[(Intercept) word:kogemata,] & 1.000 & 5693 & 0.066 & 0.034 & -0.000 & 0.066 & 0.132 \\ 
  b[(Intercept) word:kohlapuida,] & 1.000 & 7280 & 0.043 & 0.034 & -0.023 & 0.044 & 0.109 \\ 
  b[(Intercept) word:kolmas] & 1.000 & 7639 & 0.010 & 0.040 & -0.068 & 0.009 & 0.091 \\ 
  b[(Intercept) word:korjama.] & 0.999 & 6943 & 0.009 & 0.034 & -0.058 & 0.008 & 0.077 \\ 
  b[(Intercept) word:kõrtsi] & 1.000 & 6535 & -0.033 & 0.033 & -0.099 & -0.033 & 0.030 \\ 
  b[(Intercept) word:kübara] & 0.999 & 8841 & 0.023 & 0.034 & -0.042 & 0.023 & 0.091 \\ 
  b[(Intercept) word:kübarast] & 0.999 & 7968 & 0.021 & 0.033 & -0.044 & 0.021 & 0.086 \\ 
  b[(Intercept) word:kui] & 0.999 & 6816 & 0.022 & 0.031 & -0.038 & 0.021 & 0.081 \\ 
  b[(Intercept) word:Kui] & 1.000 & 8489 & 0.028 & 0.040 & -0.052 & 0.028 & 0.109 \\ 
  b[(Intercept) word:kul´li] & 0.999 & 8654 & -0.016 & 0.035 & -0.083 & -0.016 & 0.050 \\ 
  b[(Intercept) word:Küll] & 0.999 & 5194 & -0.067 & 0.030 & -0.126 & -0.067 & -0.007 \\ 
  b[(Intercept) word:külmeteve,] & 1.000 & 8682 & -0.014 & 0.041 & -0.097 & -0.015 & 0.067 \\ 
  b[(Intercept) word:kuningas,] & 0.999 & 7216 & -0.042 & 0.034 & -0.108 & -0.041 & 0.025 \\ 
  b[(Intercept) word:kuningas!] & 1.000 & 7651 & -0.026 & 0.034 & -0.092 & -0.026 & 0.040 \\ 
  b[(Intercept) word:künnirauast] & 0.999 & 8192 & 0.009 & 0.033 & -0.056 & 0.009 & 0.075 \\ 
  b[(Intercept) word:künnirauda,] & 1.000 & 7732 & 0.016 & 0.034 & -0.050 & 0.017 & 0.085 \\ 
  b[(Intercept) word:kus] & 0.999 & 7103 & 0.019 & 0.040 & -0.060 & 0.019 & 0.097 \\ 
  b[(Intercept) word:küüdse] & 0.999 & 7862 & 0.012 & 0.034 & -0.054 & 0.013 & 0.080 \\ 
  b[(Intercept) word:laane] & 1.000 & 8693 & 0.018 & 0.034 & -0.050 & 0.018 & 0.084 \\ 
  b[(Intercept) word:läbi] & 0.999 & 7518 & -0.016 & 0.027 & -0.069 & -0.015 & 0.039 \\ 
  b[(Intercept) word:lapse] & 0.999 & 8175 & -0.019 & 0.034 & -0.083 & -0.019 & 0.050 \\ 
  b[(Intercept) word:lasandi.] & 0.999 & 8789 & -0.017 & 0.034 & -0.085 & -0.017 & 0.049 \\ 
  b[(Intercept) word:Laske] & 1.000 & 7286 & -0.019 & 0.032 & -0.082 & -0.018 & 0.046 \\ 
  b[(Intercept) word:lauade] & 0.999 & 6324 & -0.004 & 0.032 & -0.067 & -0.005 & 0.056 \\ 
  b[(Intercept) word:laudjas] & 1.000 & 6368 & -0.026 & 0.034 & -0.095 & -0.026 & 0.042 \\ 
  b[(Intercept) word:laula,] & 0.999 & 7582 & -0.016 & 0.031 & -0.077 & -0.016 & 0.044 \\ 
  b[(Intercept) word:Laula,] & 0.999 & 8017 & -0.015 & 0.031 & -0.076 & -0.015 & 0.046 \\ 
  b[(Intercept) word:Leidsine] & 1.000 & 7520 & -0.014 & 0.033 & -0.080 & -0.014 & 0.050 \\ 
  b[(Intercept) word:ligi] & 0.999 & 7267 & 0.055 & 0.035 & -0.014 & 0.055 & 0.126 \\ 
  b[(Intercept) word:ligunema.] & 0.999 & 6525 & 0.047 & 0.033 & -0.018 & 0.047 & 0.112 \\ 
  b[(Intercept) word:liigu,] & 1.000 & 9371 & 0.024 & 0.034 & -0.042 & 0.024 & 0.092 \\ 
  b[(Intercept) word:linakene,] & 1.000 & 7885 & 0.039 & 0.034 & -0.025 & 0.038 & 0.105 \\ 
  b[(Intercept) word:linnu] & 1.000 & 7560 & -0.016 & 0.034 & -0.085 & -0.016 & 0.052 \\ 
  b[(Intercept) word:linnukene,] & 0.999 & 8010 & 0.018 & 0.027 & -0.034 & 0.018 & 0.071 \\ 
  b[(Intercept) word:lipa-lapa.] & 1.000 & 7025 & -0.025 & 0.034 & -0.093 & -0.025 & 0.042 \\ 
  b[(Intercept) word:lõhna] & 1.000 & 7544 & -0.015 & 0.027 & -0.067 & -0.015 & 0.039 \\ 
  b[(Intercept) word:lõke] & 1.000 & 7632 & -0.015 & 0.035 & -0.082 & -0.015 & 0.052 \\ 
  b[(Intercept) word:lõmmu] & 0.999 & 8751 & -0.019 & 0.027 & -0.073 & -0.018 & 0.037 \\ 
  b[(Intercept) word:loogakirja,] & 0.999 & 8004 & 0.015 & 0.035 & -0.054 & 0.014 & 0.083 \\ 
  b[(Intercept) word:lõõtsu] & 1.000 & 7054 & -0.003 & 0.034 & -0.069 & -0.003 & 0.062 \\ 
  b[(Intercept) word:Lõpe,] & 0.999 & 7897 & 0.054 & 0.034 & -0.013 & 0.054 & 0.122 \\ 
  b[(Intercept) word:lumi] & 1.000 & 7642 & -0.003 & 0.033 & -0.068 & -0.002 & 0.063 \\ 
  b[(Intercept) word:ma] & 1.000 & 7741 & -0.034 & 0.028 & -0.091 & -0.034 & 0.020 \\ 
  b[(Intercept) word:maada] & 0.999 & 7348 & 0.042 & 0.033 & -0.020 & 0.042 & 0.108 \\ 
  b[(Intercept) word:mal´lika,] & 1.000 & 7942 & -0.018 & 0.041 & -0.101 & -0.018 & 0.063 \\ 
  b[(Intercept) word:mal´likast] & 0.999 & 7726 & -0.021 & 0.041 & -0.101 & -0.022 & 0.058 \\ 
  b[(Intercept) word:man´nipil´li,] & 0.999 & 8491 & -0.009 & 0.035 & -0.076 & -0.009 & 0.062 \\ 
  b[(Intercept) word:mära] & 1.000 & 6982 & -0.054 & 0.041 & -0.134 & -0.054 & 0.026 \\ 
  b[(Intercept) word:marja] & 0.999 & 8294 & 0.000 & 0.034 & -0.066 & 0.001 & 0.068 \\ 
  b[(Intercept) word:mätaste] & 1.000 & 7457 & -0.048 & 0.036 & -0.118 & -0.048 & 0.023 \\ 
  b[(Intercept) word:me] & 1.001 & 7056 & -0.016 & 0.041 & -0.095 & -0.016 & 0.065 \\ 
  b[(Intercept) word:meelekene,] & 1.000 & 6798 & 0.039 & 0.035 & -0.027 & 0.039 & 0.108 \\ 
  b[(Intercept) word:meile(e)i] & 1.000 & 7809 & -0.024 & 0.034 & -0.090 & -0.023 & 0.042 \\ 
  b[(Intercept) word:miks] & 1.000 & 5890 & -0.056 & 0.041 & -0.137 & -0.055 & 0.022 \\ 
  b[(Intercept) word:mind(e)] & 1.000 & 6749 & -0.085 & 0.035 & -0.152 & -0.085 & -0.018 \\ 
  b[(Intercept) word:minna,] & 0.999 & 9109 & -0.027 & 0.033 & -0.091 & -0.027 & 0.039 \\ 
  b[(Intercept) word:mis] & 1.000 & 8126 & -0.033 & 0.041 & -0.112 & -0.032 & 0.045 \\ 
  b[(Intercept) word:mõisa] & 1.000 & 8376 & 0.026 & 0.027 & -0.027 & 0.027 & 0.079 \\ 
  b[(Intercept) word:mõlgu,] & 0.999 & 7236 & -0.023 & 0.035 & -0.091 & -0.023 & 0.044 \\ 
  b[(Intercept) word:mõõduvakk.] & 0.999 & 7630 & 0.035 & 0.033 & -0.031 & 0.035 & 0.100 \\ 
  b[(Intercept) word:mullu] & 1.000 & 7892 & -0.020 & 0.041 & -0.097 & -0.020 & 0.061 \\ 
  b[(Intercept) word:murumullu] & 1.000 & 7215 & -0.049 & 0.034 & -0.116 & -0.049 & 0.019 \\ 
  b[(Intercept) word:näl´lasandi,] & 0.999 & 8850 & -0.007 & 0.041 & -0.084 & -0.007 & 0.073 \\ 
  b[(Intercept) word:neede] & 0.999 & 6197 & -0.028 & 0.029 & -0.085 & -0.029 & 0.027 \\ 
  b[(Intercept) word:neidusida,] & 1.000 & 7910 & 0.044 & 0.040 & -0.035 & 0.043 & 0.121 \\ 
  b[(Intercept) word:neljas] & 0.999 & 8450 & 0.028 & 0.035 & -0.039 & 0.027 & 0.097 \\ 
  b[(Intercept) word:Nii] & 0.999 & 8927 & 0.021 & 0.040 & -0.060 & 0.021 & 0.098 \\ 
  b[(Intercept) word:noori] & 0.999 & 7464 & 0.049 & 0.034 & -0.018 & 0.048 & 0.114 \\ 
  b[(Intercept) word:nüüd] & 0.999 & 7869 & -0.010 & 0.040 & -0.091 & -0.009 & 0.066 \\ 
  b[(Intercept) word:Nüüd] & 0.999 & 7708 & -0.012 & 0.040 & -0.091 & -0.013 & 0.067 \\ 
  b[(Intercept) word:obese.] & 1.000 & 7186 & -0.048 & 0.033 & -0.115 & -0.048 & 0.016 \\ 
  b[(Intercept) word:Obu] & 0.999 & 7521 & -0.055 & 0.034 & -0.123 & -0.055 & 0.010 \\ 
  b[(Intercept) word:õel] & 1.000 & 6463 & 0.045 & 0.034 & -0.023 & 0.044 & 0.113 \\ 
  b[(Intercept) word:ojasse,] & 0.999 & 8936 & -0.003 & 0.034 & -0.070 & -0.003 & 0.061 \\ 
  b[(Intercept) word:ole] & 0.999 & 6739 & -0.070 & 0.025 & -0.120 & -0.070 & -0.021 \\ 
  b[(Intercept) word:olem] & 1.001 & 5195 & -0.096 & 0.035 & -0.165 & -0.095 & -0.026 \\ 
  b[(Intercept) word:oli] & 1.000 & 5262 & -0.096 & 0.019 & -0.135 & -0.096 & -0.059 \\ 
  b[(Intercept) word:olid] & 1.000 & 5432 & -0.084 & 0.025 & -0.133 & -0.083 & -0.034 \\ 
  b[(Intercept) word:om] & 1.000 & 5471 & -0.081 & 0.026 & -0.131 & -0.081 & -0.030 \\ 
  b[(Intercept) word:Ööd] & 1.001 & 5257 & 0.109 & 0.041 & 0.028 & 0.108 & 0.191 \\ 
  b[(Intercept) word:oodakuta!] & 1.000 & 6925 & 0.052 & 0.035 & -0.016 & 0.052 & 0.120 \\ 
  b[(Intercept) word:ööles] & 1.000 & 5605 & 0.089 & 0.035 & 0.021 & 0.089 & 0.157 \\ 
  b[(Intercept) word:öösed] & 0.999 & 8055 & 0.032 & 0.034 & -0.035 & 0.032 & 0.099 \\ 
  b[(Intercept) word:orjad,] & 0.999 & 6709 & -0.010 & 0.035 & -0.078 & -0.011 & 0.056 \\ 
  b[(Intercept) word:orjal] & 0.999 & 7986 & 0.043 & 0.040 & -0.035 & 0.042 & 0.121 \\ 
  b[(Intercept) word:otsa,] & 1.000 & 7429 & 0.011 & 0.033 & -0.055 & 0.012 & 0.076 \\ 
  b[(Intercept) word:õvve] & 1.000 & 6891 & -0.009 & 0.026 & -0.058 & -0.009 & 0.042 \\ 
  b[(Intercept) word:pääl.] & 0.999 & 8652 & 0.009 & 0.040 & -0.069 & 0.009 & 0.087 \\ 
  b[(Intercept) word:pääle] & 1.000 & 6486 & 0.073 & 0.035 & 0.005 & 0.073 & 0.140 \\ 
  b[(Intercept) word:päälta,] & 1.000 & 7009 & 0.008 & 0.033 & -0.055 & 0.007 & 0.071 \\ 
  b[(Intercept) word:päälta!] & 1.000 & 7161 & 0.010 & 0.040 & -0.067 & 0.010 & 0.090 \\ 
  b[(Intercept) word:paari] & 1.001 & 7213 & -0.006 & 0.034 & -0.072 & -0.006 & 0.060 \\ 
  b[(Intercept) word:pääsukesel] & 1.000 & 6909 & 0.071 & 0.034 & 0.004 & 0.070 & 0.137 \\ 
  b[(Intercept) word:päivä] & 1.000 & 6670 & 0.048 & 0.034 & -0.019 & 0.048 & 0.114 \\ 
  b[(Intercept) word:paju] & 1.000 & 6496 & 0.058 & 0.034 & -0.007 & 0.058 & 0.124 \\ 
  b[(Intercept) word:palada,] & 1.000 & 7578 & 0.030 & 0.034 & -0.036 & 0.030 & 0.099 \\ 
  b[(Intercept) word:pandi] & 0.999 & 7285 & -0.036 & 0.035 & -0.106 & -0.036 & 0.031 \\ 
  b[(Intercept) word:pani] & 1.000 & 6068 & -0.053 & 0.035 & -0.122 & -0.054 & 0.013 \\ 
  b[(Intercept) word:partisida,] & 1.000 & 7431 & -0.031 & 0.035 & -0.100 & -0.032 & 0.038 \\ 
  b[(Intercept) word:peks(a)\_–] & 1.001 & 6348 & -0.041 & 0.034 & -0.106 & -0.040 & 0.026 \\ 
  b[(Intercept) word:penki!] & 1.000 & 7572 & -0.009 & 0.035 & -0.077 & -0.009 & 0.058 \\ 
  b[(Intercept) word:Perenaine,] & 1.000 & 6371 & 0.071 & 0.028 & 0.016 & 0.071 & 0.126 \\ 
  b[(Intercept) word:petsariinist] & 1.000 & 7436 & -0.003 & 0.034 & -0.070 & -0.003 & 0.063 \\ 
  b[(Intercept) word:piakene?] & 1.000 & 7625 & 0.020 & 0.034 & -0.045 & 0.020 & 0.085 \\ 
  b[(Intercept) word:pihta] & 0.999 & 7090 & -0.036 & 0.033 & -0.101 & -0.036 & 0.027 \\ 
  b[(Intercept) word:piigakesi.] & 1.000 & 6129 & 0.080 & 0.035 & 0.015 & 0.080 & 0.150 \\ 
  b[(Intercept) word:pirdu] & 1.000 & 7710 & -0.033 & 0.027 & -0.085 & -0.032 & 0.019 \\ 
  b[(Intercept) word:pistussenna,] & 0.999 & 7848 & -0.013 & 0.033 & -0.079 & -0.013 & 0.050 \\ 
  b[(Intercept) word:pistussesta,] & 0.999 & 8949 & -0.018 & 0.034 & -0.085 & -0.018 & 0.048 \\ 
  b[(Intercept) word:põrmandulle,] & 1.000 & 7597 & -0.023 & 0.034 & -0.092 & -0.023 & 0.044 \\ 
  b[(Intercept) word:pühkijaksa,] & 1.000 & 8211 & -0.028 & 0.026 & -0.079 & -0.027 & 0.025 \\ 
  b[(Intercept) word:puhu] & 0.999 & 7510 & 0.019 & 0.034 & -0.048 & 0.019 & 0.084 \\ 
  b[(Intercept) word:puksu] & 1.000 & 7147 & -0.009 & 0.034 & -0.079 & -0.009 & 0.057 \\ 
  b[(Intercept) word:punapõs\_ki] & 1.000 & 5494 & 0.083 & 0.034 & 0.017 & 0.083 & 0.149 \\ 
  b[(Intercept) word:räästa] & 0.999 & 8138 & -0.023 & 0.034 & -0.090 & -0.022 & 0.044 \\ 
  b[(Intercept) word:rabadaie.] & 1.000 & 6674 & 0.059 & 0.034 & -0.008 & 0.059 & 0.127 \\ 
  b[(Intercept) word:raiumaie,] & 0.999 & 7297 & 0.056 & 0.034 & -0.010 & 0.057 & 0.122 \\ 
  b[(Intercept) word:ranitsa] & 0.999 & 8436 & -0.000 & 0.033 & -0.066 & -0.001 & 0.066 \\ 
  b[(Intercept) word:ranitsast] & 1.000 & 6332 & -0.003 & 0.034 & -0.067 & -0.002 & 0.063 \\ 
  b[(Intercept) word:rapsuvitsa] & 1.000 & 7175 & -0.015 & 0.033 & -0.079 & -0.015 & 0.049 \\ 
  b[(Intercept) word:riide'eida,] & 0.999 & 8611 & -0.011 & 0.040 & -0.090 & -0.011 & 0.070 \\ 
  b[(Intercept) word:riimuka,] & 0.999 & 7603 & 0.011 & 0.033 & -0.054 & 0.012 & 0.074 \\ 
  b[(Intercept) word:riimukast] & 0.999 & 7402 & -0.019 & 0.033 & -0.084 & -0.019 & 0.048 \\ 
  b[(Intercept) word:riisud] & 0.999 & 7643 & -0.020 & 0.035 & -0.090 & -0.020 & 0.049 \\ 
  b[(Intercept) word:rikkus] & 0.999 & 7122 & -0.029 & 0.027 & -0.082 & -0.029 & 0.022 \\ 
  b[(Intercept) word:rindasõl´ge,] & 0.999 & 8130 & -0.019 & 0.034 & -0.086 & -0.019 & 0.048 \\ 
  b[(Intercept) word:rindasõlesta] & 0.999 & 8144 & -0.009 & 0.034 & -0.077 & -0.010 & 0.060 \\ 
  b[(Intercept) word:roogu] & 1.000 & 8113 & 0.014 & 0.034 & -0.052 & 0.013 & 0.082 \\ 
  b[(Intercept) word:sa] & 1.000 & 5825 & -0.071 & 0.035 & -0.141 & -0.071 & -0.003 \\ 
  b[(Intercept) word:saagu,] & 1.000 & 7776 & -0.029 & 0.027 & -0.083 & -0.030 & 0.024 \\ 
  b[(Intercept) word:Saagu,] & 0.999 & 8214 & 0.007 & 0.027 & -0.047 & 0.007 & 0.060 \\ 
  b[(Intercept) word:saaniteki,] & 0.999 & 7846 & 0.026 & 0.035 & -0.042 & 0.026 & 0.094 \\ 
  b[(Intercept) word:saare] & 1.000 & 8585 & 0.056 & 0.034 & -0.010 & 0.056 & 0.124 \\ 
  b[(Intercept) word:saarekirja.] & 0.999 & 6118 & 0.084 & 0.036 & 0.016 & 0.083 & 0.154 \\ 
  b[(Intercept) word:sadu] & 0.999 & 8344 & 0.026 & 0.034 & -0.039 & 0.027 & 0.093 \\ 
  b[(Intercept) word:sain] & 0.999 & 5427 & -0.040 & 0.023 & -0.085 & -0.040 & 0.004 \\ 
  b[(Intercept) word:sajad] & 1.000 & 7673 & -0.042 & 0.034 & -0.108 & -0.041 & 0.023 \\ 
  b[(Intercept) word:sajatan:] & 1.001 & 5854 & -0.027 & 0.027 & -0.082 & -0.028 & 0.026 \\ 
  b[(Intercept) word:san´dil] & 1.000 & 7068 & -0.028 & 0.040 & -0.106 & -0.028 & 0.050 \\ 
  b[(Intercept) word:sarapuida,] & 1.000 & 7437 & 0.057 & 0.033 & -0.007 & 0.057 & 0.122 \\ 
  b[(Intercept) word:see] & 1.000 & 6842 & -0.007 & 0.040 & -0.086 & -0.007 & 0.072 \\ 
  b[(Intercept) word:seenetagu,] & 0.999 & 8658 & 0.010 & 0.034 & -0.055 & 0.010 & 0.076 \\ 
  b[(Intercept) word:sei] & 1.000 & 7735 & 0.035 & 0.042 & -0.045 & 0.035 & 0.116 \\ 
  b[(Intercept) word:siia] & 0.999 & 7259 & 0.003 & 0.026 & -0.049 & 0.003 & 0.054 \\ 
  b[(Intercept) word:siin] & 1.000 & 7308 & 0.034 & 0.034 & -0.033 & 0.033 & 0.100 \\ 
  b[(Intercept) word:Siin] & 1.001 & 6517 & 0.026 & 0.040 & -0.052 & 0.025 & 0.103 \\ 
  b[(Intercept) word:siis] & 0.999 & 7207 & -0.026 & 0.039 & -0.104 & -0.025 & 0.053 \\ 
  b[(Intercept) word:sina] & 1.000 & 7707 & -0.062 & 0.027 & -0.114 & -0.062 & -0.010 \\ 
  b[(Intercept) word:sinu] & 1.000 & 5954 & -0.087 & 0.034 & -0.153 & -0.087 & -0.021 \\ 
  b[(Intercept) word:sipa-sopa,] & 0.999 & 7016 & -0.021 & 0.035 & -0.089 & -0.020 & 0.047 \\ 
  b[(Intercept) word:sissi] & 0.999 & 7492 & -0.015 & 0.027 & -0.067 & -0.016 & 0.036 \\ 
  b[(Intercept) word:sõitsid] & 1.000 & 6096 & -0.039 & 0.040 & -0.119 & -0.039 & 0.038 \\ 
  b[(Intercept) word:sõmera,] & 0.999 & 7193 & 0.049 & 0.035 & -0.018 & 0.049 & 0.116 \\ 
  b[(Intercept) word:sõmerast] & 1.000 & 6709 & 0.032 & 0.032 & -0.031 & 0.032 & 0.095 \\ 
  b[(Intercept) word:sõnumida,] & 0.999 & 9087 & -0.017 & 0.035 & -0.084 & -0.017 & 0.049 \\ 
  b[(Intercept) word:soo] & 0.999 & 7846 & 0.006 & 0.040 & -0.071 & 0.006 & 0.087 \\ 
  b[(Intercept) word:soolavakast] & 0.999 & 8291 & 0.021 & 0.034 & -0.042 & 0.020 & 0.087 \\ 
  b[(Intercept) word:soolavakka,] & 0.999 & 7589 & 0.022 & 0.034 & -0.042 & 0.022 & 0.087 \\ 
  b[(Intercept) word:südämekene!] & 1.000 & 6147 & -0.002 & 0.035 & -0.072 & -0.002 & 0.068 \\ 
  b[(Intercept) word:südant] & 0.999 & 7899 & 0.027 & 0.035 & -0.042 & 0.026 & 0.094 \\ 
  b[(Intercept) word:süia,] & 1.000 & 5080 & 0.096 & 0.035 & 0.028 & 0.096 & 0.165 \\ 
  b[(Intercept) word:sukad] & 1.000 & 8042 & 0.023 & 0.034 & -0.043 & 0.023 & 0.090 \\ 
  b[(Intercept) word:sulased,] & 0.999 & 7266 & -0.003 & 0.035 & -0.072 & -0.003 & 0.064 \\ 
  b[(Intercept) word:sülle] & 1.000 & 6951 & 0.044 & 0.040 & -0.033 & 0.044 & 0.124 \\ 
  b[(Intercept) word:suukene,] & 0.999 & 6005 & 0.093 & 0.035 & 0.025 & 0.093 & 0.163 \\ 
  b[(Intercept) word:tää,] & 0.999 & 7455 & 0.021 & 0.041 & -0.059 & 0.020 & 0.100 \\ 
  b[(Intercept) word:taevudes,] & 0.999 & 6860 & 0.000 & 0.031 & -0.061 & 0.000 & 0.060 \\ 
  b[(Intercept) word:tagant] & 1.000 & 8954 & -0.008 & 0.034 & -0.075 & -0.008 & 0.058 \\ 
  b[(Intercept) word:taha] & 1.000 & 8478 & -0.022 & 0.034 & -0.088 & -0.022 & 0.044 \\ 
  b[(Intercept) word:talu] & 1.000 & 7367 & 0.044 & 0.026 & -0.007 & 0.044 & 0.095 \\ 
  b[(Intercept) word:tamme] & 1.000 & 9268 & 0.011 & 0.034 & -0.055 & 0.011 & 0.078 \\ 
  b[(Intercept) word:tampimaie.] & 0.999 & 8113 & 0.015 & 0.032 & -0.048 & 0.015 & 0.076 \\ 
  b[(Intercept) word:tasane,] & 0.999 & 8388 & 0.003 & 0.034 & -0.061 & 0.003 & 0.070 \\ 
  b[(Intercept) word:teine] & 1.000 & 6153 & 0.068 & 0.033 & 0.004 & 0.068 & 0.132 \\ 
  b[(Intercept) word:tõsta] & 0.999 & 7912 & -0.011 & 0.040 & -0.090 & -0.011 & 0.067 \\ 
  b[(Intercept) word:tõusis] & 0.999 & 6747 & -0.044 & 0.031 & -0.104 & -0.045 & 0.016 \\ 
  b[(Intercept) word:tsirgu] & 1.000 & 8148 & 0.034 & 0.033 & -0.030 & 0.034 & 0.098 \\ 
  b[(Intercept) word:tuastagi,] & 1.000 & 6688 & 0.030 & 0.033 & -0.034 & 0.030 & 0.095 \\ 
  b[(Intercept) word:tubadesse,] & 0.999 & 7485 & 0.009 & 0.034 & -0.056 & 0.008 & 0.075 \\ 
  b[(Intercept) word:tul´lid] & 1.000 & 7723 & -0.039 & 0.041 & -0.120 & -0.039 & 0.039 \\ 
  b[(Intercept) word:tule] & 0.999 & 8225 & 0.032 & 0.034 & -0.036 & 0.032 & 0.099 \\ 
  b[(Intercept) word:tuli] & 0.999 & 7453 & 0.009 & 0.034 & -0.060 & 0.009 & 0.077 \\ 
  b[(Intercept) word:tulla.] & 1.000 & 4612 & 0.082 & 0.041 & 0.001 & 0.081 & 0.163 \\ 
  b[(Intercept) word:tullu] & 1.000 & 8668 & -0.020 & 0.039 & -0.098 & -0.019 & 0.057 \\ 
  b[(Intercept) word:tulnud,] & 1.000 & 8399 & -0.020 & 0.040 & -0.098 & -0.019 & 0.056 \\ 
  b[(Intercept) word:türnapuida,] & 1.000 & 7896 & 0.048 & 0.035 & -0.021 & 0.048 & 0.116 \\ 
  b[(Intercept) word:tütar] & 1.000 & 8146 & -0.033 & 0.034 & -0.101 & -0.033 & 0.034 \\ 
  b[(Intercept) word:tüve] & 1.000 & 7273 & 0.051 & 0.034 & -0.015 & 0.051 & 0.120 \\ 
  b[(Intercept) word:ümmert] & 1.000 & 8661 & -0.007 & 0.028 & -0.062 & -0.007 & 0.047 \\ 
  b[(Intercept) word:uppus] & 1.000 & 7150 & -0.029 & 0.032 & -0.094 & -0.028 & 0.036 \\ 
  b[(Intercept) word:usselinki,] & 1.000 & 8084 & -0.012 & 0.040 & -0.089 & -0.012 & 0.066 \\ 
  b[(Intercept) word:vaesed] & 0.999 & 7031 & -0.020 & 0.030 & -0.079 & -0.020 & 0.038 \\ 
  b[(Intercept) word:vahele,] & 0.999 & 6831 & 0.042 & 0.027 & -0.010 & 0.042 & 0.095 \\ 
  b[(Intercept) word:vaisiküisi,] & 1.000 & 7441 & 0.007 & 0.035 & -0.060 & 0.006 & 0.076 \\ 
  b[(Intercept) word:vaisiküisist] & 0.999 & 7569 & 0.029 & 0.035 & -0.038 & 0.029 & 0.096 \\ 
  b[(Intercept) word:vait] & 1.000 & 7705 & 0.025 & 0.040 & -0.054 & 0.025 & 0.107 \\ 
  b[(Intercept) word:valada,] & 0.999 & 8224 & 0.030 & 0.034 & -0.035 & 0.031 & 0.099 \\ 
  b[(Intercept) word:valge] & 1.000 & 8275 & -0.001 & 0.034 & -0.068 & -0.001 & 0.066 \\ 
  b[(Intercept) word:valutave!] & 0.999 & 8650 & -0.010 & 0.033 & -0.073 & -0.011 & 0.054 \\ 
  b[(Intercept) word:varba] & 0.999 & 7969 & -0.007 & 0.034 & -0.073 & -0.007 & 0.059 \\ 
  b[(Intercept) word:vemmeldada!] & 1.000 & 7262 & -0.030 & 0.034 & -0.095 & -0.030 & 0.038 \\ 
  b[(Intercept) word:vennal] & 1.000 & 7477 & -0.036 & 0.040 & -0.113 & -0.037 & 0.042 \\ 
  b[(Intercept) word:vennanaistel] & 0.999 & 7466 & -0.020 & 0.035 & -0.088 & -0.021 & 0.047 \\ 
  b[(Intercept) word:vihma] & 1.000 & 7646 & 0.044 & 0.035 & -0.024 & 0.043 & 0.110 \\ 
  b[(Intercept) word:vii] & 1.000 & 6638 & -0.005 & 0.035 & -0.075 & -0.005 & 0.065 \\ 
  b[(Intercept) word:virvekirja,] & 1.000 & 8521 & 0.007 & 0.034 & -0.059 & 0.007 & 0.073 \\ 
  b[(Intercept) word:võta] & 1.000 & 6540 & -0.035 & 0.021 & -0.077 & -0.035 & 0.007 \\ 
  b[(Intercept) word:võtke] & 1.000 & 7322 & -0.071 & 0.034 & -0.138 & -0.071 & -0.005 \\ 
  b[(Intercept) word:Võtke] & 1.000 & 6387 & -0.072 & 0.035 & -0.142 & -0.073 & -0.006 \\ 
  b[(Intercept) performer:LK] & 1.000 & 2045 & 0.030 & 0.035 & -0.043 & 0.029 & 0.103 \\ 
  b[(Intercept) performer:LO] & 1.000 & 2076 & -0.046 & 0.035 & -0.120 & -0.046 & 0.025 \\ 
  b[(Intercept) performer:MH] & 1.000 & 1870 & 0.018 & 0.035 & -0.054 & 0.017 & 0.093 \\ 
  sigma & 1.002 & 1795 & 0.062 & 0.003 & 0.057 & 0.062 & 0.067 \\ 
  Sigma[word:(Intercept),(Intercept)] & 1.006 & 1191 & 0.003 & 0.000 & 0.002 & 0.003 & 0.004 \\ 
  Sigma[performer:(Intercept),(Intercept)] & 1.001 & 2557 & 0.004 & 0.006 & 0.000 & 0.002 & 0.020 \\ 
  mean\_PPD & 1.000 & 3927 & 0.201 & 0.004 & 0.194 & 0.200 & 0.208 \\ 
  log-posterior & 1.009 &  673 & 388.413 & 19.776 & 349.271 & 389.028 & 425.331 \\ 
   \bottomrule
\caption{Parameters plot with all word intercepts (shinystan)} 
\end{longtable}
%

%
%\include{chapter-appendix3}


%%%%%%%%%%%%%%%%%%%%%%%%%%%%%%%%%%%%%%%%%%%%%%%%%%%%%%%%%%%%%%%%%%%%%%
% Generate the bibliography.					     %
%%%%%%%%%%%%%%%%%%%%%%%%%%%%%%%%%%%%%%%%%%%%%%%%%%%%%%%%%%%%%%%%%%%%%%
%								     %
% NOTE: For master's theses and reports, NOTHING is permitted to     %
%	come between the bibliography and the vita. The command      %
%	to generate the index (if used) MUST be moved to before      %
%	this section.	
%\printindex%    % Include the index here. Comment out this line      %
%		% with a percent sign if you do not want an index.   %					     %
%								     %
%\nocite{*}      % This command causes all items in the 		     %
                % bibliographic database to be added to 	     %
                % the bibliography, even if they are not 	     %
                % explicitly cited in the text. 		     %
		%						     %
\bibliographystyle{apa-good.bst}  % Here the bibliography 		     %
\bibliography{qpannottes}        % is inserted.			     %
\index{Bibliography@\emph{Bibliography}}%			     %
%%%%%%%%%%%%%%%%%%%%%%%%%%%%%%%%%%%%%%%%%%%%%%%%%%%%%%%%%%%%%%%%%%%%%%


%%%%%%%%%%%%%%%%%%%%%%%%%%%%%%%%%%%%%%%%%%%%%%%%%%%%%%%%%%%%%%%%%%%%%%
% Generate the index.						     %
%%%%%%%%%%%%%%%%%%%%%%%%%%%%%%%%%%%%%%%%%%%%%%%%%%%%%%%%%%%%%%%%%%%%%%
%								     %
% NOTE: For master's theses and reports, NOTHING is permitted to     %
%	come between the bibliography and the vita. This section     %
%	to generate the index (if used) MUST be moved to before      %
%	the bibliography section.				     %
%								     %

%%%%%%%%%%%%%%%%%%%%%%%%%%%%%%%%%%%%%%%%%%%%%%%%%%%%%%%%%%%%%%%%%%%%%%


%%%%%%%%%%%%%%%%%%%%%%%%%%%%%%%%%%%%%%%%%%%%%%%%%%%%%%%%%%%%%%%%%%%%%%
% Vita page.							     %
%%%%%%%%%%%%%%%%%%%%%%%%%%%%%%%%%%%%%%%%%%%%%%%%%%%%%%%%%%%%%%%%%%%%%%

\begin{vita}
Sarah Marie Ransom-Laud
was born in Sarasota, Florida on 11 July 1990. Her early career was as a songwriter and recording artist, releasing several full-length albums combining digital and analog recording methods over the last decade. 

In 2018 she received the Bachelor of Arts degree in Linguistics from the University of Wisconsin, Milwaukee. Upon graduation, she moved to Austin with her partner, Kavi, and taught English as a Second Language (ESL) to adults until her admission into the PhD program in the department of Linguistics at the University of Texas at Austin, where she matriculated in Fall 2019. 
%Craig William McCluskey
%was born in Minneapolis, Minnesota on 20 May 1950, the son of
%Dr. William R. McCluskey and Lucilla W. McCluskey.  He received the Bachelor
%of Science degree in Engineering from the California Institute of Technology
%and was commissioned an Officer in the United States Air Force in 1971.
%He entered active duty in October, 1971, and was stationed in Denver, Colorado,
%Colorado Springs, Colorado, Panama City, Florida, and Sacramento, California.
%He separated from the USAF in 1975 and worked as an engineer for several small
%electronics companies in California before moving to Colorado Springs, Colorado
%to work for Hewlett-Packard in 1979. He left Hewlett-Packard in 1989 and joined
%a small company based in Herndon, Virginia, working out of his house as a
%``remote'' engineer designing parts of the Alexis satellite for Los Alamos
%National Laboratories. Laid off when his portion of the satellite was
%completed, he applied to the University of Texas at Austin for enrollment
%in their physics program. He was accepted and started graduate studies in
%August, 1991.

\end{vita}

\end{document}
