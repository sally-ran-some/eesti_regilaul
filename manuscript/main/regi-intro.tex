\chapter{Introduction}
\index{Introduction@\emph{Introduction}}%is

\section{On words becoming lyrics}
To join song and become lyrics, meaningful linguistic material must be modified to fit the strict temporal structure of music, while both poet and performer are tasked with preserving intelligibility sufficient for semantic interpretation of the whole. Thus the rhythm of language must fit into the song's rhythm, but enough of the language's own rhythm must remain in order for the lyrics to have meaning. 

The study of metrical prosody often focuses on the patterns in the texts independent of the songs. In this paper, I analyze the acoustic-phonetic correlates of linguistic rhythm {\it within} the context of the song. The aim of this paper is to analyze the acoustic realizations of metrics in Estonian folksong lyrics. 
Related to the Kalevala meter of Finnish epic and Balto-Finnic runosong at large, Estonian's {\it regilaul} meter is a syllabic-accentual trochaic tetrameter.  
 I define metrical as the mapping of the pattern on a frame formed of equal time intervals  \citep{essensPovel1985}. Metrical patterns have been demonstrated to be more easily replicated by humans, something necessary for both the synchronization of musicians performing together and for the transmission of an oral tradition of songs. 
 
  
   
 An oral tradition going back centuries, annotations of the songs come from ethnomusicologists, who based them off recordings collected for the Estonian Folklore Archives. 
  Previous studies have compared acoustic measurements with said transcriptions. 
Here I introduce the use of beat-tracking algorithms to provide a rigorous definition of the location of beats, using the verse annotations as a guide. 


\section{Phonetics of Metrics in Estonian Word Prosody}

Primary lexical stress in native Estonian words is fixed, falling on word-initial syllables. There are no stress minimal pairs at the lexical level: thus, primary stress is said to be {\it demarcative} or {\it identificational}\citep{lehiste1992}, functioning to indicate the boundary of a new word. Syllables following primary stress are unstressed, and secondary stress is attested to fall on each odd syllable in polysyllabic words. Duration functions at several levels of Estonian prosody: it is the strongest correlate of both clear speech and stress \citep{lippusAsuMari2014} when compared with measurements of f0 and spectral emphasis, and is independently contrastive at the segmental level as illustrated in minimal triads in \ref{qperm}. Secondary stress was found to be significantly different in duration from both primary and peninitial unstressed syllables: interestingly, secondary stressed syllables were not shown to differ greatly from their even-positioned neighbors (excepting the peninitial). For the sake of simplicity, this study focuses only on the contrast between first (primary stressed) and second (unstressed) syllables. 

%\cite{eekmeisterUralica98}

In primary stress position, there are three contrastive syllable quantities. The first (Q1) is described as short, the next (Q2) as long, and the heaviest (Q3) as overlong. 

%This ternary contrast has long been the subject of debate in the phonological literature of metrics: Q3 syllables have been analyzed both as a monosyllabic foot \citep{princeMetricalTheoryEstonian1980} and as a trimoraic syllable \citep{hayesCompensatoryLengtheningMoraic1989, kuznetsovaEstonianWordProsody2018,prillopMoraeEstonianReply2020}. 


 \begin{table}[htb]
\centering
\begin{tabular}{ccc}
\hline
Q1 & Q2 & Q3 \\
{\it kodi	} 	& {\it koodi }	& {\it koodi }	\\  
/ko.ti/		& /koː.ti/	& /koːː.ti/	\\
\hline
		& {\it koti }	& {\it kotti}	 \\
		& /kot.ti/	& /kotː.ti/	\\
\hline
		& {\it gooti}	& {\it kooti} 	\\
		& /koːt.ti/	& /koːːt.ti/	\\
\hline
\end{tabular}
\label{qperm}
\caption{segmental permutations of initial syllable quantity}
\end{table}

In the first row of \ref{qperm}, we see a minimal triad of the ternary quantity contrast in open first syllables. The Q2 and Q3 columns demonstrate all the other ways this contrast can be realized using the same segment identities in closed syllables. 

\begin{exe}
\ex \gll laul-da \\
	{[ˈlɑuːl.dɑ]} \\
	sing-\Tr{} 
	\glt	`singing'
\ex 	ööbik \\
	{[ˈøː.pikː]} \\
	nightingale.\Nom{} 
	\glt`nightingale'
\label{disyllables}
\end{exe}
Q1 and Q2 syllables can be both stressed and unstressed, while Q3 is only present in stressed positions, attracting stress to its (non-initial) syllable in compound and loan words. Peninitial syllables can only be Q1 or Q2, illustrated in \ref{disyllables}. 


%\subsection{Metrical Prosody in Estonian} 
% \begin{table}[htb]
%\centering
%\begin{tabular}{lcc}
%\hline
%
%Q1 &		 sada 		& 	kabi  \\  
%	&	 {\it `hundred'} 	&	 {\it`hoof' }\\
%\hline
%Q2 &		saada 		&	kapi \\
%	&	 {\it`send' }		&	{\it`of the cupboard' }		\\
%\hline
%Q3 &		saada 	&	 kappi 	\\
%	&	{\it`recieve' }	&	{\it`into the cupboard' }	\\
%\hline
%\end{tabular}
%
%\caption{ternary syllable weight contrast}
%\label{qexamps}
%\end{table}
%
%Estonian is famous for its ternary quantity distinction: both vowels and consonants have three distinct lengths or degrees. As can be seen in the examples in \ref{qexamps}, the length of a given segment can indicate lexical contrasts as in sada vs saada (`hundred' vs `send') (root minimal pairs) and also indicate case: the difference between `receive' and `send' is in the quantity of the first syllable' or directional case marking `of' versus `'into' the cupboard. 
%

\subsection{Metrical Structure in Music} 
In music, the smallest prosodic constituent is an individual note event whose relationship to the other notes in the song are indicated by the time signature, i.e., \lilyTimeSignature{3}{4} and \lilyTimeSignature{4}{4}. The denominator corresponds to the number of divisible beats of a ``whole'' note (\semibreve), while the numerator refers to the number of ``beats'' in a single measure. In  \lilyTimeSignature{4}{4}, a whole note is sustained for the same duration as  four quarter notes (\crotchet) in the same measure, and each measure must culminate in enough notes and rests to equal a whole note. 
 

\subsection{Metrical Principles of Estonian folksong}

Estonian {\it regilaul} is part of the Finnic runosong tradition shared by several other members of the Finnic language family: Finnish, Karelian, Votic, Ingrian, and Livonian \citep{rossLehiste2001}. The metrical basis of the tradition is a trochaic tetrameter often referred to as the Kalevala meter \citep{oras2019}, which is realized in Estonian 20th century work as syllabic-accentual trochaic tetrameter \citep{lotmanLotman2013}. 

Each verse line has four beats with eight syllable-note positions which can also be occupied by rests or, in the case of trisyllables, two sixteenth syllable-notes. The ``beat" or ``ictus" position falls on the note corresponding to the first beat of a measure, and on every other following syllable-note in the invariant form of eight eighth notes. Runic songs that follow quantity rules for trochaic meter oppose metrically strong and weak positions by means of syllable quantity and stress. Ictus position, or on the beat, prefers syllables that are both long and stressed but avoids short stressed syllables: these can occur off the beat, while this position is avoided by long syllables. 
These ``singable songs" \citep{tormis1985} follow a metrical pattern such that a given {\it regilaul} text can be sung to any of the numerous {\it regilaul} melodies \citep{rossLehiste2001}. 


\begin{figure}[ht]
\begin{center}
\includegraphics[width=300pt]{figures/045.png}
\caption{``Millal saame sinna maale"}
\label{045}
\end{center}
\end{figure}
\ref{045} illustrates the invariant pattern of eight syllables notes each occupying one eighth of a \lilyTimeSignature{4}{4} measure. 

\begin{figure}[hb]
\begin{center}
\includegraphics[width=300pt]{figures/077.png}
\caption{notation of ``Loomine" performed by Liisu Orik in 1965}
\label{077}
\end{center}
\end{figure}
\ref{077} illustrates a melody variation with seven syllables: in the first verse, the last (heavy) syllable is extended to fill a quarter note, in the second the last syllable is sung as an eighth note and followed by a quarter rest (\quaverRest). 

How do the word-prosodic requirements negotiate with the imposed prosodic hierarchy of music? The rhythmic organization of song is said to integrate the prosodic structure of the language with musical rhythmic principles \citep{palmer1992}. However, earlier studies of {\it regilaul} explored temporal aspects of the songs and found that duration characteristics that would usually indicate important semantic differences lost their distinctions partially or entirely. Assuming that the intention of the singer is for the lyrics to be understood, I hypothesize that if some durational correlates of contrasting word-prosodic constituents are made less distinct in the process of compromising with the song that some other acoustic correlate of the relevant contrast at the word-prosodic level will be present, if not enhanced.  

\begin{figure}[htbp]
\begin{center}
\includegraphics[width=300pt]{figures/094.png}
\caption{music notation of ``The King Game" as performed by Liisa Kümmel}
\label{The King Game}
\end{center}
\end{figure}


\section{Previous Studies with {\it regilaul}}

The intuitions of those who study the runosong tradition is that the burden of upholding the temporal structure of the song is the result of symbiosis between the musical rhythm and the natural prosodic features of the lyrical text \citep{ross1992, tampere1934}: the song's melody a musical abstraction of the natural prosody of spoken runic verse 
inspiring decades of research at the interface of metrical phonetics and computational musicology \cite{ruutel1999}.
%
% 
%Melodic accents coincide with the stressed syllables, the pitch resolves as the phrase resolves. Said to be a musical abstraction of the natural prosodic intonation of the 8 syllable (spoken) runoverse .



\begin{figure}[htb]
\begin{center}
\includegraphics[width=300pt]{figures/022.png}
\caption{The swinging song `Kiik tahab kindaid' analyzed in \cite{ross1989,ross1992}}
\label{022swing}
\end{center}
\end{figure}

Swinging songs are characterized by a swinging rhythm with alternating long and short notes, differing from the main body of {\it regilaul}'s iconic isochrony. 
Jaan Ross, a musicologist and native speaker of Estonian analyzed a 1936 recording of the swing song `Kiik tahab kindaid' \ref{022swing}, publishing results on syllable-note duration \cite{ross1989} and later the vowel quality of odd-numbered syllables \citep{ross1992}. In the study on syllable-note durations, 

In the second, Ross measured formant frequencies f1 and f2, finding a reduced vowel space in song compared to measurements in spoken Estonian. However, upon examination of the song, it is clear that all the vowel space measurements are from syllable-notes in non-initial positions of Estonian words: that is, the sample of vowels taken from  the song were all unstressed, and compared to a mixed sample of spoken Estonian. Thus, the conclusion needs to be evaluated again with comparable samples. 

%
In 1992, Ilse Lehiste asked "Whether there is a correlation between poetic metre and the prosodic structure of a language.\citep{lehiste1992} by means of measuring the acoustic-phonetic realizations of so-called trochaic metrical poetic patterns across several languages including Estonian and Finnish. While these languages share in the more general Balto-Finnic tradition of {\it runosong} utlizing what is called the {\it Kalevala} meter, there were significant differences in the phonetic realizations of trochees in each language. 


In 1994, their collaboration begins. Ross and Lehiste published several papers examining the temporal dimensions of Estonian word prosody and metrical prominence in {\it regilaul} folksongs. In \citep{rossLehiste1994}, they conclude that duration differences ordinarily present at the word level (stressed-unstressed) are {\it ``lost"} to the temporal restrictions of the song. In another paper, syllable-notes are again measured, this time examining the role of syllabic quantity in the song. They likewise conclude that the duration of syllable-notes in {\it regilaul} match more closely with the metrical structure of the songs, that is, the durations of syllable-notes are best predicted by their beat position in the song: on the beat, syllables are longer, off the beat shorter. They extend this finding to conclude that the song "dominates" the metrical status of the words, claiming that the intelligibility of lyrics is enhanced more strongly by ``top-down" processes (i.e., semantic context). 

\citep{rossLehiste1996}. \\

%to paraphrase: Semantically relevant oppositions of word initial short-long and long-short disyllabic units in speech are not kept completely intact in folk songs. Short-long disyllables are treated in a different manner by the performer, depending on whether their initial syllable occurs at a rise or at a fall in a foot
\citep{rossLehiste1998} Analyzed and concluded that the {\it regilaul} lyrics are the result of an interaction between word and song prosodic hierarchies. This conclusion relied critically on measuring the durations of syllable-notes, where they found being on or off the beat was the better predictor for duration. \\


Finally, in \citep{rossLehiste2001} summarize their body of work until that point and extend their findings with fine-grained acoustic phonetic measurements of segment durations within syllable-notes. In their discussion, they mention that some of the durational differences not present at the syllable-note level are present at the segmental level, in complex codas. 

\section{The present study}

To extend the findings of this body of work, the present study examines two acoustic correlates of prominence at the segmental level: namely, the syllable nucleus. Because the vowel is the most sonorant part of a syllable, it is the most acoustically and perceptually salient portion of the syllable. Measuring the vowel duration will also offer indirect information about the rhyme as a whole: the presence of codas and complex codas should have an effect on the vowel duration in isochronous syllable-note sequences. In the cases where quantity is distinguished by coda consonant length, the syllable nucleus (measured by vowel duration) will be necessarily shorter to accommodate the geminate or complex coda. In cases where quantity is indicated by the length of the vowel, the opposite should be true. This study therefore examines the ternary quantity distinction in the context of its syllable shape: no coda (CV), single coda (CVC, CVVC), and complex coda (CVCC, CVCCC) rather than collapsing all syllable shapes according to their quantity. This allows a closer look at the microprosodic features at the segmental level. 

In addition to vowel duration, I also include vowel dispersion measured by the euclidean distance from the center of each singer's vowel space on the (f1, f2) plane. A larger vowel perimeter generally corresponds to hyper-articulation or clear speech, while a smaller perimeter with hypo-articulation or reduction \cite{lindblom1990, smiljanic2005}. In natural Estonian speech, reduction is only allowed on unstressed syllables, with /i/ being the most resistant \citep{eekMeister1998}. This measurement is included for two reasons: one, the acoustic correlates of prominence in both language and music are almost always not a single cue but a convergence of several cues, and two, to see if the findings of \citep{ross1992} extend to the style of {\it regilaul} songs that make up the largest portion of the body of work (i.e., non-swinging songs). 


More recently, syllabic-accentual trochaic tetrameter has received some statistical attention to its textual structure \citep{lotmanLotman2013} 

and to rhythmic variation in singing \citep{oras2019}. 
\subsection{Hypotheses}
Null hypothesis: on or off-beat position is best predictor for vowel duration of syllables in all categories: stressed, unstressed, short, long, and overlong. This would mean that the syllable nucleus is proportional to the total syllable duration, and that the duration contrast is indeed ``lost." 

H_{1}: duration contrasts for syllable quantity will be evident in the vowel duration, with vowel duration {\it decreasing} as syllable weight increases. Given the findings of \citep{rossLehiste2001}, isochronous syllable-notes would result in heavier syllables having shorter nuclei to accommodate for the coda and complex codas that distinguish the syllable weights from each other. 

H_{2}: stress/unstress contrasts will be evident in the nucleus in terms of hypo and hyper articulation. For this I measure both nucleus duration and vowel space dispersion. 

