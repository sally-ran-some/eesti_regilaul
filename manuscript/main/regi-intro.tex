\chapter{Introduction}
\index{Introduction@\emph{Introduction}}%is

\section{When words become lyrics}
To join song and become lyrics, meaningful linguistic material must be modified to fit the strict temporal structure of music. Consequently, both poet and performer must combine two independent rhythmic systems: the rhythm of the language must fit into the song's rhythm, but enough of the language's own rhythm must remain in order for the lyrics to have meaning. 
In particular, the rhythmic features of Estonian prosody include a ternary syllable weight contrast that interacts closely with word stress \citep{lehiste1960,lehiste1965,lehiste1978,eekMeister1998,asuPire2009}.  
Duration functions at several levels of Estonian prosody: it is the strongest correlate of both clear speech and stress \citep{lippusAsuMari2014} and is independently contrastive at the segmental level. In primary stressed syllables, this segmental contrast plays a role in the phonetic realization of the ternary weight contrast illustrated in the minimal triads in \ref{qexamps}.  How do Estonian poets and singers negotiate these complex temporal aspects into a meaningful song? The aim of this paper is to analyze the acoustic-phonetic realizations of metrics in Estonian lyrical folksongs known as {\it regilaul} (\(/re.ki.laul/\)). 

 \begin{table}[htb]
\centering
\begin{tabular}{lcc}
\hline

Q1 &		 sada 		& 	kabi  \\  
	&	 {\it `hundred'} 	&	 {\it`hoof' }\\
\hline
Q2 &		saada 		&	kapi \\
	&	 {\it`send' }		&	{\it`of the cupboard' }		\\
\hline
Q3 &		saada 	&	 kappi 	\\
	&	{\it`recieve' }	&	{\it`into the cupboard' }	\\
\hline
\end{tabular}

\caption{ternary syllable weight contrast}
\label{qexamps}
\end{table}

As can be seen in the examples in \ref{qexamps}, the length of a given segment can indicate lexical contrasts as in sada vs saada (`hundred' vs `send'), or differentiate case. In {\it kapi} and {\it kappi}, marking `of' versus `'into' the cupboard is indicated by the quantity of the first syllable. 
\section{Metrics}

 I define metrical as the mapping of the pattern on a frame formed of equal time intervals. These patterns have been demonstrated to be more easily replicated by humans \citep{essensPovel1985}, necessary for the transmission of an oral tradition of songs and for the synchronization of human musicians playing together. 

\subsection{Phonetics of Estonian Segments and Syllables}


Primary word stress in native Estonian words is fixed, falling on word-initial syllables. By virtue of its predictability, it is not lexically contrastive: instead, it is described as identificational, facilitating intelligibility by demarcating prosodic boundaries \citep{lehiste1965, lehiste1978,lehiste1992, eekMeister1998, lippusAsuMari2014}. However, it is only in primary stressed syllables that the ternary quantity contrast falls. \begin{center}
\begin{exe}
\ex \gll laul-da \\
	{[ˈlɑuːl.dɑ]} \\
	sing-\Tr{} 
	\glt	`singing'
\ex 	\gll ööbik \\
	{[ˈøː.pikː]} \\
	nightingale.\Nom{} 
	\glt`nightingale'
\label{disyllables}
\end{exe}
\end{center}
Q1 and Q2 syllables can be both stressed and unstressed, while Q3 is only present in stressed positions, attracting stress to its (non-initial) syllable in compound and loan words. Pen-initial syllables can only be Q1 or Q2, illustrated in \ref{disyllables}. 

 \begin{table}[htb]
\centering
\begin{tabular}{ccc}
\hline
Q1 & Q2 & Q3 \\
{\it kodi	} 	& {\it koodi }	& {\it koodi }	\\  
/ko.ti/		& /koː.ti/	& /koːː.ti/	\\
\hline
		& {\it koti }	& {\it kotti}	 \\
		& /kot.ti/	& /kotː.ti/	\\
\hline
		& {\it gooti}	& {\it kooti} 	\\
		& /koːt.ti/	& /koːːt.ti/	\\
\hline
\end{tabular}
\label{qperm}
\caption{segmental permutations of initial syllable quantity}
\end{table}

In the first row of \ref{qperm}, we see a minimal triad of the ternary quantity contrast in open `short' first syllables (Q1). The Q2 `long' and Q3 `overlong' columns demonstrate all the other ways this contrast can be realized using the same segment identities in closed syllables. These three syllable weights can be lexically contrastive, differentiating semantically distinct roots, or morphologically distinguishing varying degrees of grammatical case, direction, aspect, \&c. 



 \subsection{Musical Metrics} 
In music, the smallest prosodic constituent is an individual note event whose relationship to the other notes in the song are indicated by the time signature, i.e., \lilyTimeSignature{3}{4} and \lilyTimeSignature{4}{4}. The denominator corresponds to the number of divisible beats of a ``whole'' note (\semibreve), while the numerator refers to the number of ``beats'' in a single measure. In  \lilyTimeSignature{4}{4}, a whole note is sustained for the same duration as  four quarter notes (\crotchet) in the same measure, and each measure must culminate in enough notes and rests to equal a whole note. 





\subsection{Metrical Principles of Estonian folksong}

\begin{figure}[ht]
\begin{center}
\includegraphics[width=300pt]{figures/045.png}
\caption{``Millal saame sinna maale"}
\label{045}
\end{center}
\end{figure}

Estonian {\it regilaul} is part of the Finnic runosong tradition shared by several other members of the Finnic language family: Finnish, Karelian, Votic, Ingrian, and Livonian \citep{rossLehiste2001}. These ``singable songs" \citep{tormis1985} follow a metrical pattern such that a given {\it regilaul} text can be sung to any of the numerous {\it regilaul} melodies \citep{rossLehiste2001}. The metrical basis of the tradition is a trochaic tetrameter often referred to as the Kalevala meter \citep{oras2019}, which is realized in Estonian 20th century work as syllabic-accentual trochaic tetrameter \citep{lotmanLotman2013}. This pattern opposes metrically strong and weak positions by means of syllable quantity and word stress. Ictus position, which corresponds to `on the beat' in musical meter, prefers syllables that are both heavy (long and overlong) and stressed, but avoids short stressed syllables. Short stressed syllables can occur off the beat, but this position is avoided by the heaviest (Q3) syllables. 

Composed of four of these trochees, each verse line has four beats with eight syllable-note positions (as in  \ref{045} ). In  \lilyTimeSignature{4}{4} measure, the beat (ictus) falls on the note corresponding to the first beat of a measure, and on every other following syllable-note, counted as one (and) two (and) three (and) four (and).
The two measures in \ref{077} illustrate a melody variation with eight syllables fit into seven positions: in each measure, two weak syllables are halved, with the trisyllabic dactyl having one eighth and two sixteenth syllable-notes. In the first measure, the "extra" position gained from this variation is occupied by the last (heavy) syllable, which is extended to fill a quarter note. In the second measure, the syllable is sung as an eighth note, and the extra eighth position filled with a rest (\quaverRest). 


\begin{figure}[hb]
\begin{center}
\includegraphics[width=300pt]{figures/077.png}
\caption{notation of ``Loomine" performed by Liisu Orik in 1965}
\label{077}
\end{center}
\end{figure}



%\begin{figure}[htbp]
%\begin{center}
%\includegraphics[width=300pt]{figures/094.png}
%\caption{music notation of ``The King Game" as performed by Liisa Kümmel}
%\label{The King Game}
%\end{center}
%\end{figure}


\section{Previous Studies with {\it regilaul}}
How do the word-prosodic requirements negotiate with the imposed prosodic hierarchy of music? The rhythmic organization of song is said to integrate the prosodic structure of the language with musical rhythmic principles \citep{palmer1992}. The intuitions of those who have studied the runosong tradition is that the burden of upholding the temporal structure of the song is the result of symbiosis between the musical rhythm and the natural prosodic features of the lyrical text \citep{ross1992, tampere1934}: the song's melody a musical abstraction of the natural prosody of spoken runic verse.





%inspiring decades of research at the interface of metrical phonetics and computational musicology \cite{ruutel1999}.
%
% 
%Melodic accents coincide with the stressed syllables, the pitch resolves as the phrase resolves. Said to be a musical abstraction of the natural prosodic intonation of the 8 syllable (spoken) runoverse .



%\begin{figure}[htb]
%\begin{center}
%\includegraphics[width=300pt]{figures/022.png}
%\caption{The swinging song `Kiik tahab kindaid' analyzed in \cite{ross1989,ross1992}}
%\label{022swing}
%\end{center}
%\end{figure}


%Jaan Ross, a musicologist and native speaker of Estonian analyzed a 1936 recording of the swing song `Kiik tahab kindaid' \ref{022swing}. Swing songs are a special category of regilaul that intentionally flauts the pattern that make up the remaining body of regilaul.  Swinging songs are characterized by a swinging rhythm with alternating long and short notes, differing from the main body of {\it regilaul}'s iconic isochrony. %
%of odd-numbered syllables \citep{ross1992}. 
% 
% In the study on syllable-note durations, syllable-note duration \cite{ross1989} and later
%
%In the second, Ross 

%

\subsection{}
%\citep{lehiste1992} examined the language-specific properties of meter, measuring the acoustic-phonetic realizations of trochaic patterns (bisyllabic feet with accented first syllables) in the poetic traditions of Finnish, Estonian, Swedish, and Lithuanian. 


Beginning in the early nineties and extending into the millenium, a collaboration between two native speakers of Estonian laid the groundwork for this study. Musicologist Jaan Ross and Phonologist Ilse Lehiste found common ground investigating the acoustic correlates of regilaul: Ross had publications on one song \cite{ross1989, ross1992}, but the inclusion of a linguist in the study of lyrical realization proved crucial. Together they studied the interactions of musical and prosodic meter, comparing nominal transcriptions of syllable-notes to their absolute durations. The findings overwhelmingly were that in nominally isochronous syllable-notes, the absolute durations patterned into two duration categories: short and long. Compared to the inherent durational properties found in Estonian word prosody (primary lexical stress and unstress, ternary syllable quantity), long and short syllable-note category was best predicted by its position relative to the beat: long notes on the beat, short ones off \citep{rossLehiste1994, rossLehiste1996,rossLehiste1998}. They concluded from these findings that semantically relevant duration contrasts ordinarily present in spoken Estonian were ``lost" in song, and interpretation of lyrical content must rely heavily on ``top-down" processes such as semantic context. Later, they summarize and revisit the question, this time also measuring the duration of segments within the syllable-notes.  In their discussion, they mention that some of the durational differences not present at the syllable-note level are present at the segmental level, most visible in heavy syllables with complex and geminate codas \citep{rossLehiste2001}


\section{The present study}


To extend the findings of this body of work, the present study examines two acoustic correlates of prominence at the segmental level: namely, the syllable nucleus. Because the vowel is the most sonorant part of a syllable, it is the most acoustically and perceptually salient portion of the syllable. Measuring the vowel duration will also offer indirect information about the rhyme as a whole: the presence of codas and complex codas should have an effect on the vowel duration in isochronous syllable-note sequences. In the cases where quantity is distinguished by coda consonant length, the syllable nucleus (measured by vowel duration) will be necessarily shorter to accommodate the geminate or complex coda. In cases where quantity is indicated by the length of the vowel, the opposite should be true. This study therefore examines the ternary quantity distinction in the context of its syllable shape: no coda (CV), single coda (CVC, CVVC), and complex coda (CVCC, CVCCC) rather than collapsing all syllable shapes according to their quantity. This allows a closer look at the microprosodic features at the segmental level. 


  Previous studies have compared acoustic measurements with the nominal values given in transcriptions made by ethnomusicologists. However, as regilaul is an oral tradition, these annotation represent neither the intuitions of the songs' composers, nor those of the performers. Here I introduce the use of beat-tracking algorithms to define beat location by acoustic means. \\
  
  
In addition to vowel duration, I also include vowel dispersion measured by the euclidean distance from the center of each singer's vowel space on the (f1, f2) plane. A larger vowel perimeter generally corresponds to hyper-articulation or clear speech, while a smaller perimeter with hypo-articulation or reduction \cite{lindblom1990, smiljanic2005}. In natural Estonian speech, reduction is only allowed on unstressed syllables, with /i/ being the most resistant \citep{eekMeister1998}. An early paper by Ross measured formant frequencies f1 and f2, finding a reduced vowel space in song compared to measurements in spoken Estonian\cite{ross1992}. However, upon examination of tokens included in the analysis, it is clear that the sample of sung syllable-notes consists entirely of syllables in non-initial positions of Estonian words: that is, the sample of vowels taken from  the song were all unstressed. Conversely, the spoken Estonian used for this comparison contained stressed and unstressed syllables. 


This measurement is included for two reasons: one, the acoustic correlates of prominence in both language and music are almost always not a single cue but a convergence of several cues. If duration is less available for word-level contrasts when it is constrained by musical meter, vowel dispersion is a viable candidate for indicating prominence in its stead.  The second reason is to follow up on the findings of \citep{ross1989,ross1992}, this time comparing vowel space in stressed and unstressed syllables within the song.  

\subsection{Hypotheses}
Earlier studies of {\it regilaul} explored temporal aspects of syllable-notes and found that duration characteristics that would usually indicate important semantic differences lost their distinctions either partially or entirely. The present study wishes to extend these findings by pursuing two possibilities. First, it is possible that apparent neutralization at one level (i.e., syllabic) could still be present at another level (i.e., segmental). Second, if durational differences usually indicative of prominence are indeed neutralized, it is possible that a secondary cue to prominence is enhanced in compensation. 

\subsection{Duration of Syllable Nuclei}
Given the findings of \citep{rossLehiste2001}, isochronous syllable-notes would result in heavier syllables having shorter nuclei to accommodate for the coda and/or complex codas that distinguish the syllable weights from each other. This finding would confirm their earlier results while simultaneously contributing acoustic evidence that prosodic information that is usually suprasegmental is preserved at the segmental level. 
	\begin{itemize}

	\item \(HQ\): duration contrasts for syllable quantity will be evident in the vowel duration, with vowel duration {\it decreasing} as syllable weight increases.
	\item \(HQ_{\emptyset}\): on or off-beat position is best predictor for vowel duration of syllables in all quantities. 

	\item \(HA\): duration contrasts for word-level stress will be evident at the segmental level after taking into account a syllables beat position in the song. 
	\item \(HA_{\emptyset}\): on or off-beat position of the song is best predictor for vowel duration of syllables in both stressed and unstressed syllables. 

	\end{itemize}

\subsection{Dispersion of Perimeter Vowels in F1,F2 Space}
An earlier study of vowel quality \cite{ross1992} found a reduction in the dispersion of vowels in regilaul singing compared to running speech. However, all vowel tokens in that study were taken from syllables that are unstressed at the words level, while the running speech comparison data included all stress positions. Thus we do not know if the vowel space is best predicted by {\it style} i.e., spoken or sung, or by word stress, or a combination of both. 
To begin probing this question further, I include measurements of vowel space of stressed and unstressed sung syllables. 
	\begin{itemize}
	\item \(HS\): stress/unstress contrasts will be evident in the nucleus in terms of hypo and hyper articulation. For this I measure both nucleus duration and vowel space dispersion. 
	\item \(HS_{\emptyset}\): vowel dispersion differences in syllables are random. 
	\end{itemize}



