\chapter{Conclusion}
\index{Conclusion@\emph{Conclusion}}%

%Vowel Duration: \\
%We failed to reject the null hypothesis for vowel duration: while the song level does dominate the temporal domain, there is an inverse relationship to vowel duration and word-level duration contrasts for Q1, Q3, and stressed syllables. So while the song always wins, there is clearly something else going on. It is possible that it is as simple as note-syllable isochrony, and the vowel decrease is simply an accommodation to give the ictus level the appropriate duration syllable. 
%
%Vowel Space: \\
%
%After confirming that song position is the dominant hierarchy for temporal constraints, we continue on to see how vowel space plays out in different levels of prosodic prominence. We are able to reject the null hypothesis with data to support a clear relationship between prominence and vowel space, this time at both prosodic levels (song and word.) This suggests that for the song, ``preserve contrast" ranks lower than the temporal constraints, but that so long as the strong positions in the song are satisfied, prominence can be indicated by vowel space at multiple levels of the prosodic hierarchy.  


