

ok the ``SWINGING SONGS" have fuckin' short stressed and long unstressed syllables, how the actual fuck did I miss that

ok, compare..things...but evidence that heavy syllables in short notes is common/acceptable/thought of as ``swinging" which...sounds like an intuition that they were ``switching" the length pattern, flouting it? YES. 

short-long, but underlying metrical pattern of text is still Kalevala. 
So, ``the beat" in a measure doesn't always correspond to the thing with the longest duration. What happens when a heavy syllable is sung in half the breadth of its word-weak neighbor? 

HYPOTHESIS: the word-level stress is {\it used} as the musical accent, the way to keep it. In the swinging melodies,
or, is the inverse relationship predictable? Do Q3 occur in off-ictus? In the non-swinging songs? 

regardless, duration is being modified in a structured, predictable way, so fine-grained acoustic-phonetic details of the signal of the sing can be examined. This way we have access to some of the acoustic signatures of the intuitions of these native singers have about their language. The song provides controls for variables of interest, to allow for exploration of the acoustic properties of sung constituents, and understand some of what the singer thought about their language (and of course, on the meaning of the song) based on the physical data of their performance. 