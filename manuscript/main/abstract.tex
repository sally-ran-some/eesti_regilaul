% XeLaTeX can use any Mac OS X font. See the setromanfont command below.
% Input to XeLaTeX is full Unicode, so Unicode characters can be typed directly into the source.

% The next lines tell TeXShop to typeset with xelatex, and to open and save the source with Unicode encoding.

%!TEX TS-program = xelatex
%!TEX encoding = UTF-8 Unicode

\documentclass[12pt]{article}
\usepackage{geometry}                % See geometry.pdf to learn the layout options. There are lots.
\geometry{letterpaper}                   % ... or a4paper or a5paper or ... 
%\geometry{landscape}                % Activate for for rotated page geometry
%\usepackage[parfill]{parskip}    % Activate to begin paragraphs with an empty line rather than an indent
\usepackage{graphicx}
\usepackage{amssymb}

% Will Robertson's fontspec.sty can be used to simplify font choices.
% To experiment, open /Applications/Font Book to examine the fonts provided on Mac OS X,
% and change "Hoefler Text" to any of these choices.

\usepackage{fontspec,xltxtra,xunicode}
\defaultfontfeatures{Mapping=tex-text}
\setromanfont[Mapping=tex-text]{Hoefler Text}
\setsansfont[Scale=MatchLowercase,Mapping=tex-text]{Gill Sans}
\setmonofont[Scale=MatchLowercase]{Andale Mono}

\title{abstract}
\author{sally ransom}
%\date{}                                           % Activate to display a given date or no date

\begin{document}
\maketitle
This paper presents an empirical analysis of acoustic-phonetic measurements of sung Estonian lyrics using an oscillator-entrainment model. In physics, entrainment is the process by which two independent systems with differing periods assume the same period. 
Measurements were extracted from {\it a capella} recordings of traditional Estonian folksongs called {\it regilaul} \(/re.ki.laul/\), which is part of the ancient indigenous oral tradition of Balto-Finnic {\it runosong}. The songs are described as utliizing a trochaic tetrameter, or four measures of trochaic (long-short) feet. In verses with nominally isochronous notes, the absolute durations of notes falling off the beat are shorter compared with their "on the beat" neighbors.  Conversely, Estonian demonstrates foot isochrony in both conversational and read speech. Thus, the shorter the first syllable, the longer the second, so that the duration of the second syllable is inversely proportional to that of the first. Additionally, primary stress in native Estonian words is fixed to the first syllable, with unstress immediately following. The interaction of stress and length in word prosody with the metrical pattern of the song results in trochees that are long-short across word boundaries: short positions in the song can be stressed (word-initial) or unstressed (pen-initial). Previous studies of regilaul found that the absolute durations of nominally isochronous syllable-notes patterned into two categories, short and long, irrespective of word stress. Unstressed syllables in musically prominent positions (``on" the beat) were as long as stressed syllables in their same positions, and both stressed and unstressed syllables ``off" the beat were shorter still. This is the expected result of the entrainment of syllable-period to musical note-period. I further investigate effect of interaction of these two rhythmic systems at the sub-syllabic period level, the alternation of consonant and sonorant phoneme segments. Results show that word-level stress and unstress is evident in the vowel durations of syllable-notes off the musical beat, but not apparent in syllable-notes that are on the beat. Therefore, when musical trochees cross word boundaries, word-initial syllables are acoustically distinct from pen-initial syllables. 




% For many users, the previous commands will be enough.
% If you want to directly input Unicode, add an Input Menu or Keyboard to the menu bar 
% using the International Panel in System Preferences.
% Unicode must be typeset using a font containing the appropriate characters.
% Remove the comment signs below for examples.

% \newfontfamily{\A}{Geeza Pro}
% \newfontfamily{\H}[Scale=0.9]{Lucida Grande}
% \newfontfamily{\J}[Scale=0.85]{Osaka}

% Here are some multilingual Unicode fonts: this is Arabic text: {\A السلام عليكم}, this is Hebrew: {\H שלום}, 
% and here's some Japanese: {\J 今日は}.



\end{document}  