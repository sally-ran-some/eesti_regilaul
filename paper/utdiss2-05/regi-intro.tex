\chapter{Introduction}
\index{Introduction@\emph{Introduction}}%is

\section{On words becoming lyrics}

This project takes lyrical folksongs as an opportunity to examine the interface of language and music. In particular, I investigate the temporal interactions between linguistic prominence (i.e., stress) and musical metre in Estonian lyrical folksongs known as {\it regilaul}. Taking each song as an experiment, I examine the results of the integration of linguistic and musical rhythm. 
%The prosodic or suprasegmental structure of a language is concerned with the level of sound features above that of the individual segment or phoneme. Features such a stress and tone are implemented at the word level to contribute to lexical meaning, and at higher levels such as clause, sentence, or paragraph. At the higher levels, prosodic features usually modify the meaning of the level in question (such as clause or sentence) rather than the meaning of individual lexical items contained within. 

A crucial component of a language's rhythmic system is the suprasegmental pattern by which stressed and unstressed elements are arranged. A stressed element in a language is a constituent that is perceived as having greater prominence, and produced with more emphasis relative to its surrounding constituents. 

Acoustically, the difference between a stressed and an unstressed syllable is found not in a single cue but in a convergence of several cues: often duration, increased variability in f0, and increased vowel space, corresponding to a notion of localized hyperarticulation\cite{smiljanicProductionPerceptionClear2005,de1995supraglottal}. 

In music, the metrical structure is the patterned alternation of strong and weak beats, forming a nested hierarchy of accent levels. The acoustic manifestations of musical accent are duration and intensity. 
\citep{palmerLinguisticProsodyMusical1992}. 

Estonian, has a fixed primary stress system, also has three degrees or weights of syllables, but these three must fit into the binary weight system of the musical metre. 

The rhythmic organization of song integrates the prosodic structure of the language with musical rhythmic principles \citep{palmerLinguisticProsodyMusical1992}. 

In this corpus phonetics study of song performances, I measure the vowel duration of isochronous syllable-notes. 


\


The literature on the relationship between linguistic rhythm and musical rhythm is vast. Some proposed systems of generative metrics have been applied to music, and some metrical phonology theories draw on constructs from music theories in their formalisms. \citep{palmerLinguisticProsodyMusical1992,lerdahl1983overview,liberman1977stress}. 

%Both song composition and performance are domains within which the relationship between linguistic and musical rhythm can be explored. Performance especially, as we have not only the intention of the composition and the composer's sensitivity to prosodic rules, but also physical evidence provided by a performance, allowing for direct observation of the result of the intended composition and the reality of the performance's interpretation.
%\citep{temperleyMusicLanguageCorrelationsScotch2011}.
%\citep{carpenterRhythmSpeechMusic2016}




Duration can be a phonetic correlate of stress, and it can also be independently contrastive at the segmental level \citep{lehistePhoneticsMetrics1992}. 


%Whether there is a correlation between poetic metre and the prosodic structure of a language.

%A trochee is a strong, stressed syllable followed by a short(er), weak syllable. 




\citep{lehistePhoneticsMetrics1992} investigated the actual phonetic realization of different metres in different languages, finding evidence for one language's trochee, for example, to be realized in a way that is systematically different from another language's trochee. 

%styles of versification
%
%cooperation between music and words

stressed syllable less reduced, more intelligible, 

Fine-grained acoustic-phonetic


We ask: what happens to a word, acoustically, when meaningful linguistic utterance, having its of grammar of rhythm, is inserted into the grammar of the metre of the song? 
In lyrical songs, the text setting as applied to the melody represents the collaboration of two grammatical domains: the song and the language. 

%And what are these rhythms?  In the abstract sense, the human intuition of rhythm is well known. Connecting the intuitive concept of rhythmic prominence with the physical acoustic events of a performance is the primary goal of this paper. 
%
%
The question at hand is whether or not the established durational correlate of stress and quantity in spoken Estonian are preserved in song. By examining the phonetic correlates of prominence in these songs, it is possible to uncover asymmetries in cue-weighting strategies and consequently rule ordering. 
%
