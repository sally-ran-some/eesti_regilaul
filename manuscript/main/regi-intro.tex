\chapter{Introduction}
\index{Introduction@\emph{Introduction}}%is

\section{On words becoming lyrics}

\subsection{song lyrics as the music-language interface}

In order to join music in song and become lyrics, language elements undergo modification to be fit into the musical structure. At the same time, the interpretability of those linguistic elements which constitute contrasts in semantic meanings has motivation for preservation. If the performer wishes the audience to perceive meaningful lyrical content, these contrasts must be crystallized at some level of salience in the acoustic signal. 

This project takes folksong performance as an opportunity to examine the interface of language and music. In particular, I investigate the acoustic manifestations of the word-prosodic features attested in spoken Estonian when they are sung in a temporally restricted domain such as music. 




\subsection{Estonian Phonetics}
The role of primary stress in Estonian is described by Ilse Lehiste as {\it identificational} rather than contrastive \citep{lehistePhoneticsMetrics1992}. In other words, there are no stress minimal pairs at the lexical level, so the prominence cue is to indicate the onset of a new word. This is sometimes also called {\it demarcative} stress.



Three proposed levels of stress: primary stress, unstress, and secondary stress. \citep{lippusAcousticStudyEstonian2014a}

Duration can be a phonetic correlate of stress, and it can also be independently contrastive at the segmental level \citep{lehistePhoneticsMetrics1992}. 
Primary lexical stress in native Estonian words is fixed, falling on word-initial syllables. 
\cite{eekmeisterUralica98}
\begin{exe}
\ex \gll laul-da \\
	{[ˈlɑuːl.dɑ]} \\
	sing-\Tr{} 
	\glt	`singing'
\ex 	ööbik \\
	{[ˈøː.pikː]} \\
	nightingale.\Nom{} 
	\glt`nightingale'
\end{exe}

Estonian has three syllable weights, also called degrees or quantities, that are contrastive in primary stress position. The first degree or Q1 is described as short, 
%In \ref{quant_cont}, we see two minimal pairs illustrating this contrast. 

This ternary contrast has long been the subject of debate in the phonological literature of metrics: Q3 syllables have been analyzed both as a monosyllabic foot \citep{princeMetricalTheoryEstonian1980} and as a trimoraic syllable \citep{hayesCompensatoryLengtheningMoraic1989, kuznetsovaEstonianWordProsody2018,prillopMoraeEstonianReply2020}. 


 \begin{table}[htb]
\centering
\begin{tabular}{lcc}
\hline

Q1 &		 sada 		& 	kabi  \\  
	&	 {\it `hundred'} 	&	 {\it`hoof' }\\
\hline
Q2 &		saada 		&	kapi \\
	&	 {\it`send' }		&	{\it`of the cupboard' }		\\
\hline
Q3 &		saada 	&	 kappi 	\\
	&	{\it`recieve' }	&	{\it`into the cupboard' }	\\
\hline
\end{tabular}
\label{qexamps}
\caption{ternary syllable weight contrast}
\end{table}



Acoustically, the difference between a stressed and an unstressed syllable is found not in a single cue but in a convergence of several cues: often duration, increased variability in f0, and increased vowel space, corresponding to a notion of localized hyperarticulation \citep{smiljanicProductionPerceptionClear2005, de1995supraglottal, lindblom1990}. 



\subsection{Metrical Principles of Estonian folksong}


In music, the smallest prosodic constituent is an individual note event whose relationship to the other notes in the song are indicated by the time signature, i.e., 3/4 or 4/4. The denominator corresponds to the number of divisions of a ``whole'' note, while the numerator refers to the number of ``beats'' in a single measure. So the prosodic heirarchy in music begins at the note level, then to each measure as constituent, larger phrases and motifs, and eventually the highest level which is the entire song. 

In Eesti, attested segments are organized into syllables, which in turn combine in strong-weak relationships to form prosodic feet, which make up the words of phrases and so on. 
both regilaul and Eesti have heirarchichal metrical systems by which long and short, stressed and unstressed constituents are arranged. 
 In a regilaul verse line, each syllable corresponds to an eighth note in the measure, with every odd positioned syllable-note coinciding with the ``beat'' onset in a typical 4/4 measure. Falling on the beat in music generally coincides with increased prominence compared to notes occurring ``off'' the beat. This is acoustically manifested in increased duration of notes on the beat compared to nominally isochronous notes in other positions. That is, in a measure containing eight eighth notes, the absolute durations of the four eighth notes corresponding to the beat will be longer than the absolute durations of the four eighth notes in even positions, while maintaining the perception of the nominal isochrony. 


How do the word-prosodic requirements negotiate with the imposed prosodic hierarchy of music? The rhythmic organization of song is said to integrate the prosodic structure of the language with musical rhythmic principles \citep{palmerLinguisticProsodyMusical1992}. However, earlier studies of {\it regilaul} explored temporal aspects of the songs and found that duration characteristics that would usually indicate important semantic differences lost their distinctions partially or entirely. Assuming that the intention of the singer is for the lyrics to be understood, I hypothesize that if some durational correlates of contrasting word-prosodic constituents are made less distinct in the process of compromising with the song that some other acoustic correlate of the relevant contrast at the word-prosodic level will be present, if not enhanced.  


\begin{figure}[htbp]
\begin{center}
\includegraphics[width=300pt]{figures/094.png}
\caption{music notation of ``The King Game" as performed by Liisa Kümmel}
\label{The King Game}
\end{center}
\end{figure}

%\section{The Runic Song Tradition} 
%
%
%A given {\it regilaul} verse line contains eight beats, generally evenly divided into a single measure so that each of the eight syllables corresponds with one eighth note. The ictus position of a song's line is thus determinable by counting every other beat as ictus, starting with the first.
%
%The first song festival in Estonia held in \cite{ruutelTRADITIONALMUSICESTONIA2004}. 
%
%
%One runic invariant is that the tonal center is placed on ``the both syntactically stable 
%
%
%dominating pitch value of the most stable syntactical positions of a given melody's typological group
%
%
%the syllable-note is the basic unit of the melody-line. Thus syllables take up the space of the note in their position of the melody as prescribed. 



%
\subsection{The singing giant: a common folklore epic} 
%Estonian, Finnish, Karelian, Ingrian runic songs
%
%whether it be Finnish, Estonian, Karelian, Votic, Ingrian and to same extent even Livonian an Veps)
%
%
the singable song \cite{tormisKalevalaEstonianPerspective1985}
%
%
% ancient epic tale shared by many members of the Finno-Ugric language family
% 
% The common folklore of Finno Ugric language family: Kalavala in Finland, Kalevipoeg (Kalev's son) in Estonia. Kalev is a giant.  
 
 
 \cite{tormisProblemsThatRegilaul2007}






 
\section{Previous Studies with regilaul}



 
Melodic accents coincide with the stressed syllables, the pitch resolves as the phrase resolves. Said to be a musical abstraction of the natural prosodic intonation of the 8 syllable (spoken) runoverse \cite{ruutelResultsComputerizedComparative1999}.


\begin{figure}[htbp]
\begin{center}
\includegraphics[width=300pt]{figures/069.png}
\caption{default}
\label{default}
\end{center}
\end{figure}



The question of the acoustic dimensions of Estonian word prosody and the metrical patterns in traditional lyrical folksongs is not a new one. In 1990, an Estonian musicologist measured syllable duration and vowel quality in a single {\it regilaul} song, publishing a paper on each measurement. 
same song, timing and quality measured
In \citep{rossFormants90}, Ross measured formant frequencies f1 and f2 at eighth note positions in \citep{thesong}, finding a reduced vowel space in song compared to measurements in spoken Estonian. However, upon examination of the song, it is clear that all the vowel space measurements are from syllable-notes in non-initial positions of Estonian words: that is, the sample of vowels taken from  the song were all unstressed, and compared to a mixed sample of spoken Estonian. Thus, the conclusion needs to be evaluated again with comparable samples. 
\citep{rossStudyTimingEstonian1989a} \\
%
In 1992, Ilse Lehiste asked "Whether there is a correlation between poetic metre and the prosodic structure of a language.\citep{lehistePhoneticsMetrics1992} by means of measuring the acoustic-phonetic realizations of so-called trochaic metrical poetic patterns across several languages including Estonian and Finnish. While these languages share in the more general Balto-Finnic tradition of {\it runosong} utlizing what is called the {\it Kalevala} meter, there were significant differences in the phonetic realizations of trochees in each language. 


In 1994, their collaboration begins. Ross and Lehiste published several papers examining the temporal dimensions of Estonian word prosody and metrical prominence in {\it regilaul} folksongs. In \citep{rossLostProsodicOppositions1994}, they conclude that duration differences ordinarily present at the word level (stressed-unstressed) are {\it ``lost"} to the temporal restrictions of the song. In another paper, syllable-notes are again measured, this time examining the role of syllabic quantity in the song. They likewise conclude that the duration of syllable-notes in {\it regilaul} match more closely with the metrical structure of the songs, that is, the durations of syllable-notes are best predicted by their beat position in the song: on the beat, syllables are longer, off the beat shorter. They extend this finding to conclude that the song "dominates" the metrical status of the words, claiming that the intelligibility of lyrics is enhanced more strongly by ``top-down" processes (i.e., semantic context). 

\citep{rossTradeoffQuantityStress1996}. \\

%to paraphrase: Semantically relevant oppositions of word initial short-long and long-short disyllabic units in speech are not kept completely intact in folk songs. Short-long disyllables are treated in a different manner by the performer, depending on whether their initial syllable occurs at a rise or at a fall in a foot
\citep{rossTimingEstonianFolk1998} Analyzed and concluded that the {\it regilaul} lyrics are the result of an interaction between word and song prosodic hierarchies. This conclusion relied critically on measuring the durations of syllable-notes, where they found being on or off the beat was the better predictor for duration. \\


Finally, in %\citep{book_jril_2001} summarize their body of work until that point and extend their findings with fine-grained acoustic phonetic measurements of segment durations within syllable-notes. 

\section{The present study}

I do not dispute the findings that syllable-notes are best predicted by their position in the song. Instead, I argue that instead of an hierarchical interaction between language and music, what occurs is more akin to entrainment. Specifically, bidirectional entrainment of two oscillators. 

The syllable-notes, fit into the temporal structure of the music, still have possible acoustic-phonetic dimensions by which to indicate word boundaries (stressed-unstressed) and semantically-relevant quantity contrasts.  The durational differences of stress and quantity in spoken Estonian could be indicated either at another level of the prosodic hierarchy, i.e.,  the segment, or by enhancing another acoustic cue for prominence: i.e.,  vowel space dispersion. 


An EEG study of native Estonian speakers uncovered a perceptual asymmetry: shortening of constituents did not inducing MNN, but lengthening does \citep{eestiMNNasymmetry}. This suggests that stressed syllables, long in spoken Estonian, could be realized as shorter in a song without sounding ``off," but that lengthening of shorter elements would be less preferable. So long as syllable-notes maintain the relative durations in the song (i.e., heavy syllables are still longer than light ones), the metrical pattern of the musical phrase can be met. 

\subsection{Hypotheses}
Null hypothesis: on or off-beat position is best predictor for vowel duration of syllables in all categories: stressed, unstressed, short, long, and overlong. This would mean that the syllable nucleus is proportional to the total syllable duration, and that the duration contrast is indeed ``lost." 

H_{1}: duration contrasts for syllable quantity will be evident in the vowel duration, with vowel duration {\it decreasing} as syllable weight increases. Given the findings of \citep{rosslehiste}, isochronous syllable-notes would result in heavier syllables having shorter nuclei to accomodate for the coda and complex codas that distinguish the syllable weights from each other. 

H_{2}: stress/unstress contrasts will be evident in the nucleus in terms of hypo and hyper articulation. For this I measure both nucleus duration and vowel space dispersion. 

%\cite{rossStudyTimingEstonian1989,rossLostProsodicOppositions1994,rossTradeoffQuantityStress1996,sargDoesMelodicAccent2006, sargMelodicAccentEstonian2007,orasMusicalManifestationsTextual2011}
