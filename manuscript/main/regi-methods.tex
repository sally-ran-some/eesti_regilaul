\chapter{Methods}
\index{Methods@\emph{Methods}}%

\section{Design}
To test these hypotheses, vowel duration from the nuclei of first (primary stressed) and unstressed syllables of disyllabic feet. \footnote{The exclusion of secondary stress is due to recent findings that secondary stress in spoken Estonian does not exhibit significant differences compared to unstressed syllables of polysyllabic words besides the pen-initial \cite{asuLippus2018}. }

Within each song, samples of syllable-notes consisted only of those matching the nominal isochrony. For example, in verse lines mostly comprised of syllable-notes divided evenly into eighths, neither quarter notes nor sixteenth notes were taken. Syllables in phrase-final position were generally excluded under this criteria. 

% and location on the F1,F2 plane are taken 

\section{Constructing the Corpus}

I first describe the source materials and the selection criteria for the sample corpus of {\it regilaul} folksongs. Following this, the annotation and measurement procedure is detailed. Then the procedure for assembling the corpus of songs and their text annotations is covered before proceeding to the inclusion criteria for vowel duration and dispersion measurements.

\subsection{Materials}

Songs for this paper were accessed via The Anthology of Estonian Traditional Music \citep{tampere2016}. Originally published on four vinyl discs in 1970, the digital version showcases a robust sample of the massive collection of {\it regilaul} in Estonian Folklore Archives. In addition to audio, the  compilation includes  photographs, sheet music, and performer demographics of 98 {\it regilaul} songs and 17 instrumental tunes. 

For this paper, I took a subset of the available songs based on shared characteristics such as region, time of field recording, and ethnomusicologists.  Once several regions were identified as possible candidates, a native Estonian speaker was consulted on the final selection. The nine songs analyzed in this study were all recorded in Parnümaa county from 1961-1966 by Herbert Tampere, Erna Tampere, and Ottilie Kõiva for the Estonian Folklore Archives\citep{oras2002, tampere2016}. 



\subsection{Annotating the Song Audio }

 Each song's lyrics are copied from the site and saved as .txt files in Estonian orthography, each line of the file corresponding to one melody line.  
Audio files of the selected songs are downloaded from the archive in .ogg format, which is the highest resolution of the two lossy formats available from the digital anthology. Each song is then imported into a Logic Pro X \citep{logic2014} session for beat detection, tempo mapping, and trimming. 
To make the tempo map, the session is set to {\it flex tempo}. From here a beat onset detection algorithm is  given the estimated bpm and time signature as a starting point and then run on the song audio. 

From here, a MIDI track is programmed to create a metronome specifically tailored to the song according to the tempo map. Using the aforementioned {\it flex tempo}, the MIDI track adjusts note and measure length to match the fluctuations documented in the by the beat tracker. Once rendered, the metronome and the song audio file are trimmed to match exactly, and the metronome is converted with a script into a textgrid of beat and measure intervals in PRAAT\citep{boersna2022}\footnote{Using onset detection algorithms such as these \citep{bKeeper2007} in phonetics research, especially in the interdisciplinary field of linguistics and musicology, will be particularly beneficial to answering questions about rhythm: finding a way to bring our intuitions and impressions about ``the beat" together with the acoustic phenomenon. By automating the annotation and measurement process using open source tools, the author hopes to share these machines with those who have similar research interests, and also to invite contributors to the data of this corpus of text data time-aligned to queryable audio signal data.}. 

 
The texts of verse lines typeset in Estonian orthography are inserted into each verse-phrase interval (already annotated thanks to the metronome). On each verse line, the eSpeak forced aligner for Estonian \citep{eSpeak1995} to reverse-synthesize from the phrase input to the word and phonemic level. Because this forced aligner is trained on spoken, not sung Estonian, manual verification and realignment of the vowel segments used in this dataset was often necessary. \\

\subsection{Adjustment Criteria for Vowel Durations}
The beginning of the vowel was aligned according to a combination of acoustic correlates: 



\begin{itemize}
\item vowel considerations
\begin{itemize}
\item {\bf intensity(dB) contours}: intensity within 2dB of the steady-state medial portion of the vowel, with a positive slope less than one.
\item {\bf fundamental frequency(Hz)} stabilizing into that syllable-note's pitch category
\item {\bf resonant frequencies F1, F2, and F3}: visible and approaching steady-state, combined wtih  the presence of a voicing bar.
\end{itemize}
\item onset manner particulars: 
\begin{itemize}
\item following {\bf plosive onsets}, vowel boundary is after a burst
\item following {\bf fricative onsets},  the end of visible high-frequency noise in the spectrum was reliable. The phoneme /s/ also consistently showed a carat in the frequency track immediately preceding the transition to vowel.
\item boundaries between {\bf approximant onsets} and vowel nuclei were determined chiefly by the steady state of formants. 
\item following {\bf nasal onsets}, vowel intensity actually lowered, but a negative slope approaching zero reliably followed the offset of visible anti-formants.
\end{itemize} 
\item The offsets of vowels into codas:
\begin{itemize}
\item negative slopes approaching zero for intensity; positive preceding nasal. 
\item Formants were allowed more variation in transitions to codas
\item approximant codas {[l, j, w]} were excluded from the dataset, as neither the forced-aligner nor the phonetician could determine a reliable way to define the boundary between them.
\end{itemize} 
\end{itemize}



In cases of vowel adjacency across syllable boundaries, the presence of a visible glottal stop and support from other criteria would qualify both for inclusion. In the absence of these cues, both nuclei were excluded from the measurements. Other exclusions were due to ambient noise (i.e., churchbells in song 41), ambiguity of word boundaries due to wordplay (consulted with native speaker), and cases of adjacent identical vowels across word boundaries. 
%epenthesized vowels (i.e., pandi mind paju raiumaie) having mind(e) \\


In all cases, if the aforementioned cues were unavailable, ambiguous, or misaligned, the token was elided for this analysis. From all nine songs in the corpus, a total of 757 vowel nuclei met the criteria for inclusion in duration measurements. 

%For the measurements of F1, F2 space, formants were extracted from the mid point of the vowel. 


At this point, the audio recording of each song has corresponding textgrid tiers annotated for beat, verse lines, and force-aligned word and phoneme levels.
\subsection{Connecting acoustic measurements to text corpus}

The last step in preprocessing is to integrate the annotation of the song audio with the lexical content of the song. This study accomplishes the task using an open-source natural language toolkit in python called estinltk \url{https://github.com/estnltk} \citep{estnltk2020}. Among other things, the toolkit has a robust dictionary of Estonian grammar, including phonetic transcription of syllables with corresponding quantity and stress data. This is especially important data for testing hypotheses about the ternary quantity contrast, which is not always apparent from the orthography. 
%Linguistic descriptions of the Estonian language date back as early as the seventeenth century, but the ternary quantity contrast was not documented until native Estonian linguists contributed their intuitions. The non-native linguists had only described lexical stress \citep{sargEarlyHistoryEstonian2005a}. \\



Once the annotations are complete, python scripts are used aggregate the audio and text that form this microcorpus: text files of lyrics are connected to the textgrid files corresponding to the audio, and corresponding duration measurements are extracted from PRAAT using the parselmouth library\citep{parselmouth2018, python1995}. The process of aggregating the acoustical and textual data for this corpus is available in the two jupyter notebooks in folder preprocess_jupyter at \url{https://github.com/sally-ran-some/eesti_regilaul} and also viewable in the appendices. 




\section{Statistical Analysis} 
For each hypothesis, linear mixed-effects models were fit using the lme4 package in R \citep{lme4,r2022}. Due to the asymmetry of distribution of the three quantities in Estonian (Q3 is never unstressed), the analysis of vowel duration is broken into two. The first analysis concerns the semantically-relevant lexical contrasts of the three Estonian quantities entrained to the binary long-short pattern of the musical meter. A subset of only primary stressed syllables is used to build this model. 

\lipsum[2-3]


To examine the effects of word stress on vowel duration, a subset of data excluding Q3 syllables was taken. While retaining Q1 and Q2, the interaction of syllable quantity and word stress will be apparent, as both Q1 and Q2 fall in stressed and unstressed positions. 

\lipsum[2-3]

%For models with duration as dependent variable, random intercepts were set for individual song, singer, and word. For HQ, fixed effects were syllable quantity, beat position (ictus), and the interaction of those terms. For HA, the fixed effects were ictus, stress, and the interaction of ictus and stress. 

%For the model using vowel dispersion (HS) as the predictor, random intercepts were set for performer, word, and segment quality. Fixed effects were ictus, stress, and the interaction of ictus and stress. 
