\chapter{Discussion}
\index{Discussion@\emph{Discussion}}%

\section{Temporal Prosodic Features Crystalized at Segmental level in isochronous syllables}

These results support the hypothesis that prosodic timing modifications resulting from the synchronization of independent rhythms (in this case, syllables and notes into syllable-notes) do not result in the subordination of one system to another. In this case, when syllables and notes become one timing unit, the temporal acoustic correlates often found at suprasegmental levels (syllable, foot, word) are found at the segmental levels. 

This interaction is then more analagous to entrainment of two independent rhythmic systems than to language being forcibly pigeon-holed into the metrical structure of music. 


%\section{Vowel Dispersion}
%Results for the predictive power of ictus and off-ictus, stressed and unstressed syllables and vowel space dispersion were not significant. The patterns that seem to emerge visually in the graphs of distributions suggest that the situation might be ameliorated by increasing the dataset. For this measurement, there were fewer available tokens than for vowel duration, and therefore less statistical power. Further examination is needed to determine whether or not the observed pattern is simply lacking in power or whether the premises that lead to this hypothesis are missing something entirely. 


\section{Future Studies} 

This study used a sample of nine songs and three singers, all recorded in the 1960s. Annotation has already begun on the remaining songs that fit into this sample's criteria: In total, there are seventeen songs and seven singers from Parnumaa county recorded in the 1960s.
% Increasing the size of the audio corpus would facilitate exploratory analysis in countless dimensions of music and language, and also shed light on the issue of vowel space unresolved here. 
Several of the singers featured in this sample set were also recorded speaking. Annotating their natural speech would provide a valuable contribution to the song corpus, as findings from the songs could be compared with speech of the same person.  

Extending the findings of vowel duration and the ternary quantity contrast has several obvious paths: synchronic analysis of with song samples from the same approximate time period but differing according to region, or even language: several other Balto-Finnic families have a trochaic tetrameter folksong tradition. Diachronic analysis with song samples of same singers in the same region at different points in time is another possibility with data from the Estonian Folklore Archives. Both these goals are achievable only with the continued annotation of the corpus of regilaul, which is quite demanding work. As I continue to build this corpus, I am also actively exploring ways in which to automate the process. The inclusion of beat tracking software eliminated much observer subjectivity, and also facilitated the forced-aligner: by automatically grouping verse lines into measures, the aligner was given phrase groupings to synthesize and compare, rather than attempting to align the entire song in a linear fashion. However, as the forced aligner used here was made specifically for speech, one way to improve the accuracy of the forced aligner (decreasing manual adjustment of annotations) would be to train a the aligner on sung material using supervised machine learning. As I plan to continue with the annotations either way, I can use the corrections I make as training material for the algorithm, with the hopeful result that the forced-aligner will eventually reach some threshold of accuracy, voiding the need for manual adjustments. 

\section{ML plans outline}
%Kaldi: uses DNN (Deep Neural Net) rather than linear model like MFA (montreal forced-aligner). 

%https://www.kaldi-asr.org/doc/
%https://www.eleanorchodroff.com/tutorial/kaldi/index.html

for several/general uralic languages \citep{leinonen2021}. 
Would be good to try for other runosongs! 

%https://github.com/lingsoft/aalto-kaldi-align-elg

for Estonian specifically \citep{ratsep2022}
\section{Conclusion}

This study examined fine-grained acoustic-phonetic features of Estonian prosody in the context of traditional folksongs known as regilaul. The data support the hypothesis that duration contrasts inherent in the ternary quantities of Estonian are still present at the segmental level, even after undergoing modification to fit syllables into isochronous notes of the song. The data also supports that the durational correlates of word-level stress are also crystallized and evident at the segmental level, so while it isn't lexically contrastive, the role of primary stress and unstress to mark word boundaries in spoken Estonian is still present in sung Estonian. 
%Vowel quality patterns were measured but not significant for this dataset. 

This indicates a relationship more akin to the collaboration of two independent rhythmic systems rather than one rhythmic system dominating the other. Combined with the fact that any regilaul text can be sung to any existing regilaul melody brings into question whether {\it spoken} metrical verse text is truly independent of the temporal constraints of the musical meter they are made with, or somewhere between language and music. 
