\chapter{Introduction}
\index{Introduction@\emph{Introduction}}%is

\section{On words becoming lyrics}
\subsection{song lyrics as the music-language interface}

In order to join music in song and become lyrics, language elements undergo modification to be fit into the musical structure. At the same time, the interpretability of those linguistic elements which constitute contrasts in semantic meanings has motivation for preservation. If the performer wishes the audience to perceive meaningful lyrical content, these contrasts must be crystallized at some level of salience in the acoustic signal. 

This project takes folksong performance as an opportunity to examine the interface of language and music. In particular, I investigate the acoustic manifestations of the word-prosodic features attested in spoken Estonian when they are sung in a temporally restricted domain such as music. 


both regilaul and Eesti have heirarchichal metrical systems by which long and short, stressed and unstressed constituents are arranged. In music, the smallest prosodic constituent is an individual note event whose relationship to the other notes in the song are indicated by the time signature, i.e., 3/4 or 4/4. The denominator corresponds to the number of divisions of a ``whole'' note, while the numerator refers to the number of ``beats'' in a single measure. So the prosodic heirarchy in music begins at the note level, then to each measure as constituent, larger phrases and motifs, and eventually the highest level which is the entire song. In Eesti, attested segments are organized into syllables, which in turn combine in strong-weak relationships to form prosodic feet, which make up the words of phrases and so on. 



 In a regilaul verse line, each syllable corresponds to an eighth note in the measure, with every odd positioned syllable-note coinciding with the ``beat'' onset in a typical 4/4 measure. Falling on the beat in music generally coincides with increased prominence compared to notes occurring ``off'' the beat. This is acoustically manifested in increased duration of notes on the beat compared to nominally isochronous notes in other positions. That is, in a measure containing eight eighth notes, the absolute durations of the four eighth notes corresponding to the beat will be longer than the absolute durations of the four eighth notes in even positions, while maintaining the perception of the nominal isochrony. 



%Acoustically, the difference between a stressed and an unstressed syllable is found not in a single cue but in a convergence of several cues: often duration, increased variability in f0, and increased vowel space, corresponding to a notion of localized hyperarticulation \citep{smiljanicProductionPerceptionClear2005, de1995supraglottal, lindblom1990}. 

How do the word-prosodic requirements negotiate with the imposed prosodic heirarchy of music? The rhythmic organization of song is said to integrate the prosodic structure of the language with musical rhythmic principles \citep{palmerLinguisticProsodyMusical1992}. However, earlier studies of {\it regilaul} explored temporal aspects of the songs and found that duration characteristics that would usually indicate important semantic differences lost their distinctions partially or entirely. Assuming that the intention of the singer is for the lyrics to be understood, I hypothesize that if some durational correlates of contrasting word-prosodic constituents are made less distinct in the process of compromising with the song that some other acoustic correlate of the relevant contrast at the word-prosodic level will be present, if not enhanced.  


%
%Whether there is a correlation between poetic metre and the prosodic structure of a language.\citep{lehistePhoneticsMetrics1992} investigated the actual phonetic realization of different metres in different languages, finding evidence 






%to paraphrase: Semantically relevant oppositions of word initial short-long and long-short disyllabic units in speech are not kept completely intact in folk songs. Short-long disyllables are treated in a different manner by the performer, depending on whether their initial syllable occurs at a rise or at a fall in a foot
\citep{rossTimingEstonianFolk1998} Analyzed and concluded that the {\it regilaul} lyrics are the result of an interaction between word and song prosodic heirarchies. This conclusion relied critically on measuring the durations of syllable-notes, where they found being on or off the beat was the better predictor for duration. \\


trade-off Q for S? 
\citep{rossTradeoffQuantityStress1996}. \\
They conclude that duration contrasts ordinarily present at the word level are {\it ``lost"} \citep{rossLostProsodicOppositions1994} to the temporal restrictions of the song. 

same song, timing and quality measured
\citep{rossFormants90}
\citep{rossStudyTimingEstonian1989a} \\

I argue that the cooperation of language in music is more than an interaction: that is, I argue that lyrical folksongs are examples of bidirectional {\it entrainment} of two independed oscillatory processes. I do not question the findings of Ross \& Lehiste, but wish to extend them. If the the syllable is required to fit in the prosodic level of the song, the linguistic contrasts of the words could be present either at a lower level of the prosodic heirarchy, i.e.,  the segment, or the contrast could be preserved in another acoustic dimension: i.e.,  vowel space dispersion. 


Null hypothesis: on or off-beat position is best predictor for vowel duration of syllables in all categories: stressed, unstressed, short, long, and overlong. This would mean that the syllable nucleus is proportional to the total syllable duration, and that the duration contrast is indeed ``lost." 

H_{1}: duration contrasts for syllable quantity will be evident in the vowel duration, with vowel duration {\it decreasing} as syllable weight increases. Given the findings of \citep{rosslehiste}, isochronous syllable-notes would result in heavier syllables having shorter nuclei to accomodate for the coda and complex codas that distinguish the syllable weights from each other. 

H_{2}: stress/unstress contrasts will be evident in the nucleus in terms of hypo and hyper articulation. For this I measure both nucleus duration and vowel space dispersion. 

Perceptual asymmetry: shortening not inducing MNN, but lengthening does \citep{eestiMNNasymmetry}. I expect that while syllable-notes that are on the beat will be prominent compared to those off the beat, unstressed syllables will be reduced compared to stressed syllables in the same position. 
