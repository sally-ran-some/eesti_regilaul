\chapter{Introduction}
\index{Introduction@\emph{Introduction}}

Lyrical songs are the result of bidirectional entrainment of language and music.

%Perceptual asymmetry: shortening not inducing MNN, but lengthening does \cite{eestiMNNasymmetry}. I expect that while syllable-notes that are on the beat will be prominent compared to those off the beat, unstressed syllables will be reduced compared to stressed syllables in the same position. 
%
%Will the lyrics prefer lengthening or shortening of vowel nuclei to preserve the contrast?



Estonian is useful because it has both fixed word-level stress and three quantity degrees. \cite{thatone} Analyzed and concluded that the {\it regilaul} lyrics are the result of an interaction between word and song prosodic heirarchies. This conclusion relied critically on measuring the durations of syllable-notes, where they found that song position (on or off the beat) was the better predictor for duration. They conclude that duration contrasts ordinarily present at the word level are {\it ``lost"} \cite{rosslostprosodic}   to the temporal restrictions of the song. 

I argue that the cooperation of language in music is more than an interaction: that is, I argue that lyrical folksongs are examples of bidirectional {\it entrainment} of two independed oscillatory processes. I do not question the findings of Ross \& Lehiste, but wish to extend them. If the the syllable is required to fit in the prosodic level of the song, the linguistic contrasts of the words could be present either at a lower level of the prosodic heirarchy, i.e.,  the segment, or the contrast could be preserved in another acoustic dimension: i.e.,  vowel space dispersion. 


In order to test the hypothesis of bidirectional entrainment, I look at the syllable nucleus. 

Null hypothesis: on or off-beat position is best predictor for vowel duration of syllables in all categories: stressed, unstressed, short, long, and overlong. This would mean that the syllable nucleus is proportional to the total syllable duration, and that the duration contrast is indeed ``lost." 

H_{1}: duration contrasts for syllable quantity will be evident in the vowel duration, with vowel duration {\it decreasing} as syllable weight increases. Given the findings of \cite{rosslehiste}, isochronous syllable-notes would result in heavier syllables having shorter nuclei to accomodate for the coda and complex codas that distinguish the syllable weights from each other. 

H_{2}: stress/unstress contrasts will be evident in the nucleus in terms of hypo and hyper articulation. For this I measure both nucleus duration and vowel space dispersion. 


Since \cite{r and l} found that song is the best predictor for syllable-note duration, I expect that the song is the higher level of the prosodic heirarchy. 

Entrainment: 
In order for entrainment to happen, two oscillators must be able to act independently of each other: resonance, while it is oscillatory, is not an example of entrainment. If a tuning fork is removed from a box, the box no longer oscillates. Both language and music meet the criteria for independent oscillation: independently of a song, the linguistic constituents have periodicity, relative prominence, heirarchy. Likewise, music without any lyrics at all has periodicity and heirarchy. 

unidirectional: If R \& L's findings extend to the segmental level, then the song is oscillating within the shape of the words of the language, and rhythmic contrasts from the linguistic/word level rely heavily on ``top down" processing, i.e., the semantic context of the verses in the song. 

bidirectional: If R \& L's findings do not extend below the syllabic level: that is, if prominence contrasts are preserved at a level below the syllable-note, then this would be bidirectional entrainment, as the two oscillators are operating independently. (complex wave metaphor). 

