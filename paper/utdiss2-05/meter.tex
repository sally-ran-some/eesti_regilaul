\chapter{The Rhythmic Domain}
\index{The Rhythmic Domain@\emph{The Rhythmic Domain}}
I begin with a short discussion on the four dimensions necessary to examine the notion of rhythm and define the concept for the sake of this paper. This is especially necessary because our psychological ``right now" is only a few seconds, but with a song we are able to store much longer spans of time as complete entities with structured constituents. 

Oscillations!

what we perceive as tone is very fast rhythm
what we term rhythm is very slow tone

Both the physical events, i.e., the periodic frequency of a signal x in a specified time, and the psychological events, i.e., the pitch as perceived by auditors. Since we know that pitch is not always the same as frequency, it follows that this applies to lower frequencies as well: i.e., beats. 

The psychological event data give us access to the underlying cognitive structure of the rhythm user. The acoustic signal details the manifestations of a singer or talker's temporal organization, which cues they are using to differentiate constituents of these structures. 

A rhythm's accent structure functions in songs is to indicate the the position within a given constituent. It can be manifested acoustically in a variety of ways: lengthening of duration, faster attack, increased force. 


It seems that many theories of metrical prominence were imagined as the wave forms of pure tones, when the reality is much more complex. While there appears to be an available binary categorization level for perceived oscillatory magnitudes, i.e., Strong/weak note pairs, at least four dimensions are necessary to capture the relevant structural relationships of spatial-temporal elements of a song as a constituent. 

\section{Acoustic Dimensions}



First, physical dimensions of measurements


The position of air particles in time

\subsection{Dimension One: Particles as Points}
An air particle is a single point
a speck of dust
Spectral Slice for x moment in time
x at time m
\subsection{Dimension Two: Point Locations}
Frequency = how fast particles move
move at different rates, but at x time, how were the points dispersed? What is their density, what is their shape (resonant frequencies) 
x,y at time m 
\subsection{Dimension Three: Concentration of Points}
euc((x,y), (x2,y2)) 
\subsection{Dimension Four: Points in Motion} 

duration of constituent category 


\section{Perceptual Dimensions}

\subsection{Linguistic Audition}
\subsection{Musical Audition} 

falsely hear ``beat" wherever told, even especially isochronous equitone sequences
MNN studies

asymmetry: short-long easier than long-short
shortening not induce MNN, only lengthening 
underspecified for the length of short(er) constituents so long as the relative different-ness is preserved. 