\chapter{Handout}


The rhythmic organization of song integrates the prosodic structure of the language with musical rhythmic principles \citep{palmerLinguisticProsodyMusical1992}. 

Duration can be a phonetic correlate of stress, and it can also be independently contrastive at the segmental level \citep{lehistePhoneticsMetrics1992}. 


%Whether there is a correlation between poetic metre and the prosodic structure of a language.

%A trochee is a strong, stressed syllable followed by a short(er), weak syllable. 




\citep{lehistePhoneticsMetrics1992} investigated the actual phonetic realization of different metres in different languages, finding evidence for one language's trochee, for example, to be realized in a way that is systematically different from another language's trochee. 

Ross found vowel reduction in certain notes, but they were all unstressed, off-ictus for that song\citep{rossFormants90}.


The role of primary stress in Estonian is described by Ilse Lehiste as {\it identificational} rather than contrastive \citep{lehistePhoneticsMetrics1992}. In other words, there are no stress minimal pairs at the lexical level, so the prominence cue is to indicate the onset of a new word. This is sometimes also called {\it demarcative} stress.



Three proposed levels of stress: primary stress, unstress, and secondary stress. \citep{lippusAcousticStudyEstonian2014a}

Primary lexical stress in native Estonian words is fixed, falling on word-initial syllables. 
\cite{eekmeisterUralica98}
\begin{exe}
\ex \gll laul-da \\
	{[ˈlɑuːl.dɑ]} \\
	sing-\Tr{} 
	\glt	`singing'
\ex 	ööbik \\
	{[ˈøː.pikː]} \\
	nightingale.\Nom{} 
	\glt`nightingale'
\end{exe}

Estonian has three syllable weights, also called degrees or quantities, that are contrastive in primary stress position. The first degree or Q1 is described as short, 
%In \ref{quant_cont}, we see two minimal pairs illustrating this contrast. 

This ternary contrast has long been the subject of debate in the phonological literature of metrics: Q3 syllables have been analyzed both as a monosyllabic foot \citep{princeMetricalTheoryEstonian1980} and as a trimoraic syllable \citep{hayesCompensatoryLengtheningMoraic1989, kuznetsovaEstonianWordProsody2018,prillopMoraeEstonianReply2020}. 


 \begin{table}[htb]
\centering
\begin{tabular}{lcc}
\hline

Q1 &		 sada 		& 	kabi  \\  
	&	 {\it `hundred'} 	&	 {\it`hoof' }\\
\hline
Q2 &		saada 		&	kapi \\
	&	 {\it`send' }		&	{\it`of the cupboard' }		\\
\hline
Q3 &		saada 	&	 kappi 	\\
	&	{\it`recieve' }	&	{\it`into the cupboard' }	\\
\hline
\end{tabular}
\label{qexamps}
\caption{ternary syllable weight contrast}
\end{table}
