\chapter{Discussion}
\index{Discussion@\emph{Discussion}}%

\section{Prosodic Features Crystalized at Segmental level in isochronous syllables}

\section{Future Studies} 


%\subsection{long vowels and monophthongs}
%An estonian word game supports the notion of many-to-one phonemes for long vowels, but a one-to-one status for diphthongs.
%
%\begin{exe}
%
%\ex  long vowels (Q1,Q2, Q3 contrast) 
%	\begin{itemize} 
%	\item sada `hundred' $\rightarrow$ sapida 
%	\item  saada `send' (2nd sg. imp.) $\rightarrow$ sapiida (*sapiada)(*saapida)
%	\item saada `get' , {\it -da} infinitive $\rightarrow$ sapii:da (*sapiada)(*saapida)
%	\end{itemize}
%\ex 	 diphthongs \\
%laulus `in the song' (iness.sg) $\rightarrow$ lapiulus (*laupilus)
%\end{exe}
%
%In both cases, `pi' carries the prosodic characteristics of the displaced syllable. the stress is moved from first syllable to 'pi,' and also the length. crucially, the diphthong is split at the segmental level, while the long vowels are not.  Long vowels, then, are treated as monophonematic, whereas diphthongs are treated as biphonematic. 
%
%This word game supports the notion that long (identical) vowels and diphthongs in  Estonian should potentially be treated as different categories for duration. The analysis therefore will examine long monophthongs and diphthongs both together and separately for interpreting  the measurement of vowel duration. 
%
%\cite{lehiste1985}

While the ultimate goal is to continue annotation of the entire available corpus of {\it regilaul}, for the



In addition to regilaul, the archive maintains collections of other folkloric primary and secondary materials, as well as folklore and folk music traditions of other Balto-Finnic peoples: the Ingrian Finns, Votians, Karelians, Livonians and Izhorians. Collections also include materials from Ob-Ugrians: Khanty, Mansi, Udmurtian, Mari, and Hungarian, as well as folklore of nearby non-Finno-Ugric language users: Russian inhabitants, Estonian Swedes, Ngasans, and Latvians. 
The total sound archive of the EFA has over 100,000 individual recording pieces. Over 26,000 of these recordings are songs, dating from the early 20th century until the 1970s. In the database, regilaul songs are classified according to function, e.g., work songs, calendar songs, wedding songs, and then by musical characteristics. Newer ``end-rhymed" Estonian folksongs are arranged by theme and then by texts. Finally, instrumental folk music is placed into categories dance and other 
%\subsection{exploratory computation analysis comparing acoustic measurements to text constituents}
%
%The annotation possibilities with this toolkit are vast, especially with respect to morphosyntax. 
\subsection{Comparing with Speech} 

Several of the songs used in this study have accompanying recordings of natural running speech of the respective singers: this was one motivation in choosing the recordings from the 1960s rather than earlier. These results could be further supported by an acoustic analysis of the speech of the performers, confirming, for example, that quantity is a predictor for vowel length in their natural speech. 

%\subsection{Secondary Stress}
%
%Secondary stress
%Typically coincides with odd numbered syllables
%Secondary stress does not differ from un stress! 
%
% (Asu and Lippus 2018)
%\citep{asuAcousticCorrelatesSecondary2018}
%
%\subsection{Spectral Cues}
%
%An obvious next domain of exploratory analysis is vowel space and how it interacts with ictus and stress. 

%One runic invariant is that the tonal center is placed on ``the both syntactically stable 
%dominating pitch value of the most stable syntactical positions of a given melody's typological group
%
%Melodic accents coincide with the stressed syllables, and  the tune resolves as the phrase resolves. It has been described as a musical abstraction of the natural prosodic intonation of the spoken runoverse, which also utlizes the Kalevala metre% \citep{ruutelResultsComputerizedComparative1999}.
%
%
%more recently, melodic accent
%\citep{sargDoesMelodicAccent2006} 
%\citep{sargMelodicAccentEstonian2007}



%The results confirm the findings of Ross \& Lehiste's studies on timing in Estonian folk songs, while illuminating potential signatures of prominence contrast preservation when duration is lost to the song. Vowel duration usually increases with syllable prominence at the word level but, in sung verse, decreases as syllable weight or prominence increases. In order to confirm foot and note isochrony, the next logical step is to measure whole syllables and words and their relative proportions. Measuring onset consonant duration will potentially illuminate the vowel shortening in stressed syllables. 

\subsection{metrical analysis of regilaul texts}
Alongside the limited availability of digital recordings of the songs is a text database of regilaul song lyrics \url{https://www.folklore.ee/regilaul/andmebaas/?ln=en}. These texts contain the lyrical and musical transcriptions made by the linguists and ethnomusicologists who collected the songs, and by scholars who have contributed later. There are often different versions of the same songs in different regions, with different lyrics or different melodies. My first query of this database will be to analyze the frequency of stress and ictus conflicts in the larger pool of songs, to see if the asymmetry noted in this smaller set bears out at scale. 
 For example, are there more occurrences of syllables strong at the word level ending up in weak metrical positions, or more weak word-level syllables in strong metrical positions? How often do overlong (Q3, heaviest possible) syllables end up in metrically weak positions, if at all? Another simple exploratory analysis could incorporate frequency information with respect to the collective lexicon of all regilaul.  
 
